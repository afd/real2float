%-----------------------------------------------------------------------------
%
%               Template for sigplanconf LaTeX Class
%
% Name:         sigplanconf-template.tex
%
% Purpose:      A template for sigplanconf.cls, which is a LaTeX 2e class
%               file for SIGPLAN conference proceedings.
%
% Guide:        Refer to "Author's Guide to the ACM SIGPLAN Class,"
%               sigplanconf-guide.pdf
%
% Author:       Paul C. Anagnostopoulos
%               Windfall Software
%               978 371-2316
%               paul@windfall.com
%
% Created:      15 February 2005
%
%-----------------------------------------------------------------------------


\documentclass[preprint]{sigplanconf}

% The following \documentclass options may be useful:

% preprint      Remove this option only once the paper is in final form.
% 10pt          To set in 10-point type instead of 9-point.
% 11pt          To set in 11-point type instead of 9-point.
% authoryear    To obtain author/year citation style instead of numeric.
\usepackage{times}
\usepackage{listings}
\def\lstlanguagefiles{defManOcaml.tex}
\lstset{language = Ocaml}
\newcommand{\code}[1]{\lstinline{#1}}
%\begin{lstlisting}\end{lstlisting}

\usepackage[utf8]{inputenc}
\usepackage[T1]{fontenc}
\usepackage{lmodern}
\usepackage{graphicx}  % for pdf, bitmapped graphics files
\usepackage{amsmath} % assumes amsmath package installed
\usepackage{amssymb}  % assumes amsmath package installed
\usepackage{color}
\usepackage{myalgo}
\let\proof\relax
\let\endproof\relax
\usepackage{amsthm}
%\usepackage{ntheorem}
\usepackage[draft]{hyperref}
\usepackage{subfigure}
\usepackage{enumerate}
%\usepackage{caption}
\usepackage{multirow}
\usepackage{tikz, pgfplots}
\pgfplotsset{compat=1.8}
%\usepackage{natbib} %for bibliography via bibtex
%\usepackage{enumitem}
%\lstMakeShortInline{$}
\newcommand{\add}[1]{#1}
\newcommand{\del}[1]{\textcolor{gray}{#1}}

%\newcommand{\P}{\mathbb{P}}
%\newcommand{\Q}{\mathbb{Q}}
%\newcommand{\E}{\mathbb{E}}
\def\sizefig{0.35}
\def\sizesmallfig{0.30}
\def\sizetinyfig{0.24}
\newcommand{\setA}{\mathcal{A}} % Semialgebraic functions
\newcommand{\setD}{\mathcal{D}} % Dictionnary of Univariate transcendental functions
\newcommand{\setU}{\mathcal{U}} % Univariate functions := \setT \cup {sqrt, abs, power functions}
\newcommand{\suppf}[1]{\text{supp}(#1)}
\newcommand{\mons}[2]{\N_{#1}^{#2}}
\newcommand{\R}{\mathbb{R}}
\newcommand{\F}{\mathbb{F}}
\newcommand{\Pcal}{\mathcal{P}}
\newcommand{\Pcalp}{\Pcal^{(p)}}
\newcommand{\Scal}{\mathcal{S}}
\newcommand{\N}{\mathbb{N}}
\newcommand{\x}{\mathbf{x}}
\newcommand{\e}{\mathbf{e}}
\newcommand{\y}{\mathbf{y}}
\newcommand{\z}{\mathbf{z}}
\newcommand{\nbenchs}{23}
\newcommand{\alphab}{\boldsymbol{\alpha}}
\newcommand{\epsilonb}{\boldsymbol{\epsilon}}
\newcommand{\deltab}{\boldsymbol{\delta}}
\renewcommand{\b}{\mathbf{b}}
\newcommand{\f}{\mathbf{f}} 
\newcommand{\Plam}{\P_{\lambda}}
\newcommand{\Plamp}{\P_{\lambda}^{(p)}}
\def\P{\mathbf{P}}
\def\Q{\mathbf{Q}}
\def\L{\mathbf{L}}
\def\D{\mathbf{D}}
\def\q{\mathbf{q}}
\newcommand{\M}{\mathbf{M}}
\def\m{\mathbf{m}}
\def\H{\mathbf{H}}
\def\h{\mathbf{h}}
\def\f{f}
\def\a{\mathbf{a}}
\def\m{\mathbf{m}}
\def\p{\mathbf{p}}
\def\S{\mathbf{S}}
\def\B{\mathbf{B}}
\def\E{\mathbf{E}}
\def\K{\mathbf{K}}
\def\S{\mathbf{S}}
\def\Q{\mathbf{Q}}
\def\X{\mathbf{X}}
\def\Y{\mathbf{Y}}
\newcommand{\A}{\mathbf{A}}
\newcommand{\Shat}{\hat{\S}}
\newcommand{\Bb}{\mathbf{B}}
\newcommand{\flam}{{f}_{{\lambda}}}
\newcommand{\flamfun}[1]{f_{\lambda}(#1)}
\newcommand{\flamfunp}[2]{f^{(#2)}_{\lambda}(#1)}
\newcommand{\flamx}{\flamfun{\x}}
\newcommand{\flampx}{\flamfunp{\x}{p}}
\newcommand{\flamstar}{f^*({\lambda})}
\newcommand{\flamstarj}[1]{f_{#1}^*({\lambda})}
\newcommand{\Mcal}{\mathcal{M}}
\newcommand{\nsdp}{n_{\text{sdp}}}
\newcommand{\divzero}{\text{Div0}}
\newcommand{\nlift}{n_{\text{lift}}}
\newcommand{\Slift}{\S_{\text{lift}}}
\renewcommand{\prec}{\text{prec}}
\newcommand{\mlift}{m_{\text{lift}}}
\newcommand{\dlift}{r_{\text{lift}}}
\newcommand{\msdp}{m_{\text{sdp}}}
\newcommand{\xlamstar}{{\x}^*({\lambda})}
\newcommand{\Kpol}{\K_{\text{poly}}}
\newcommand{\transpose}{\top}%\newcommand{\transpose}{\mathbf{\intercal}}
\DeclareMathOperator{\vol}{vol}
%\DeclareMathOperator{\op}{op}
\DeclareMathOperator{\conv}{conv}
\newcommand{\red}[1]{\textbf{{\color{red}#1}}}
\newcommand{\iaboundfun}[2]{\mathtt{ia\_bound}(#1, #2)}
\newcommand{\iabound}{\mathtt{ia\_bound}}
\newcommand{\sdpboundfun}[3]{\mathtt{sdp\_bound}(#1, #2, #3)}
\newcommand{\sdpbound}{\mathtt{sdp\_bound}}
\newcommand{\boundfun}[7]{\mathtt{bound}(#1, #2, #3, #4, #5, #6, #7)}
\newcommand{\bound}{\mathtt{bound}}
\newcommand{\boundnlprogfun}[7]{\mathtt{bound\_nlprog}(#1, #2, #3, #4, #5, #6, #7)}
\newcommand{\boundnlprog}{\mathtt{bound\_nlprog}}
\newcommand{\sdppolyfun}[3]{\mathtt{sdp\_poly}(#1, #2, #3)}
\newcommand{\sdppoly}{\mathtt{sdp\_poly}}
\newcommand{\liftfun}[3]{\mathtt{lift}(#1, #2, #3)}
\newcommand{\lift}{\mathtt{lift}}
\newcommand{\poly}{_\text{poly}}
\newcommand{\sa}{_\text{sa}}
\newcommand{\sdpsafun}[3]{\mathtt{sdp\_sa}(#1, #2, #3)}
\newcommand{\sdpsa}{\mathtt{sdp\_sa}}
\newcommand{\sdptranscfun}[3]{\mathtt{sdp\_trancs}(#1, #2, #3)}
\newcommand{\sdptransc}{\mathtt{sdp\_transc}}
%\newcommand{\brev}{\color{red}}
%\newcommand{\erev}{\color{black}}
\newcommand{\sthreefp}{\mathtt{s3fp}}
\newcommand{\realtofloat}{\mathtt{Real2Float}}
\newcommand{\hol}{\text{\sc Hol-light}}
\newcommand{\op}{\mathtt{op}}
\newcommand{\bop}{\mathtt{bop}}
\newcommand{\coq}{\text{\sc Coq}}
\newcommand{\ocaml}{\text{\sc OCaml}}
\newcommand{\rosa}{\mathtt{Rosa}}
\newcommand{\sdpa}{\text{\sc Sdpa}}
\newcommand{\fptaylor}{\mathtt{FPTaylor}}
\newcommand{\nlcertify}{\mathtt{NLCertify}}
\makeatletter
\newcommand*{\circled}{\@ifstar\circledstar\circlednostar}
\newcommand*{\squared}{\@ifstar\squaredstar\squarednostar}
\makeatother

\newcommand*\circledstar[1]{%
  \tikz[baseline=(C.base)]
    \node[%
      fill,
      circle,
      minimum size=1.em,
      text=white,
%      font=\sffamily,
      inner sep=0.5pt
    ](C) {\texttt{#1}};%
}
\newcommand*\circlednostar[1]{%
  \tikz[baseline=(C.base)]
    \node[%
      draw,
      circle,
      minimum size=1.em,
%      font=\sffamily,
      inner sep=0.5pt
    ](C) {\texttt{#1}};%
}
\newcommand*\squaredstar[1]{%
  \tikz[baseline=(C.base)]
    \node[%
      fill,
      rectangle,
      minimum size=1.em,
      text=white,
%      font=\sffamily,
      inner sep=0.5pt
    ](C) {\texttt{#1}};%
}
\newcommand*\squarednostar[1]{%
  \tikz[baseline=(C.base)]
    \node[%
      draw,
      rectangle,
      minimum size=1.em,
%      font=\sffamily,
      inner sep=0.5pt
    ](C) {\texttt{#1}};%
}
%\newcommand{\II}{\mathbb{I}}
%\theoremstyle{plain}
%\newtheorem{thm}{Theorem}[section]
\newtheorem{theorem}{Theorem}[section]
\newtheorem{lemma}[theorem]{Lemma}
\newtheorem{proposition}[theorem]{Proposition}
\newtheorem{corollary}[theorem]{Corollary}

\theoremstyle{plain}
\newtheorem{definition}[theorem]{Definition}
\newtheorem{hypothesis}[theorem]{Assumption}
\newtheorem{conjecture}[theorem]{Conjecture}
\newtheorem{assumption}[theorem]{Assumption}
\newtheorem{example}{Example}

%\theoremstyle{remark}
\newtheorem{remark}{Remark}
%\newtheorem*{note}{Note}
\newtheorem{case}{Case}




\begin{document}

\special{papersize=8.5in,11in}
\setlength{\pdfpageheight}{\paperheight}
\setlength{\pdfpagewidth}{\paperwidth}

\conferenceinfo{CONF 'yy}{Month d--d, 20yy, City, ST, Country} 
\copyrightyear{20yy} 
\copyrightdata{978-1-nnnn-nnnn-n/yy/mm} 
\doi{nnnnnnn.nnnnnnn}

% Uncomment one of the following two, if you are not going for the 
% traditional copyright transfer agreement.

%\exclusivelicense                % ACM gets exclusive license to publish, 
                                  % you retain copyright

%\permissiontopublish             % ACM gets nonexclusive license to publish
                                  % (paid open-access papers, 
                                  % short abstracts)

%\titlebanner{banner above paper title}        % These are ignored unless
%\preprintfooter{short description of paper}   % 'preprint' option specified.

\title{Automated Precision Tuning using Semidefinite Programming}
%\subtitle{Subtitle Text, if any}

\authorinfo{Name1}
           {Affiliation1}
           {Email1}
\authorinfo{Name2\and Name3}
           {Affiliation2/3}
           {Email2/3}

\maketitle

\begin{abstract}

Rounding errors cannot be avoided when implementing numerical programs with finite precision.         
The ability to reason about rounding is especially important if one wants to explore a range of potential representations, for instance in the world of FPGAs. This problem becomes challenging when the program does not employ solely linear operations as non-linearities are inherent to many interesting computational problems in real-world applications. 

Existing solutions to reasoning are limited in presence of nonlinear correlations between variables, leading to either imprecise bounds or high analysis time. Furthermore, while it is easy to implement a straightforward method such as interval arithmetic, sophisticated techniques are less straightforward to implement in a formal setting. Thus there is a need for methods which output certificates that can be formally  validated inside a proof assistant.

We present a framework to provide upper bounds of absolute rounding errors. This framework is based on optimization techniques employing  semidefinite programming and sums of squares certificates, which can be formally checked inside the Coq theorem prover.
Our tool covers a wide range of nonlinear programs, including polynomials and transcendental operations as well as conditional loops.                                                             We illustrate the efficiency and  precision of this tool on non-trivial programs coming from biology, optimization and space control.
\end{abstract}

\category{CR-number}{subcategory}{third-level}

% general terms are not compulsory anymore, 
% you may leave them out
%\terms
%term1, term2

\keywords
hardware precision tuning; round-off error; numerical accuracy; floating-point arithmetic; fixed-precision arithmetic; semidefinite programming; sums of squares; correlation sparsity pattern.
\section{Introduction} %3p
\label{sec:intro}
%
Constructing numerical programs which perform accurate computation turns out to be difficult, due to finite numerical precision of implementations such as floating-point or fixed-point representations. Finite-precision numbers induce rounding errors,  and knowledge of the range of these rounding errors is required to fulfil safety criteria of critical programs, as typically arising in modern embedded systems such as aircraft controllers. Such a knowledge can be used in general for developing accurate numerical software, but appears to be particularly relevant while considering algorithms migration onto reconfigurable hardware (e.g. FPGAs). The advantage of architectures based on FPGAs is that they allow more flexible choices, rather than choosing either for IEEE standard single or double precision. Indeed, in this case, we benefit from a more flexible number representation while ensuring guaranteed bounds on the program output. 

To obtain lower bounds over rounding errors, one can rely on testing approaches, such as meta-heuristic search~\cite{Borges12Test} or under-approximation tools (e.g.~$\sthreefp$~\cite{Chiang14s3fp}). Here, we are interested in handling efficiently the complementary over-approximation problem, namely to obtain precise upper bounds over the error. This problem boils down to finding tight abstractions of linearities or non-linearities while being able to bound the resulting approximations in an efficient way.  
%
For computer programs consisting of linear operations, automatic error analysis can be obtained with well-studied optimization techniques based on SAT/SMT solvers~\cite{hgbk2012fmcad}, affine arithmetic~\cite{fluctuat}. However, non-linear operations are key to many interesting computational problems arising in physics, biology, controller implementations and global optimization. 
Recently, two promising frameworks have been designed to provide upper bounds for rounding errors of nonlinear programs. The corresponding algorithms rely on Taylor-interval methods~\cite{fptaylor15}, implemented in the $\fptaylor$ tool, and on combining SMT with affine arithmetic~\cite{Darulova14Popl}, implemented in the $\rosa$ real compiler.

The common drawback of these two frameworks is that they do not fully take into account the correlations between program variables. Thus they may output coarse error bounds or perform analysis within a large amount of time.  The $\rosa$ tool does not provide any formal guarantees. To the best of our knowledge, the $\fptaylor$ software is the only academic tool which can produce formal proof certificates. This is based on the framework developed in~\cite{SolovyevH13} to verify nonlinear inequalities in $\hol$~\cite{hollight} using Taylor-interval methods. However, most of computation performed in the informal optimization procedure end up being redone inside the $\hol$ proof assistant, yielding a formal verification which is computationally demanding.
%The formal verification of these certificates is computationally expensive. This follows from the fact that most of computation performed in the informal optimization procedure end up being redone inside the proof assistant (see for instance in~\cite{SolovyevH13} the formal verification of nonlinear inequalities in $\hol$~\cite{hollight} using Taylor-interval methods).

The aim of this work is to provide a formal framework to perform automated precision analysis of computer programs that manipulate finite-precision data using nonlinear operators. For such programs, guarantees can be provided with certified programming techniques.
Semidefinite programming (SDP) is relevant to a wide range of mathematical fields, including combinatorial optimization, control theory and matrix completion. In 2001, Lasserre introduced a hierarchy of SDP relaxations~\cite{Lasserre01moments} for approximating polynomial infima. Our method to bound the error is a decision procedure based on an specialized variant of Lasserre hierarchy~\cite{Las06SparseSOS}. The procedure relies on SDP to provide sparse sum-of-squares decompositions of positive polynomials. This framework handles polynomial program analysis (involving the operations $+,\times,-$) but can be extended to the more general class of semialgebraic or transcendental programs (involving $\sqrtsign, /, \min, \max, \arctan, \exp$), following the approximation scheme described in~\cite{Magron15sdp}.

\subsection{Overview of our Method}
%
We present an overview of our method and of the capabilities of related techniques, using an example.
Consider a program implementing the following polynomial expression $f$:
\begin{align*}
f(\x) := x_2 \times x_5 + x_3 \times x_6 - x_2 \times x_3  - x_5 \times x_6 \\
+ x_1 \times ( - x_1 +  x_2 +  x_3  - x_4 +  x_5 +  x_6) \,,
\end{align*}
%
where the six-variable vector $\x :=  (x_1, x_2, x_3, x_4, x_5, x_6)$ is the input of the program. 
This function $f$ is involved in many inequalities arising from the the proof of Kepler Conjecture~\cite{halesalgo}, yielding challenging global optimization problems.
For this example, the set $\X$ of possible input values is a product of closed intervals: $\X = [4.00, 6.36]^6$.
%but could be defined in general with a set of inequality constraints among the variables $x_1, \dots, x_6$. 

The polynomial expression $f$ is obtained by performing 15 basic operations (1 negation, 3 subtractions, 6 additions and 5 multiplications). 
When executing this program with a set of floating-point numbers $\hat{\x} :=  (\hat{x}_1, \hat{x}_2, \hat{x}_3, \hat{x}_4, \hat{x}_5, \hat{x}_6) \in \X$, one actually computes a floating-point result $\hat{f}$, where all operations $+, -, \times$ are replaced by the respectively associated floating-point operations $\oplus, \ominus, \otimes$. 
The results of these operations comply with IEEE 754 standard arithmetic~\cite{IEEE} (see relevant background in Section~\ref{sec:fpbackground}). For instance, one can write $\hat{x}_2 \otimes \hat{x}_5 =  (x_2 \times x_5) (1 + e_1)$, by introducing an error variable $e_1$ such that $-\epsilon \leq e_1 \leq \epsilon$, where the bound $\epsilon$ is the machine precision (e.g.~$\epsilon = 2^{-24}$ for single precision). One would like to bound the absolute rounding error $|r(\x, \e)| := | \hat{f}(\x, \e) - f (\x) |$ over  all possible input variables $\x \in \X$ and error variables $e_1, \dots, e_{15} \in [-\epsilon, \epsilon]$. Let us define $\E := [-\epsilon, \epsilon]^{15}$ and $\K := \X \times \E$, then our bound problem can be cast as finding the maximum $r^\star$ of $\mid r \mid$ over $\K$, yielding the following nonlinear optimization problem:
%
\begin{align}
\begin{split}
\label{eq:roptim}
r^\star := & \max_{(\x, \e) \in \K} | r(\x, \e) | \\
 = & \ \ \max \{-\min_{(\x, \e) \in \K} r(\x, \e), \max_{(\x, \e) \in \K} r(\x,\e)\} \enspace,
\end{split}
\end{align}
%
One can directly try to solve these two polynomial optimization problems using classical SDP relaxations~\cite{Lasserre01moments}.
As in~\cite{fptaylor15}, one can also decompose the error term $r$ as the sum of a term $l(\x,\e)$, which is affine w.r.t.~$\e$, and a nonlinear term $h(\x,\e) := r(\x,\e) - l(\x,\e)$. Then the triangular inequality yields:
%
\begin{equation}
\label{eq:lhoptim} 
r^\star \leq \max_{(\x, \e) \in \K} |l(\x, \e)| + \max_{(\x, \e) \in \K} |h(\x, \e)| \enspace. 
\end{equation}
%
It follows for this example that $l(\x,\e) = x_2 x_5 e_1 + x_3 x_6 e_2 +  (x_2 x_5 + x_3 x_6) e_3 + \dots + f(\x) e_{15} = \sum_{i=1}^{15} s_i(\x) e_i$. The {\em Symbolic Taylor expansions} method~\cite{fptaylor15} consists of using Taylor-interval optimization to compute a rigorous interval enclosure of each polynomial $s_i$, $i = 1,\dots,15$, over $\X$ and finally obtain an upper bound of $|l| + |h|$ over $\K$. Our method uses sparse semidefinite relaxations for polynomial optimization (derived from \cite{Las06SparseSOS}) to bound $l$ as well as basic interval arithmetic to bound $h$. The following results have been obtained on an Intel Core i5 CPU ($2.40\, $GHz).
% over $\X \times \E$
\begin{itemize}
\item A direct attempt to solve the two polynomial problems occurring in Equation~\eqref{eq:roptim} fails as the SDP solver (in our case $\sdpa$~\cite{sdpa7}) runs out of memory. 
\item Using our method, one obtains an upper bound of $789 \epsilon$ for $|l| + |h|$ over $\K$ in less than one second.
\item Using basic interval arithmetic, one obtains 17.1 times more quickly a coarser bound of $2023 \epsilon$. 
\item Symbolic Taylor expansions implemented in the $\fptaylor$ tool~\cite{fptaylor15} provide an intermediate bound of $936 \epsilon$ but 16.3 times slower than with our implementation. 
\item Finally, our bound is also obtained with the $\rosa$ real compiler~\cite{Darulova14Popl} but 4.6 times slower than with our implementation.
\end{itemize}

\if{
For instance, given a four-variable vector $\x := (x_1, x_2, x_3, x_4)$, consider the  determinant computation of the $2 \times 2$ matrix  $\begin{pmatrix}
x_1 & x_2 \\
x_3 & x_4 \\
\end{pmatrix}$.
%
Then, the polynomial $f$ which represents the absolute floating-point error is given by $f(\x, \epsilonb) := [x_1 x_4 (1 + \epsilon_1) - x_3 x_2 (1 + \epsilon_2)] (1 + \epsilon_3) - x_1 x_4 + x_3 x_2$.
This polynomial $f$ is already of degree 3 and involves 7 variables.
}\fi




\subsection{Contributions}
Our key contributions can be summarized as follows:
\begin{itemize}
\item We present an optimization algorithm providing certified over-approximations for round-off errors of nonlinear programs. This algorithm is based on sparse sums of squares programming~\cite{Las06SparseSOS}. By comparison with other methods, our algorithm allows us to obtain tighter upper bounds, while overcoming scalability and numerical issues inherent in SDP solvers~\cite{Todd01semidefiniteoptimization}. Our algorithm can currently handle  programs implementing polynomial functions, but also involving non-polynomial components, including either semialgebraic or transcendental operations (e.g. $/, \sqrtsign, \arctan, \exp$), as well as conditional loops.  Programs containing iterative or while loops are not currently supported.
%\item 
\item Our framework is fully implemented in the $\realtofloat$ tool.  Among several features, the tool can optionally perform formal verification of round-off error bounds for polynomial programs, inside the $\coq$ proof assistant~\cite{CoqProofAssistant}. The last software release of $\realtofloat$ provides $\ocaml$~\cite{OCaml} and $\coq$ libraries and is freely available\footnote{\url{https://forge.ocamlcore.org/frs/download.php/1519/real2float.tar.gz}}.
%\begin{center}
%
%\end{center}
%
Our implementation tool is built in top of the $\nlcertify$ verification system~\cite{icms14}. Precision and efficiency of the tool are evaluated on several benchmarks coming from the existing literature. Numerical experiments demonstrate that our method competes well with recent approaches relying on Taylor-interval approximations~\cite{fptaylor15} or combining SMT solvers with affine arithmetic~\cite{Darulova14Popl}.
\end{itemize}
%


The paper is organized as follows.
%
In Section~\ref{sec:background}, we recall mandatory background on rounding errors due to finite precision arithmetic before describing our nonlinear program semantics (Section~\ref{sec:fpbackground}). Then we explain how to perform certified polynomial optimization based on semidefinite programming (Section~\ref{sec:sdpbackground}) and finally how to obtain formal bounds while checking the certificates inside the $\coq$ proof assistant (Section~\ref{sec:coqbackground}).
%
Section~\ref{sec:fpsdp} contains the main contribution of the paper, namely how to compute tight over-approximations for rounding errors of nonlinear programs with sparse semidefinite relaxations.
%
Finally, Section~\ref{sec:benchs} is devoted to the evaluation of our nonlinear verification tool $\realtofloat$ on benchmarks arising from control systems, optimization, physics and biology.

%Verifying simple linear algebra algorithms can be troublesome from the  computational point of view.
\section{Preliminaries}
\label{sec:background}

\subsection{Program Semantics and Floating-point Arithmetic}
\label{sec:fpbackground}
We adopt the standard practice~\cite{higham2002accuracy} to approximate a real float $x$ with its closest floating-point representation $\hat{x} = x (1 + e)$, with $|e|$ being less than the machine precision $\epsilon$. This model is only valid when neglecting both overflow and denormal range values.
The operator $\hat{\cdot}$ is called the rounding operator and can be selected among rounding to nearest, rounding toward zero (resp.~$\infty$).
The scientific notation of a binary (resp.~decimal) floating-point number $\hat{x}$ is a triple $(s, sig, exp)$ consisting of a sign bit $s$, a {\em significand} $sig \in [1, 2)$ (resp.~$[1, 10)$) and an {\em exponent} $exp$, so that its numerical evaluation yields $(-1)^{s} \times sig \times 2^{exp}$ (resp.~$(-1)^{s} \times sig \times 10^{exp}$). 

The value of $\epsilon$ actually gives the upper bound of the relative floating-point error and is equal to $2^{-\prec}$, where $\prec$ is called the {\em precision}, referring to the number of significand bits used. For single precision floating-point, one has $\prec = 24$. For double (resp.~quadruple) precision, one has $\prec = 53$ (resp.~$\prec=113$). Let us define $\R$ the set of real numbers and $\F$ the set of binary floating-point numbers.
For each real-valued operation $\bop_\R \in \{+, -, \times, \slash \}$ complying with IEEE 754 standard arithmetic~\cite{IEEE}, the result of the corresponding floating-point operation $\bop_\F \in \{\oplus, \ominus, \otimes, \oslash \}$ satisfies the following:
\begin{equation}
\label{eq:roundbop}
\bop_\F \, (\hat{x}, \hat{y}) = \bop_\R \, (x, y) \, (1 + e) \enspace, \quad \mid e \mid \leq \epsilon = 2^{-\prec} \enspace.
\end{equation}
%
Other operations include special functions taken from a {\em dictionary} $\setD$, containing the unary functions
$\tan$, $\arctan$, $\cos$, $\arccos$, $\sin$, $\arcsin$, $\exp$, $\log$, $(\cdot)^{r}$ with $r\in \R\setminus\{0\}$. For each $f_\R \in \setD$, the corresponding floating-point evaluation satisfies 
\begin{equation}
\label{eq:roundtransc}
f_\F (\hat{x}) = f_\R (x) (1 + e) \enspace, \quad \mid e \mid \leq \epsilon (f_\R) \enspace.
\end{equation}
%$f_\F (\hat{x}) = f_\R (x) (1 + e)$, with $|e| \leq \epsilon (f_\R)$. 
The value of the relative error bound $\epsilon (f_\R)$ differs from the machine precision $\epsilon$ in Equation~\eqref{eq:roundbop} and has to be properly adjusted. We refer the interested reader to~\cite{VerifCADTransc} for relative error bound verification of transcendental functions (see also~\cite{VerifHOLTransc} for formalization in $\hol$).
%
%In a similar fashion, 
\subsubsection*{\textit{Program semantics}}
%
We consider generic programs encoded in an ML-like language:
\begin{lstlisting}
let box_prog    $x_1 \dots x_n = [(a_1, b_1); \dots ; (a_n, b_n)]$;;
let obj_prog    $x_1 \dots x_n = [(f(\x), \epsilon_{\text{total}})]$;;
let cstr_prog   $x_1 \dots x_n = [g_1 (\x), \dots, g_k(\x)]$;;
let uncert_prog $x_1 \dots x_n = [u_1, \dots, u_n]$;;
\end{lstlisting}
Here, the first line encodes interval constraints for input variables, namely $\x := (x_1, \dots, x_n) \in [a_1, b_1]\times \dots \times [a_n, b_n]$.
The second line provides the function $f(\x)$ as well as the total rounding error bound $\epsilon_{\text{total}}$.
Then, one encodes polynomial nonnegativity constraints over the input variables, namely $g_1(\x) \geq 0, \dots, g_k(\x) \geq 0$. Finally, the last line allows the user to specify a numerical constant $u_i$ to associate a given uncertainty to the variable $x_i$, for each $i= 1, \dots, n$.

The type of numerical constants is denoted by \code{C}. In our current implementation, the user can choose either 64 bits floating-point or arbitrary-size rational numbers. This type \code{C} is used for the terms $\epsilon_{\text{total}}$, $u_1, \dots, u_n$, $a_1, \dots, a_n$, $b_1, \dots, b_n$.

The inductive type of polynomial expressions with coefficients in \code{C} is \code{polC} defined as follows:
\begin{lstlisting}
type polC = | Pc of C | Px of positive 
| Psub of polC$\,$*$\,$polC | Pneg of polC 
| Padd of polC$\,$*$\,$polC | Pmul of polC$\,$*$\,$polC
\end{lstlisting}
%
The constructor \code{Px} takes a positive integer as argument to represent either an input or local variable.
%The polynomial expressions $g_1(\x), \dots, g_k(\x)$ have this type \code{polC}.
The inductive type \code{nlexpr} of nonlinear expressions (such as $f(\x)$) is defined as follows:
\begin{lstlisting}
type nlexpr = 
| Pol of polC | Neg of nlexpr
| Add of nlexpr$\,$*$\,$nlexpr 
| Mul of nlexpr$\,$*$\,$nlexpr 
| Sub of nlexpr$\,$*$\,$nlexpr 
| Div of nlexpr$\,$*$\,$nlexpr | Sqrt of nlexpr 
| Transc of transc$\,$*$\,$nlexpr
| IfThenElse of polC$\,$*$\,$nlexpr$\,$*$\,$nlexpr
| Let of positive$\,$*$\,$nlexpr$\,$*$\,$nlexpr
\end{lstlisting}
%
The type \code{transc} corresponds to the dictionary $\setD$ of special functions. For instance, the term~\lstinline|Transc ($\exp$, $f(\x)$)| represents the program implementing $\exp(f(\x))$.
Given a polynomial expression $p$ and two nonlinear expressions $f$ and $g$, the term ~\lstinline|IfThenElse($p(\x)$, $f(\x)$, $g(\x)$)| represents the conditional program implementing~\lstinline|if ($p(\x) \geq 0$) $f (\x)$ else $g (\x)$|. The constructor \code{Let} allows us to define local variables in an ML fashion, e.g.~\lstinline|let $t_1 = 331.4 + 0.6 * T$ in $-t_1 * v /((t_1 + u) * (t_1 + u))$| (part of the \textit{doppler1} program considered in Section~\ref{sec:benchs}).
%
%\end{lstlisting}
%

Finally, one obtains rounded nonlinear expressions using an inductive procedure~\lstinline|rounding : nlexpr $\to$ nlexpr|, defined accordingly to Equation~\eqref{eq:roundbop} and Equation~\eqref{eq:roundtransc}. When an uncertainty $u_i$ is specified for an input variable $x_i$, the corresponding rounded expression is given by $x_i \, (1 + e)$, with $\mid e \mid \, \leq u_i$.
%to when solving optimization problems involving maximal absolute rounding errors. allowing to consider a single error variable bounded using $(k + 1) \epsilon$, thus saving $(k - 1)$ error variables.

%One can exploit sparsity in a way similar to the one described in~\cite{Waki06SparseSOS,Las06SparseSOS} to handle high dimensional problems.
\subsection{Sparse semidefinite relaxations for polynomial optimization}
\label{sec:sdpbackground}
Here, we recall mandatory background about the method that we use to handle the optimization problem of Equation~\eqref{eq:roptim}, when the nonlinear function $r$ is a polynomial expression.

\subsubsection*{\textit{Sums of squares certificates and semidefinite programming}}
We note basic facts about generation of sums of squares certificates for polynomial optimization, using semidefinite programming.
Denote by $\R[\x]$ the vector space of polynomials and by $\R_{2 d}[\x]$ the restriction of $\R[\x]$ to polynomials of degree at most $2 d$. Let us define the set of sums of squares:
\begin{equation}
\label{eq:cone_sos}
\Sigma[\x] := \Bigl\{\sum_i q_i^2, \, \text{ with } q_i \in \R[\x] \Bigr\}\enspace,
\end{equation}
%
as well as its restriction $\Sigma_{2 d}[\x] := \Sigma[\x] \bigcap \R_{2 d}[\x]$ to polynomials of degree at most $2 d$. For instance, the following bivariate polynomial  $\sigma (\x) := 1 + (x_1^2 - x_2^2)^2$ lies in $\Sigma_4[\x] \subseteq \R_4[\x]$.

At some point, optimization methods based on sums of squares use the implication $p \in \Sigma[\x] \implies \forall \x \in \R^n, \, p(\x) \geq 0$, i.e. the inclusion of $\Sigma[\x]$ in the set of nonnegative polynomials.

Given $r \in \R[\x]$, one considers the following polynomial minimization problem:
\begin{equation}
\label{eq:minpop}
r^*  :=  \inf_{\x \in \R^n} \, \{ \, r (\x) \, : \, \x \in \K \, \} \enspace,
\end{equation}
%
where the set of constraints $\K \subseteq \R^n$ is defined by
%
\[\K := \{ \x \in \R^{n} : g_1 (\x) \geq 0, \dots, g_k (\x) \geq 0\}\enspace,\]
for polynomial functions $g_1, \dots, g_k$. The set $\K$ is called a {\em basic semialgebraic} set. Membership to semialgebraic sets is ensured by satisfying conjunctions of polynomial nonnegativity constraints. 
%
\begin{remark}
\label{rk:arch}
 When the input variables satisfy interval constraints $\x \in [a_1, b_1] \times \dots \times [a_n, b_n]$ then one can easily show that there exists some integer $M > 0$ such that $M - \sum_{i=1}^n x_i^2 \geq 0$. 
In the sequel, we assume that this nonnegativity constraint appears explicitly in the definition of $\K$. Such an assumption is mandatory to prove the convergence of semidefinite relaxations stated in Theorem~\ref{th:densesdp}.
\end{remark}
%
In general, the objective function $r$ and the set of constraints $\K$ can be nonconvex, which makes the resolution of Problem~\eqref{eq:minpop} difficult to solve in practice. 
Note that one can rewrite Problem~\eqref{eq:maxpop} as the equivalent maximization problem:
\begin{equation}
\label{eq:maxpop}
r^*  :=  \sup_{\x \in \R^n, \mu \in \R} \{ \, \mu \, : \, r (\x) - \mu \geq 0 \,, \ \x \in \K \, \} \,.
\end{equation}
%
Given a nonnegative polynomial $p \in \R[\x]$, the existence of a sums of squares decomposition $p = \sum_i q_i^2$ valid
over $\R^n$ is ensured by the existence of a symmetric real matrix $Q$, solution of the following linear matrix feasibility problem:
\begin{align}
\label{eq:sdp}
p(\x) = \m_d(\x)^\intercal \, \Q \, \m_d(\x) \,, \quad \forall \x \in \R^n, \,
\end{align}
%
where $\m_d(\x) := (1, x_1, \dots, x_n, x_1^2,x_1 x_2,\dots, x_n^d)$ and the matrix $\Q$ has only nonnegative eigenvalues. Such a matrix $\Q$ is called {\em positive semidefinite}. The vector $\m_d$ and matrix $\Q$ have both a size equal to $s_n^d := \binom{n + d}{d}$. Problem~\eqref{eq:sdp} can be handled with semidefinite programming (SDP) solvers, such as {\sc Mosek}~\cite{mosek} or {\sc SDPA}~\cite{sdpa7} (see~\cite{Vandenberghe94SDP} for specific background about SDP). Then, one computes the ``LDL'' decomposition $\Q = \L^\intercal \D \L$ (variant of the classical Cholesky decomposition), where $\L$ is a lower triangular matrix and $\D$ is a diagonal matrix. Finally, one obtains $p(\x) =  (\L \,
\m_d(\x))^\intercal \, \D \, (\L \, \m_d(\x)) = \sum_{i=0}^{s_n^d} q_i(\x)^2$. Such a decomposition is called a sums of squares (SOS) {\em certificate}.
%
\begin{example}
\label{ex:sdp}
Let us define $p(\x) := \frac{1}{4} + x_1^4 - 2 x_1^2 x_2^2 + x_2^4$. With $\m_2 (\x) = (1, x_1, x_2, x_1^2, x_1 x_2, x_2^2)$, one solves the linear matrix feasibility problem $p(\x) = \m_2 (\x)^\intercal \, \Q \, \m_2(\x)$. One can show that the solution writes $\Q = \L^\intercal \D \L$ for a $6 \times 6$ matrix $\L$ and a diagonal matrix $\D$ with entries $(\frac{1}{2},0,0,1,0,0)$, yielding the SOS decomposition: $p(\x) = (\frac{1}{2})^2 + (x_1^2 - x_2^2)^2 =: \sigma(\x)$. It is enough to prove that $p$ is nonnegative.
\end{example}
%
\subsubsection*{\textit{Dense SDP relaxations for polynomial optimization}}
We first explain how to obtain tractable appoximations of Problem~\eqref{eq:maxpop}. Define $g_0 := 1$. The hierarchy of SDP relaxations developed by Lasserre \cite{Lasserre01moments} provides lower bounds of $r^*$, through solving the following optimization problems $(\P_d)$:
\[
(\P_d):\left\{			
\begin{array}{rlr}
p_d^\star := \sup\limits_{\sigma_j, \mu} & \mu \enspace, \\			 
\text{s.t.} & r (\x) - \mu = \sum_{j = 0}^{k} \sigma_j(\x) g_j(\x) \,, \ \forall \x \,, \\
\\
& \mu\in \R \,, \sigma_j \in \Sigma[\x] \,, \quad \ \, \quad  \ j = 0,\dots,k \,, \\
\\
& \deg (\sigma_j g_j) \leq  2 d,             \quad \ \, \, \qquad  j = 0,\dots,k \,.\\
\end{array} \right.
\]
%
The next theorem is a consequence of the assumption mentioned in Remark~\ref{rk:arch}.
\begin{theorem}[Lasserre~\cite{Lasserre01moments}]
\label{th:densesdp}
Let $p_d^{\star}$ be the optimal value of the sparse SDP relaxation~$(\P_d)$.
Then, the sequence of optimal values $(p_d^\star)_{d \in \N}$ is nondecreasing and converges to $r^\star$.
\end{theorem}
%
The size of the truncated SDP variables
grows polynomially with the SDP-relaxation order $d$.
Indeed, at fixed $n$, the relaxation~$(\P_d)$ involves $O((2 d)^{n})$ SDP
variables and $(k + 1)$ linear matrix inequalities (LMIs) of size
$O(d^n)$. When $d$ increases, then more accurate lower bounds of $r^\star$ can be obtained, at an increasing computational cost.
At fixed $d$,  the relaxation $(\P_d)$ involves $O(n^{2d})$ SDP variables and $(d + 1)$ linear matrix inequalities (LMIs) of size
$O(n^{d})$.

\if{
There are several ways to decrease the size of the SDP problems. 
First, symmetries in SDP relaxations for polynomial optimization problems can be exploited to replace one SDP problem~$(\P_d)$ by
several smaller SDPs~\cite{Riener2013SymmetricSDP}. Notice that it is possible only if the multivariate polynomials of the initial problem are invariant under the action of a finite subgroup $G$ of the group $GL_{n}(\R)$. 
}\fi
%

\subsubsection*{\textit{Exploiting sparsity}} 
Here we remind how to exploit the structured sparsity of the
problem to replace one SDP problem~$(\P_d)$ by an SDP problem~$(\S_d)$ of
size $O (\kappa^ {2 d})$ where $\kappa$ is the average size
of the maximal cliques of the correlation pattern of the polynomial
variables (see~\cite{Waki06SparseSOS,Las06SparseSOS} for more details). We now present these notions as well as the formulation of sparse SDP relaxations~$(\S_d)$.

We note $\N^n$ the set of $n$-tuple of nonnegative integers. The support of a polynomial $r(\x) := \sum_{\alphab \in \N^n} r_{\alphab} \x^{\alphab}$ is defined as $\suppf{r} := \{ \, \alphab \in \N^n \, : \, r_{\alphab} \neq 0 \, \}$. For instance the support of $p(\x) := \frac{1}{4} + x_1^4 - 2 x_1^2 x_2^2 + x_2^4$ is $\suppf{p} = \{ \, (0,0), (4, 0), (2,2), (0,4) \, \}$.

Let $F_j$ be the index set of variables which are involved in the polynomial $g_j$, for each $j=1, \dots, k$.
The correlative sparsity is represented by the 
$n \times n$ correlative sparsity matrix (csp matrix) $\mathbf{R}$ defined by:
\begin{equation*}
\label{eq:csp}
\mathbf{R}(i, j) := \left \{
\begin{array}{ll}
  1 & \text{ if }  i = j \enspace, \\
  1 & \text{ if }  \exists \alphab \in \suppf{f} \text{ such that } \alpha_i, \alpha_j \geq 1 \,, \\
  1 & \text{ if }  \exists k \in \{1, \dots, m\} \text{ such that } i, j \in F_k  \,,\\
  0 & \text{otherwise .} 
\end{array} \right.
\end{equation*}

We define the undirected csp graph $G(N, E)$ with
 $N = \{ 1, \dots, n \}$ and $E = \{\{i, j\} : i, j \in N , \ i < j , \mathbf{R}(i, j) = 1 \}$. 
Then, let $C_1,\dots, C_m \subseteq N$ denote the maximal cliques of $G(N, E)$ and 
%and define the sets of supports: 
%\[\mons{d}{C_q} := \{ \alphab \in  \mons{d}{n} : \alpha_i = 0 \text{ if } i %\notin C_q \}, \ (q=1 ,\dots,l)\enspace. \]
 define $n_j := \#C_j$, for each $j=1 ,\dots,m$.

\begin{remark}
\label{rk:sparsearch}
From the assumption of Remark~\ref{rk:arch}, one can add the $m$ redundant additional constraints:
\begin{equation}
\label{eq:assum_sos_sparse}
g_{k + j} := n_j M^2 - \sum_{i \in C_j} {x_i^2} \geq 0\,, \  j=1 ,\dots, m \,,
\end{equation}
set $k' = k + m$, define the compact semialgebraic set:
\[\K' := \{\, \x \in \R^n \, : \, g_1 (\x) \geq 0, \dots, g_{k'} (\x) \geq 0 \,\} \,,\]
and modify Problem~\eqref{eq:minpop} into the following optimization problem:
\begin{equation}
\label{eq:sparseminpop}
r^*  :=  \inf_{\x \in \R^n} \, \{ \, r (\x) \, : \, \x \in \K' \, \} \,.
\end{equation}
\end{remark}
%Notice that the sums of squares of polynomials that belong to $\Sigma [\x, \mons{d}{C_q}]$ only involve variables $x_i (i \in C_q)$.
For each $j=1 ,\dots,m$, we note $\R_{2 d}[\x, C_j]$ the set of polynomials of $\R_{2 d}[\x]$ which involve the variables $(x_i)_{i \in C_j}$. We note $\Sigma [\x, C_j] := \Sigma [\x] \bigcap \R_{2 d}[\x, C_j]$. Similarly, we define $\Sigma [\x, F_j]$, for each $j=1, \dots, k$.

The following program is the sparse variant of the SDP program $(\P_d)$:
\[
(\S_d):\left\{			
\begin{array}{rl}
r_d^\star := \sup\limits_{\mu, \sigma_j} & \mu\enspace, \\	 
\text{s.t.} & r (\x) - \mu = \sum_{j = 0}^{k'} \sigma_j(\x) g_j(\x) \,, \ \forall \x \,, \\
\\
& \mu\in \R \,,\  \sigma_0 \in \sum_{j = 1}^m \Sigma [\x, C_j] \,, \\
\\
& \sigma_j \in \Sigma[\x, F_j]  \,,\ j = 1,\dots,k' \,, \\
\\
& \deg (\sigma_j g_j) \leq 2 d  \,,\ j = 0,\dots,k' \,,
\end{array} \right.
\]
%
where $\sigma \in \sum_{j = 1}^m \Sigma [\x, C_j]$ if and only if there exist $\sigma^1 \in \Sigma[\x, C_1], \dots, \sigma^m \in \Sigma[\x,C_m]$ such that $\sigma (\x) = \sum_{j = 1}^m \sigma^j (\x)$, for all $\x \in \R^n$.
%

The number of SDP variables of the relaxation~$(\S_d)$ is $\sum_{j=1}^m \binom{n_j + 2 d}{2 d}$. At fixed $d$, it yields an SDP problem with $O(\kappa^{2d})$ variables, where $\kappa := \frac{1}{m} \sum_{j=1}^m n_j$ is the average size of the cliques $C_1, \dots, C_m$.
%
Moreover, the cliques $C_1, \dots, C_m$ satisfy the running intersection property: 
%
\begin{definition}[RIP]
\label{def:rip}
Let $m \in \N_0$  and $I_1, \dots, I_m$ be subsets of $\{1, \dots, n\}$. We say that $I_1, \dots, I_m$ satisfy the running intersection property (RIP) when for all $i=1, \dots, m$, there exists an integer $k < i$ such that $I_i \cap (\cup_{j < i} I_j) \subseteq I_k$.
%\[ \forall i = 2, \dots, r, (\exists k < i \,, \ (I_i \bigcap \bigcup_{j < i} I_j) \subset I_k)\enspace. \]
\end{definition}
This RIP property together with the assumption mentioned in Remark~\ref{rk:sparsearch} allow to state the sparse variant of Theorem~\ref{th:densesdp}:
%The optimal values of $\r_d$ converge to the global minimum $f^*$, as a corollary of~\cite[Theorem 3.6]{Las06SparseSOS}.
%
\begin{theorem}[\protect{Lasserre~\cite[Theorem 3.6]{Las06SparseSOS}}]
\label{th:sparsesdp}
Let $r_d^{\star}$ be the optimal value of the sparse SDP relaxation~$(\S_d)$. Then the sequence $(r_d^{\star})_{d \in \N}$ is nondecreasing and converges to $r^\star$.
\end{theorem}
The interested reader can find more details in~\cite{Waki06SparseSOS} about additional ways to exploit sparsity in order to derive analogous sparse SDP relaxations.
We illustrate the benefits of the SDP relaxations~$(\S_d)$ with the following example:
\begin{example}
\label{ex:sparse}
Consider the polynomial $f$ mentioned in Section~\ref{sec:intro}:
$f(\x) := x_2 x_5 + x_3 x_6 - x_2 x_3  - x_5 x_6 
+ x_1 ( - x_1 +  x_2 +  x_3  - x_4 +  x_5 +  x_6)$.
%
Here, $n = 6, d = 2, N = \{1,\dots, 6 \}$. The $6 \times 6$ correlative sparsity matrix $\mathbf{R}$ is:
\[
\mathbf{R} = 
\begin{pmatrix}
  1 & 1 & 1 & 1 & 1 & 1 \\
  1 & 1 & 1 & 0 & 1 & 0 \\
  1 & 1 & 1 & 0 & 1 & 1 \\
  1 & 0 & 0 & 1 & 0 & 0 \\
  1 & 1 & 1 & 0 & 1 & 1 \\
  1 & 0 & 1 & 0 & 1 & 1 
 \end{pmatrix}
\]
The csp graph $G$ associated to $\mathbf{R}$ is depicted in Figure~\ref{fig:csp_deltax}. 
%
\begin{figure}[!ht]	
\begin{center}
\includegraphics[scale=0.6]{csp_deltax.pdf}
\caption{Correlative sparsity pattern graph for the variables of $f$}
\label{fig:csp_deltax}
\end{center}
\end{figure}
%
The maximal cliques of $G$ are $C_1 := \{1, 4\}$, $C_2 := \{1, 2, 3, 5\}$ and $C_3 := \{1, 3, 5, 6\}$. For $d=2$, the dense SDP relaxation~$(\P_2)$ involves $\binom{6 + 4}{4} = 210$ variables against $\binom{2 + 4}{4} + 2 \binom{4 + 4}{4} = 115$ for the sparse variant~$(\S_2)$. The dense SDP relaxation~$(\P_3)$ involving $924$ variables against $448$ for the sparse variant~$(\S_3)$. 
This difference becomes significant while considering that the time complexity of semidefinite programming is polynomial w.r.t. the number of variables with an exponent greater than 3 (see~\cite[Chapter 4]{BenTal01} for more details).
\end{example}
%
\subsection{Computer assisted proofs for polynomial optimization}
\label{sec:coqbackground}
Here, we briefly recall some existing features of the $\coq$ proof assistant to handle formal polynomial optimization.
%about the mechanisms of formal polynomial optimization within the $\coq$ proof assistant.
For more details on the $\coq$ system, we recommend the
documentation available in~\cite{bertot2004interactive}.
Giving a polynomial $r$ and a set of constraints $\K$, one can obtain a lower bound of $r$ by solving any instance of Problem~$(\P_d)$. Then, one can verify formally the correctness of the lower bound $r_d^\star$, using the SOS certificate output $\sigma_0, \dots, \sigma_k$. Indeed it is enough to prove the polynomial equality $r(\x) - r_d^\star = \sum_{j=0}^k \sigma_j(\x) g_j(\x)$ inside $\coq$. Such equalities can be efficiently proved using $\coq$'s ring tactic~\cite{ring05} via the mechanism of computational reflection~\cite{Boutin97usingreflection}.
For the sake of clarity, let consider the unconstrained case, i.e.~$\K = \R^n$. One encodes an SOS certificate $\sigma_0(\x) = \sum_{i=1}^m q_i^2$  with the sequence of polynomials $[q_1; \dots; q_m]$, with each $q_i$ being of type \code{polC} (see Section~\ref{sec:fpbackground}). To prove the equality $r = \sigma_0$, our version of the ring tactic normalizes both $r$ and the sequence $[q_1; \dots; q_m]$ and compares the two normalization results. This mechanism is illustrated in Figure~\ref{fig:reflexion} with the polynomial $p := \frac{1}{4} + x_1^4 - 2 x_1^2 x_2^2 + x_2^4$ (see Example~\ref{ex:sdp}) being encoded by \code{p : polC}  and the polynomials $1/2$ and $x_1^2 - x_2^2$ being encoded respectively by $\mathtt{q_1}$ and $\mathtt{q_2}$. 

\if{
In the general case, one applies a correctness lemma:
\begin{lstlisting}
Lemma correct_pop env $r$ cert_pop $ $:  
g_nonneg env g $\to$ checker g $\fpop$ $\mu_k^-$ cert_pop  $\eqcoq$ true $\to$ 
$\mu_k^-$ $\leq$ $\evalexpr{\fpop}$.
\end{lstlisting}
}\fi
%
\begin{figure}[!ht]
\centering
\includegraphics[scale=0.75]{reflexion.pdf}
\caption{An illustration of computational reflection}	
\label{fig:reflexion}
\end{figure}
%
In the general case, this computational step is done through a \code{checker_sos} procedure which returns a Boolean value. If this value is true, one applies a correctness lemma, whose conclusion yields the nonnegativity of $r - r_d^\star$ over $\K$.
In practice, the SDP solvers are implemented in floating-point precision, thus the above equality between $r - r_d^\star$ and the SOS certificate does not hold. However, following Remark~\ref{rk:arch}, each variable lies in a closed interval, thus one can bound the remainder polynomial $\epsilon(\x) := r(\x) - r_d^\star - \sum_{j=0}^k \sigma_j(\x) g_j(\x)$ using basic interval arithmetic, so that the lower bound $\epsilon^\star$ of $\epsilon$ yields the valid inequality: $\forall \x \in \K, r(\x) \geq r_d^\star + \epsilon^\star$.
For more explanation, we refer the interested reader to the formal framework developed in~\cite{jfr14}. Note that this formal verification remains valid when considering the sparse variant~$(\S_d)$.
%
\section{Guaranteed Roundoff Error Bounds using SDP Relaxations}
\label{sec:fpsdp}
In this section, we present our new algorithm to bound rounding errors of nonlinear transcendental programs, relying on sparse SDP relaxations. After stating our general algorithm (Section~\ref{sec:transcsdp}), we detail how this procedure can handle polynomial programs (Section~\ref{sec:polsdp}) and then present extensions to the non-polynomial case (Section~\ref{sec:nonpolsdp}).

\subsection{The General Optimization Framework}
\label{sec:transcsdp}
%
Here we consider a given program which implements a nonlinear transcendental expression $f$ with input variables $\x$ satisfying a set of constraints $\X$. We assume that  $\X$ is included in a box (i.e.~a product of closed intervals) $[\a, \b] := [a_1, b_1] \times \dots \times [a_n, b_n]$ and that $\X$ is encoded as follows: 
\[ 
\X := \{\, \x \in \R^n \, : \, g_1 (\x) \geq 0, \dots, g_{k} (\x) \geq 0 \,\} \,,
\]
for polynomial functions $g_1, \dots, g_k$. 
%
Then, we denote by $\hat{f}(\x,\e)$ the rounded expression of $f$ after applying the ~\lstinline|rounding| procedure (see Section~\ref{sec:fpbackground}), introducing the introduction of error variables $\e$. 

The algorithm \code{bound}, depicted in Figure~\ref{alg:bound}, takes as input $\x$, $\X$, $f$, $\hat{f}$, $\e$ as well as the set $\E$ of bound constraints over $\e$. Here we assume that our program implementing $f$ does not involve conditionals (this case will be discussed later in Section~\ref{sec:nonpolsdp}). For a given machine $\epsilon$, one has $\E := [-\epsilon, \epsilon]^m$, with $m$ being the number of error variables. This algorithm actually relies on the sparse SDP optimization procedure $(\S_d)$ (see Section~\ref{sec:sdpbackground} for more details), thus~\code{bound} also takes as input a relaxation order $d \in \N$. The algorithm provides as output an interval enclosure $I_d$ of the absolute error $\mid \hat{f}(\x,\e) - f(\x) \mid$ over $\K$. 
From this interval $I_d:= [\underline{f_d}, \overline{f_d}]$, one can compute $f_d := \max \{- \underline{f_d}, \overline{f_d} \}$, which is a sound upper bound of the maximal absolute error $r^\star := \max_{(\x,\e)\in \K} \mid \hat{f}(\x,\e) - f(\x) \mid $.

\begin{figure}[!ht]
\begin{algorithmic}[1]                    
\Require input variables $\x$, input constraints $\X$, nonlinear expression $f$, rounded expression $\hat{f}$, error variables $\e$, error constraints $\E$, relaxation order $d$
\Ensure interval enclosure $I_d$ of the absolute error $\mid \hat{f} - f  \mid$ over $\K := \X \times \E$
\State Define the absolute error $r(\x, \e) := \hat{f}(\x,\e) - f(\x)$ \label{line:r}
\State Compute $l(\x,\e) := r(\x, 0) + \sum_{j=1}^m \frac{\partial r(\x,\e)} {\partial e_j} (\x,0) \, e_j$ \label{line:l}
\State Compute $h := r - l$ \label{line:h}
\State Compute interval bounds for $h$: $I^h := \iaboundfun{h}{\K}$ \label{line:iabound}
\State Compute interval bounds for $l$: $I_d^l := \sdpboundfun{l}{\K}{d}$  \label{line:sdpbound}
\State \Return $I_d := I_d^l + I^h$ 
\end{algorithmic}
\caption{\code{bound}}
\label{alg:bound}
\end{figure}

After computing the absolute rounding error $r := \hat{f} - f$ (Line~\lineref{line:r}), one decomposes $r$ as the sum of an expression $l$ which is affine w.r.t.~the error variable $\e$ and a remainder $h$ (Line~\lineref{line:h}). One way to obtain $l$ is to compute the vector of partial derivatives of $r$ w.r.t.~$\e$ evaluated at $(\x, 0)$ and finally to take the inner product of this vector and $\e$ (Line~\lineref{line:l}). Then, the idea is to compute a precise bound of $l$ and a coarse bound of $h$. The underlying reason is that $h$ involves error term products of degree greater than 2 (e.g.~$e_1 e_2$), yielding an interval enclosure $I^h$ of \textit{a priori} much smaller width, compared to the interval enclosure $I^l$ of $l$. One obtains $I^h$ using the procedure $\iabound$ implementing basic interval arithmetic (Line~\lineref{line:iabound}). 
%
\subsection{Polynomial Programs}
\label{sec:polsdp}
%
We first describe our $\sdpbound$ optimization algorithm when implementing polynomial programs. In this case, $\sdpbound$ calls an auxiliary procedure $\sdppoly$.
The bound of $l$ is provided through solving two sparse SDP instances of Problem~$(\S_d)$, at relaxation order $d$. We now give more explanation about the $\sdppoly$ procedure.

We can map each input variable $x_i$ to the integer $i$, for all $i=1,\dots,n$, as well as each error variable $e_j$ to $n+j$, for all $j=1,\dots,m$. Then, define the sets $C_1 := \{1,\dots,n,n+1\}, \dots, C_m := \{1,\dots,n,n+m\}$. Here, we take advantage of the sparsity correlation pattern of $l$ by using $m$ distinct sets of cardinality $n+1$ rather than a single one of cardinality $n+m$, i.e.~the total number of variables. 
After noticing that $r(\x,0) = \hat{f}(\x,0) - f(\x) = 0$, one can scale the optimization problems by writing 
\begin{align}
\label{eq:lscale}
l(\x,\e) = \sum_{j=1}^m s_j (\x) e_j = \epsilon \sum_{j=1}^m s_j (\x) \frac{e_j}{\epsilon} \,,
\end{align}
%
with $s_j(\x) := \frac{\partial r(\x,\e)} {\partial e_j} (\x,0)$, for all $j=1,\dots,m$. Replacing $\e$ by $\e/\epsilon$ leads to compute an interval enclosure of $l/\epsilon$ over $\K' := \X \times [-1, 1]^m$.
Recall that from Remark~\ref{rk:arch}, there exists an integer $M > 0$ such that $M - \sum_{i=1}^n x_i^2 \geq 0$, as the input variable satisfy box constraints.
Moreover, to fulfil the assumption of Remark~\ref{rk:sparsearch},  one encodes $\K'$ as follows: 
\begin{align*}
\K' := \{\, (\x,\e) \in \R^{n+m} \, : \, g_1 (\x) \geq 0, \dots, g_k(\x) \geq 0 \,, \\
g_{k+1}(\x,e_1) \geq 0, \dots, g_{k+m} (\x, e_m) \geq 0 \,\} \,,
\end{align*}
%
with $g_{k+j}(\x, e_j) := M + 1 -  \sum_{i=1}^n x_i^2 + e_j^2$, for all $j=1,\dots, m$. 
The index set of variables involved in $g_j$ is $F_j := N = \{1, \dots, n\}$ for all $j=1, \dots, k$. 
The index set of variables involved in $g_{k+j}$ is $F_j := C_j$ for all $j=1, \dots, m$. 

Then, one can compute a lower bound of the minimum of $l'(\x,\e) := l(\x, \e) / \epsilon$ over $\K'$ by solving the following optimization problem:
%
\begin{align}
\begin{split}
\label{eq:lscalesdp1}
\left\{			
\begin{array}{rl}
\underline{l_d'} := \sup\limits_{\mu, \sigma_j} & \mu\enspace, \\	 
\text{s.t.} & l' - \mu = \sigma_0 + \sum_{j = 1}^{k+m} \sigma_j g_j \,, \\
\\
& \mu\in \R \,,\ \sigma_0 \in \sum_{j = 1}^m \Sigma [(\x, \e), C_j] \,, \\
\\
& \sigma_j \in \Sigma[(\x,\e), F_j] \,, \ j = 1,\dots,k+m \,, \\
\\
& \deg (\sigma_j g_j) \leq 2 d  \,, \ j = 1,\dots,k+m \,.
\end{array} \right.
\end{split}
\end{align}
%
A feasible solution of Problem~\eqref{eq:lscalesdp1} ensures the existence of $\sigma^1 \in \Sigma[(\x,e_1)], \dots, \sigma^m \in \Sigma[(\x,e_m)]$ such that $\sigma_0 = \sum_{j=0}^m \sigma^j$, allowing the following reformulation:
%
\begin{align}
\begin{split}
\label{eq:lscalesdp2}
\left\{			
\begin{array}{rl}
\underline{l_d'} := \sup\limits_{\mu, \sigma_j} & \mu\enspace, \\	
\text{s.t.} & l' - \mu = \sum_{j=1}^m \sigma^j + \sum_{j = 1}^{k+m} \sigma_j g_j \,, \\
\\
& \mu\in \R \,, \ \sigma_j \in \Sigma[\x] \,, \ j = 1,\dots,m \,, \\
\\
& \sigma^j  \in \Sigma [(\x, e_j)] \,,  \deg (\sigma^j) \leq 2 d  \,, \ j = 1,\dots,m \,, \\
\\
&  \quad \deg (\sigma_j g_j) \leq 2 d  \,, \ j = 1,\dots,k+m \,.
\end{array} \right.
\end{split}
\end{align}
%
An upper bound $\overline{l_d'}$ can be obtained by replacing $\sup$ with $\inf$ and $l' - \mu$ by $\mu - l'$ in Problem~\eqref{eq:lscalesdp2}.
Our optimization procedure $\sdppoly$ computes the lower bound $\underline{l_d'}$ as well as an upper bound $\overline{l_d'}$ of $l'$ over $\K'$ then returns the interval $I_d^l := [\epsilon \, \underline{l_d'}, \epsilon \, \overline{l_d'}] $, which is a sound enclosure of the values of $l$ over $\K$.
%
\begin{remark}
We emphasize two advantages of the decomposition $r := l + h$ and more precisely of the linear dependency of $l$ w.r.t.~$\e$: scalability and robustness to SDP numerical issues.
First, no computation is required to determine the correlation sparsity pattern of $l$, by comparison to the general case. Thus, it becomes much easier to handle the optimization of $l$ with the sparse SDP Problem~\eqref{eq:lscalesdp2} rather than with the corresponding instance of the dense relaxation~$(\P_d)$. While the latter involves $\binom{n + m+ 2 d}{2 d}$ SDP variables, the former involves only $m \, \binom{n + 1 + 2 d}{2 d}$ variables, ensuring the scalability of our framework.

In addition, the linear dependency of $l$ w.r.t.~$\e$ allows us to scale the error variables and optimize over a set of variables lying in $\K' := \X \times [-1, 1]$. It ensures that the range of input variables does not significantly differ from the range of error variables. This condition is mandatory while considering SDP relaxations because most SDP solvers (e.g.~{\sc Mosek}~\cite{mosek}, {\sc SDPA}~\cite{sdpa7}) are implemented using double precision floating-point. It is impossible to optimize $l$ over $\K$ (rather than $l'$ over $\K'$) when the maximal value $\epsilon$ of error variables is less than $2^{-53}$, due to the fact that SDP solvers would treat each error variable term as 0, and consequently $l$ as the zero polynomial. Hence, this decomposition insures our framework from the numerical issues related to finite-precision implementation of SDP solvers.
\end{remark}
Let us define the interval $I^l := [\underline{l}, \overline{l}]$, with $\underline{l} := \inf_{(\x,\e) \in \K} l(\x,\e)$ and $\overline{l} := \sup_{(\x,\e) \in \K} l(\x,\e)$.
The next lemma states that one can approximate $I^l$ as closely as desired using the $\sdppoly$ procedure.
\begin{lemma}[Convergence of the $\sdppoly$ procedure]
\label{th:cvg_sdppoly}
Let $I_d^l$ be the interval enclosure obtained by calling the procedure $\sdppolyfun{l}{\K}{d}$. Then the sequence $(I_d^l)_{d \in \N}$ converges to $I^l$.
\end{lemma}
%
\begin{proof}
It is sufficient to show the similar convergence result for $l' = l/\epsilon$, as it implies the convergence for $l$ by a scaling argument.
The sets $C_1,\dots, C_m$ satisfy the RIP property (see Definition~\ref{def:rip}). Moreover, the encoding of $\K'$ satisfies the assumption mentioned in Remark~\ref{rk:sparsearch}. Thus, Theorem~\ref{th:sparsesdp} implies that the sequence of lower bounds $(\underline{l_d'})_{d \in \N}$ converges to $\underline{l'} := \inf_{(\x,\e) \in \K'} l'(\x,\e)$. Similarly, the sequence of upper bounds converge to $\overline{l'}$, yielding the desired result.
\end{proof}
%
Lemma~\ref{th:cvg_sdppoly} guarantees asymptotic convergence to the exact enclosure of $l$ when the relaxation order $d$ tends to infinity. However, it is more reasonable in practice to keep this order as small as possible to obtain tractable SDP relaxations. Hence, we generically solve each instance of Problem~\eqref{eq:lscalesdp2} at the minimal relaxation order, that is $d_0 := \max \{\lceil \deg l / 2\rceil) , \max_{1 \leq j \leq k+m} \{ \lceil \deg (g_j) / 2\rceil) \} \}$. 
%
\subsection{Non-polynomial and Conditional Programs}
\label{sec:nonpolsdp}
We now present how the general optimization procedure $\sdpbound$ can be extended to other classes of nonlinear programs, including semialgebraic and transcendental functions. Then, we treat the case of programs involving conditional loops.
%
\subsubsection*{\textit{Semialgebraic programs}} 
Here we assume that the function $l$ is semialgebraic, that is involves non-polynomial components such as divisions or square roots.
Following~\cite{LasPut10}, we explain how to transform the optimization problem $\inf_{(\x,\e) \in \K} l (\x, \e)$ into a polynomial optimization problem, so that we can use the sparse SDP program~\eqref{eq:lscalesdp2}. One way to perform this reformulation consists of introducing lifting variables to represent non-polynomial operations.

We first illustrate the extension to semialgebraic programs with the following example.
\begin{example}
Let consider the program implementing the rational function $f : [0, 1] \to \R$ defined by $f(x_1) := \frac{x_1}{1 + x_1}$. Applying the rounding procedure (with machine $\epsilon$) yields $\hat{f}(x_1,\e) := \frac{x _1(1 + e_2)}{(1 + x_1)(1 + e_1)}$ and the decomposition $r(x_1, \e) := \hat{f}(x_1,\e) - f(x_1) = l(x_1,\e) + h(x_1\,e) = s_1 (x_1) e_1 + s_2 (x_1) e_2 + h(x_1,\e)$. One has $s_1(x_1) = \frac{\partial r(x_1,\e)} {\partial e_1} (x_1,0) = -\frac{x_1}{1 + x_1}$ and $s_2(x_1) = - s_1(x_1)$.

Let $\K := [0, 1] \times [-\epsilon, \epsilon]^2$. One introduces a lifting variable $x_2 := \frac{x_1}{1 + x_1}$ to handle the division operator and encode the equality constraint $p(\x) :=  x_2 (1 + x_1) - x_1 = 0$ with the two inequality constraints $p (\x) \geq 0$ and $-p(\x) \geq 0$. To ensure the compactness assumption, one bounds $x_2$ within $I := [0, 1/2]$, using basic interval arithmetic.

Let $\Kpol := \{(\x,\e) \in [0, 1] \times I \times [\epsilon, \epsilon]^2 : p(\x) \geq 0 \,,\  - p(\x) \geq 0 \}$. Then the rational optimization problem involving $l$ is equivalent to $\inf_{(\x,\e) \in \Kpol} x_2 (-e_1 + e_2)$, a polynomial optimization that we can handle with the $\sdppoly$ procedure, described in Section~\ref{sec:polsdp}.
\end{example}
%

In the semialgebraic case, $\sdpbound$ calls an auxiliary procedure $\sdpsa$. 
Given input variables $\y := (\x,\e)$, input constraints $\K := \X \times \E$ and a 
semialgebraic function $l$,  $\sdpsa$ first applies a recursive procedure $\lift$ 
which returns variables $\y\poly$, constraints $\K\poly$ and a polynomial $f\poly$ such that the 
interval enclosure $I^l$ of $l(\y)$ over $\K$ is equal to the interval enclosure of 
the polynomial $l\poly(\y\poly)$ over $\K\poly$. 
Finally, $\sdpsa$ returns the interval enclosure $I^l_d := \sdppolyfun{l\poly}{\K\poly}{d}$. We detail the lifting procedure $\lift$ in Figure~\ref{alg:lift} for the constructors \code{Pol}(Line~\eqref{line:liftpol}), \code{Div} (Line~\eqref{line:liftdiv}) and \code{Sqrt} (Line~\eqref{line:liftsqrt}). 
The interval $I$ obtained through the $\iabound$ procedure (Line~\eqref{line:liftia}) allows us to constrain the additional variable $x$ to ensure the assumption of Remark~\ref{rk:sparsearch}.
For the sake of consistency, we omit the other cases (\code{Neg}, \code{Add}, \code{Mul} and \code{Sub}) where the procedure is straightforward. For a similar procedure in the context of global optimization, we refer the interested reader to~\cite[Chapter 2]{MagronPhD}.

\begin{figure}[!ht]
\begin{algorithmic}[1]                 
\Require input variables $\y$, input constraints $\K$, semialgebraic expression $f$
\Ensure variables $\y\poly$, constraints $\K\poly$, polynomial expression $f\poly$
    \State $I := \iaboundfun{f}{\K}$ \label{line:liftia}
	\If {\lstinline|$f = \ $ Pol ($p$)|} \label{line:liftpol} $\y\poly := \y$, $\K\poly := \K$, $\f\poly := p$
	\ElsIf {\lstinline|$f = \ $ Div ($g$, $h$)|} \label{line:liftdiv} 
	\State $\y_g, \K_g, g\poly := \liftfun{\y}{\K}{g}$ 
	\State $\y_h, \K_h, h\poly := \liftfun{\y}{\K}{h}$
    \State $\y\poly := (\y_g,\y_h,x)$ \hspace{1cm} $f\poly := x$ 
	\State $\K\poly := \{\y\poly \in \K_g \times \K_h \times I : x g\poly = f\poly \}$
	\ElsIf {\lstinline|$f = \ $ Sqrt ($g$)|} \label{line:liftsqrt} 
	\State $\y_g, \K_g, g\poly := \liftfun{\y}{\K}{g}$ 
	\State $\y\poly := (\y_g,x)$ \hspace{1cm} $f\poly := x$ 
	\State $\K\poly := \{\y\poly \in \K_g \times I : x^2 = g\poly \}$\\
\code{...}
	\EndIf
%
\State \Return $\y\poly, \K\poly, f\poly$
\end{algorithmic}
\caption{\code{lift}}
\label{alg:lift}
\end{figure}

Finally, notice that the final set of variables $\y\poly$ can be decomposed as $(\x\poly, \e)$, where $\x\poly$ gathers the input variables together with the lifting variables.  
Denoting by $n\poly$ the cardinality of $\x\poly$, one easily shows that the sets $\{1, \dots, n\poly, e_1\},\dots,\{1, \dots, n\poly, e_m\}$ satisfy the RIP, thus ensuring to solve efficiently the corresponding instances of Problem~\eqref{eq:lscalesdp2}.
%
\subsubsection*{\textit{Transcendental programs}}
Here we assume that the function $l$ is transcendental,~i.e.~involves univariate non-semialgebraic components such as $\exp$ or $\sin$. For each univariate transcendental function $f_{\R}$ in our dictionary set $\setD$, one assumes that $f_{\R}$ is twice differentiable, so that the univariate function $g := f_{\R} + \frac{\gamma}{2} |\cdot|^2$ is convex on $I$ for large enough $\gamma > 0$ (for more details, see the references~\cite{agk04, mceneaney-livre}). It follows that there exists a constant $\gamma \leq \sup_{x\in I} -f_{\R}''(x)$ such that for all $x_i \in I$:
\begin{align}
\begin{split}
\label{eq:maxplus}
\forall x \in I, \quad f_{\R} (x)  \geq f_{x_i}^-(x) \,,\\
\text{with } f_{x_i}^- :=  -\frac{\gamma}{2} (x-x_i)^2 +f_{\R}'(x_i) (x - x_i) + f_{\R} (x_i) \,,
\end{split}
\end{align}
implying that for all $x \in I$, $f_{\R} (x)  \geq \max_{x_i \in I} f_{x_i}^-(x)$. Similarly, one obtains an upper-approximation $\min_{x_i \in I} f_{x_i}^+(x)$.
Figure~\ref{fig:logexp} provides such approximations for the function $f_{\R}(x) := \log (1 + \exp(x))$ on the interval $I := [-8, 8]$.
%
\begin{figure}[!ht]
\begin{center}
\includegraphics[scale=0.8]{logexp.pdf}
\caption{Maxplus Approximations for $x \mapsto \log(1 + \exp{x})$: $\max \{ f_0^- (x), f_8^- (x)\} \leq  \log(1 + \exp{x}) \leq \min \{ f_0^+ (x), f_8^+ (x)\} $}\label{fig:logexp}
\end{center}
\end{figure}
%

For transcendental programs, our procedure $\sdpbound$ calls the auxiliary procedure $\sdptransc$. Given input variables $(\x,\e)$, constraints $\K$ and a transcendental function $l$, $\sdptransc$ first computes a semialgebraic lower (resp.~upper)  approximation $l^-$ (resp.~$l^+$) of $l$ over $\K$. For more details in the context of global optimization, we refer the reader to~\cite{Magron15sdp}. Then, calling the procedure $\sdpsa$ allows us to get interval enclosures of $l^-$ as well as $l^+$.
We illustrate the procedure to handle transcendental programs with the following example.
\begin{example}
\label{ex:logexp}
Let consider the program implementing the transcendental function $f : [-8, 8] \to \R$ defined by $f(x_1) := \log (1 + \exp(x_1))$. Applying the rounding procedure  yields $\hat{f}(x_1,\e) := \log [(1 + \exp(x_1) (1 + e_1)) \, (1 + e_2)](1 + e_3)$. 
Here, $|e_2|$ is bounded by the machine $\epsilon$ while $|e_1|$ (resp.~$|e_3|$) are bounded with an adjusted absolute error $\epsilon_1 := \epsilon(\exp)$ (resp.~$\epsilon_3 := \epsilon(\log)$).
Let $\K:= [-8,8] \times [-\epsilon_1, \epsilon_1] \times [\epsilon, \epsilon] \times [-\epsilon_3, \epsilon_3]$.

One obtains the decomposition $r(x_1, \e) := \hat{f}(x_1,\e) - f(x_1) = l(x_1,\e) + h(x_1,\e) = s_1 (x_1) e_1 + s_2 (x_1) e_2 + s_3 (x_1) e_3 + h(x_1, \e)$, with 
$s_1(x_1) = \frac{\exp(x_1)} {1 + \exp(x_1)}$, $s_2(x_1) = 1$ and $s_3(x_1) = \log (1 + \exp(x_1)) = f(x_1)$. Figure~\ref{fig:logexp} provides a lower approximation $s_3^- := \max\{f_0^-,f_8^-\}$ of $s_3$ as well as an upper approximation $s_3^+ := \min \{f_0^+,f_8^+\}$. One can get similar approximations $s_1^-$ and $s_1^+$ for $s_1$. 

One first obtains (coarse) interval enclosures $I_2 = \iaboundfun{s_1}{\K}$ and $I_3 = \iaboundfun{s_3}{\K}$ and one introduces extra variables $x_2 \in I_2$ and $x_3 \in I_3$ to represent $s_1$ and $s_3$ respectively.
Then, the interval enclosure of $l$ over $\K$ is equal to the interval enclosure of $l\sa(\x,\e) := x_2 e_1 + e_2 + x_3 e_3$ over the set $\K\sa:= \{(x_1,\e) \in \K \,, (x_2, x_3) \in I_2 \times I_3 \,, s_1^-(x_1) \leq x_2 \leq s_1^+(x_1) \,, s_3^-(x_1) \leq x_3 \leq s_3^+(x_1) \}$.
\end{example}
%
\subsubsection*{\textit{Programs with conditionals}}
Finally, we explain how to extend our bounding procedure to nonlinear programs involving conditionals through the recursive algorithm given in Figure~\ref{alg:bound_nlprog}.
The $\boundnlprog$ algorithm relies on the $\bound$ procedure (see Figure~\ref{alg:bound} in Section~\ref{sec:transcsdp}) to compute rounding error bounds of programs implementing transcendental functions (Line~\lineref{line:noncnd}).
From Line~\lineref{line:cnd} to Line~\lineref{line:endcnd}, the algorithm handles the case when the program implements a function $f$ defined as follows:
\[   
f (\x) := 
     \begin{cases}
       g(\x) &\text{if } p(\x) \geq 0,\\
       h(\x) &\text{otherwise}.
     \end{cases}
\]
The first branch output is $g$ while the second one is $h$.

A preliminary step consists of computing the rounding error enclosure $I_d^p := [\underline{p_d}, \overline{p_d}]$ (Line~\lineref{line:polcnd}) for the program implementing the polynomial $p$. 
Then the procedure computes bounds related to the divergence path error, that is the maximal value between the four following errors: 
\begin{itemize}
\item (Line~\lineref{line:I1}) the error obtained while computing the rounded result $\hat{h}$ of the second branch instead of computing the exact result $g$ of the first one, occurring for the set of variables $(\x,\e)$ such that $\hat{p}(\x,\e) \leq 0 \leq p(\x)$. For scalability and numerical issues, we consider an over-approximation $\X_1$ (Line~\lineref{line:X1}) of this set, where the variables $\x$ satisfy the relaxed constraints $0 \leq p(\x) \leq \overline{p_d}$.
\item (Line~\lineref{line:I2}) the error obtained while computing the rounded result $\hat{g}$ of the first branch instead of computing the exact result $h$ of the second one, occurring for the set of variables $(\x,\e)$ such that $p(\x) \leq 0 \leq \hat{p}(\x,\e)$. We also consider an over-approximation $\X_2$ (Line~\lineref{line:X2}) of this set, where the variables $\x$ satisfy the relaxed constraints $\underline{p_d} \leq p(\x) \leq 0$.
\item (Line~\lineref{line:I3}) the rounding error corresponding to the program implementation of $g$.
\item(Line~\lineref{line:I4}) the rounding error corresponding to the program implementation of $h$.
\end{itemize}
%
\begin{figure}[!ht]
\begin{algorithmic}[1]
\Require input variables $\x$, input constraints $\X$, nonlinear expression $f$, rounded expression $\hat{f}$, error variables $\e$, error constraints $\E$, relaxation order $d$
\Ensure interval enclosure $I_d$ of the error $\mid \hat{f} - f  \mid$ over $\K := \X \times \E$
%
\If{\lstinline|$f = \ $ IfThenElse ($p, g, h$)|} \label{line:cnd}
\State $I_d^p := \boundfun{\x}{\X}{p}{\hat{p}}{\e}{\E}{d} = [\underline{p_d}, \overline{p_d}]$ \label{line:polcnd}
\State $\X_1 := \{ \x \in \X : 0 \leq p(\x) \leq \overline{p_d} \}$ \label{line:X1}
\State $\X_2 := \{ \x \in \X : \underline{p_d} \leq p(\x) \leq 0 \}$\label{line:X2}
\State $\X_3 := \{ \x \in \X : 0 \leq p(\x) \}$\label{line:X3}
\State $\X_4 := \{ \x \in \X : p(\x) \leq 0 \}$\label{line:X4}
\State $I_d^1 := \boundnlprogfun{\x}{\X_1}{g}{\hat{h}}{\e}{\E}{d}$\label{line:I1}
\State $I_d^2 := \boundnlprogfun{\x}{\X_2}{h}{\hat{g}}{\e}{\E}{d}$\label{line:I2}
\State $I_d^3 := \boundnlprogfun{\x}{\X_3}{g}{\hat{g}}{\e}{\E}{d}$\label{line:I3}
\State $I_d^4 := \boundnlprogfun{\x}{\X_4}{h}{\hat{h}}{\e}{\E}{d}$\label{line:I4}
\State \Return $I_d := I_d^1 \cup I_d^2 \cup I_d^3 \cup I_d^4$ \label{line:endcnd}
\Else \State \Return $I_d := \boundfun{\x}{\X}{f}{\hat{f}}{\e}{\E}{d}$ \label{line:noncnd}
\EndIf
%
\end{algorithmic}
\caption{\code{bound_nlprog}}
\label{alg:bound_nlprog}
\end{figure}
%
\subsubsection*{\textit{Simplification of error terms}}
%
In addition, our algorithm \code{bound_nlprog} integrates several features to reduce the number of error variables. First, it memorizes all sub-expressions of the nonlinear expression tree to perform common sub-expressions elimination. 
%associate  the same error variable to the same operation when the operands are identical. 
%For instance, several occurrences of $x_1 + x_2$ are replaced each time by $(x_1 + x_2) (1 + e_1)$.
We can also simplify error term products, thanks to the following lemma.
\begin{lemma}[\protect{Higham~\cite[Lemma 3.3]{higham2002accuracy}}]
\label{th:redproduct}
Let $\epsilon < \frac{1}{k}$ and $\gamma_k := \frac{k \epsilon}{1 - k \epsilon}$. Then, for all $e_1, \dots, e_k \in [-\epsilon, \epsilon]$, there exists $\theta_k$ such that ${\displaystyle \prod_{i=1}^k (1 + e_i) = 1 + \theta_k}$ and $\mid \theta_k \mid \leq \gamma_k$.
\end{lemma}
Lemma~\ref{th:redproduct} implies that for any $k$ such that $\epsilon < \frac{1}{k}$, one has $\gamma_k \leq (k + 1) \epsilon$. Our algorithm has an option to automatically derive safe over-approximations of the absolute rounding error while introducing only one variable $e_1$ (bounded by $(k + 1) \epsilon$) instead of $k$ error variables $e_1, \dots, e_k$ (bounded by $\epsilon$). The cost of solving the corresponding optimization problem can be significantly reduced but it yields coarser error bounds.
%
\section{Implementation Benchmarks}
\label{sec:benchs}
%
Now, we present experimental results obtained by applying our general $\boundnlprog$ algorithm (see Section~\ref{sec:fpsdp}, Figure~\ref{alg:bound_nlprog}) to various examples coming from physics, biology, space control and optimization. 
The  $\boundnlprog$ algorithm is implemented in a tool called $\realtofloat$, built in top of the $\nlcertify$ nonlinear verification package, relying on $\ocaml$ (Version $4.02.1$) and $\coq$ (Version $8.4\text{pl}5$) and interfaced with the SDP solver $\sdpa$ (Version $7.3.6$). The SDP solver output numerical SOS certificates, which are converted into rational SOS using the {\sc Zarith} $\ocaml$ library (Version $1.2$), implementing arithmetic operations over arbitrary-precision integers.
For more details about the installation and usage of $\realtofloat$, we refer to the dedicated web-page of $\nlcertify$\footnote{\url{http://nl-certify.forge.ocamlcore.org}} and the user manual of $\realtofloat$\footnote{see the \texttt{README.md} file in the top level directory}. 
All examples are displayed in Appendix A as the corresponding $\realtofloat$ input text files and satisfy our nonlinear program semantics (see Section~\ref{sec:fpbackground}). Results have been obtained on an Intel Core i5 CPU ($2.40\, $GHz).
%
\subsection{Benchmark Presentation}
%For each problem presented in Table~\ref{table:error}, 
For each example, we compared the quality of the rounding error bounds (Table~\ref{table:error}) and corresponding execution times (Table~\ref{table:cpu}) while running our tool $\realtofloat$, $\fptaylor$ (version from May $2015$)~\cite{fptaylor15} and $\rosa$ (version from  May $2014$)~\cite{Darulova14Popl}.
A given program implements a nonlinear function $f(\x)$, involving variables $\x$ lying in a set $\X$ contained in a box $[\a, \b]$.
Applying our rounding model on $f$ yields the nonlinear expression $\hat{f}(\x,\e)$, involving additional error variables $\e$ lying in a set $\E$. 
At a given semidefinite relaxation order $d$, our tool computes the upper bound $f_d$ of the absolute rounding $\mid f - \hat{f} \mid $ over $\K := \X \times \E$ and verifies that it is less than a requested number $\epsilon_{\text{total}}$. As we keep the relaxation order $d$ as low as possible to ensure tractable SDP programs, it can happen that $f_d > \epsilon_{\text{total}}$. 
 In this case, we subdivide a randomly chosen interval of the box  $[\a, \b]$ in two halves to obtain two sub-sets $\X_1$ and $\X_2$, fulfilling $\X := \X_1 \cup \X_2$, and apply the $\boundnlprog$ algorithm on both sub-sets until we succeed to certify that $\epsilon_{\text{total}}$ is a sound upper bound of the rounding error.

The number $\epsilon_{\text{total}}$ is compared with the upper bounds computed by $\fptaylor$ which relies on Taylor Symbolic expansions~\cite{fptaylor15}, $\rosa$ which relies on SMT and affine arithmetic~\cite{Darulova14Popl}, as well as a third procedure IA relying on interval arithmetic. We designed IA to follow the same steps than $\boundnlprog$ together with the sub-procedure $\bound$ but to compute an interval enclosure of $l$ using basic interval arithmetic instead of calling the $\sdpbound$ procedure (see Line~\lineref{line:sdpbound} of the algorithm depicted in Figure~\ref{alg:bound}). For comparison purpose, we also ran simulation tests through instantiation of input variables to random inputs, yielding lower bounds of the absolute error (the technique is the same as in the $\rosa$ paper~\cite{Darulova14Popl}).
%
\begin{table*}[!ht]
%\small
\begin{center}
\caption{Comparison results of upper and lower bounds for absolute rounding errors (the best results are emphasized using \textbf{bold fonts})}
\begin{tabular}{p{2.3cm}ccccccc}
\hline
\multirow{1}{*}{Benchmark} &  \multirow{1}{*}{\texttt{id}} & \multirow{1}{*}{precision} & \multirow{1}{*}{$\realtofloat$} & $\rosa$  & $\fptaylor$  &\multirow{1}{*}{IA} & \multirow{1}{*}{lower bound} \\
%& & & & \cite{Darulova14Popl} & \cite{fptaylor15} & & \\
\hline  
\\      
\multicolumn{8}{l}{Programs involving polynomial and semialgebraic functions}    \\
\\
\hline
%\multirow{2}{*}{Method 1} & $n^{(1)}$/$m^{(1)}$ &  $40/30$ & $212/111$ & $1039/350 $ & $4211/915$ & $130768/1991$ & $40251/3822$ \\
\multirow{1}{*}{\texttt{doppler1}} & \texttt{a}
& (double) & $2.75\text{e--}13$ & $4.97\text{e--}13$ & $\mathbf{1.57\textbf{e--}13}$ & $\divzero$ & $7.11\text{e--}14$\\
\multirow{1}{*}{\texttt{doppler2}} & \texttt{b}
& (double) & $8.04\text{e--}13$ & $1.29\text{e--}12$ & $\mathbf{2.87\textbf{e--}13}$ & $\divzero$ & $1.14\text{e--}13$\\
\multirow{1}{*}{\texttt{doppler3}} & \texttt{c}
& (double) & $1.37\text{e--}13$ & $2.03\text{e--}13$ & $\mathbf{8.16\textbf{e--}14}$ & $\divzero$ & $4.27\text{e--}14$\\
%\multirow{1}{*}{doppler1$\star$}
%& (double) & $5.05\text{e--}05$ & $2.36\text{e--}06$ & $6.40\text{e--}07$ & $3.95\text{e+}02$ & $5.97\text{e--}07$\\
\multirow{1}{*}{\texttt{rigidBody1}} & \texttt{d}
& (double) & $\mathbf{3.80\textbf{e--}13}$ & $5.08\text{e--}13$ & $3.87\text{e--}13$ & $\mathbf{3.80\textbf{e--}13}$ & $2.28\text{e--}13$\\
\multirow{1}{*}{\texttt{rigidBody2}} & \texttt{e}
& (double) & $\mathbf{3.98\textbf{e--}11}$ & $6.48\text{e--}11$ & $5.24\text{e--}11$ & $\mathbf{3.98\textbf{e--}11}$ & $2.19\text{e--}11$\\
\multirow{1}{*}{\texttt{verhulst}} & \texttt{f}
& (double) & $\mathbf{3.40\textbf{e--}16}$ & $6.82\text{e--}16$ & $3.50\text{e--}16$ & $6.22\text{e--}01$ & $2.23\text{e--}16$\\
\multirow{1}{*}{\texttt{carbonGas}} & \texttt{g}
& (double) & $\mathbf{1.10\textbf{e--}08}$ & $1.60\text{e--}08$ & $1.25\text{e--}08$ & $3.10\text{e--}03$ & $4.11\text{e--}09$ \\
%\multirow{1}{*}{jetEngine} & (double) & ? & ? & ? & ? & ? \\
\multirow{1}{*}{\texttt{predPrey}} & \texttt{h}
& (double) & $\mathbf{1.57\textbf{e--}16}$ & $1.98\text{e--}16$ & $1.87\text{e--}16$ & $2.02\text{e--}16$ & $1.47\text{e--}16$ \\
\multirow{1}{*}{\texttt{kepler0}} & \texttt{i}
& (double) & $\mathbf{8.76\textbf{e--}14}$ & $9.07\text{e--}14$ & $1.04\text{e--}13$ & $1.03\text{e--}13$ & $2.23\text{e--}14$\\
\multirow{1}{*}{\texttt{kepler1}} & \texttt{j}
& (double) & $\mathbf{2.96\textbf{e--}13}$ & $4.43\text{e--}13$ & $4.49\text{e--}13$ & $6.52\text{e--}13$ & $7.58\text{e--}14$\\
\multirow{1}{*}{\texttt{kepler2}} & \texttt{k}
& (double) & $\mathbf{1.90\textbf{e--}12}$ & $2.16\text{e--}12$ & $2.08\text{e--}12$ & $3.00\text{e--}12$ & $3.03\text{e--}13$\\
\hline
\multirow{2}{*}{\texttt{sineTaylor}} & \multirow{2}{*}{\texttt{l}}
& (double) & $\mathbf{5.53\textbf{e--}16}$ & $9.57\text{e--}16$ & $6.71\text{e--}16$ & $9.39\text{e--}16$ & $4.45\text{e--}16$\\
& & (float) & $\mathbf{2.97\textbf{e--}07}$ & $1.03\text{e--}06$ & $3.51\text{e--}07$ & $5.07\text{e--}07$ & $1.79\text{e--}07$\\
\hline 
\multirow{2}{*}{\texttt{sineOrder3}} & \multirow{2}{*}{\texttt{m}}
& (double) & $\mathbf{6.68\textbf{e--}16}$ & $1.11\text{e--}15$ & $9.96\text{e--}16$ & $8.82\text{e--}16$ & $3.34\text{e--}16$\\
& & (float) & $\mathbf{3.58\textbf{e--}07}$ & $1.19\text{e--}06$ & $5.35\text{e--}07$ & $4.74\text{e--}07$ & $2.12\text{e--}07$\\
\hline
\multirow{2}{*}{\texttt{sqroot}} & \multirow{2}{*}{\texttt{n}}
& (double) & $\mathbf{7.56\textbf{e--}16}$ & $8.41\text{e--}16$ & $7.87\text{e--}16$ & $8.48\text{e--}16$ & $4.45\text{e--}16$\\
& & (float) & $\mathbf{4.06\textbf{e--}07}$ & $9.03\text{e--}07$ & $4.23\text{e--}07$ & $4.56\text{e--}07$ & $2.45\text{e--}07$\\
%& (16bit) & $3.33\text{e--}3$ & $5.97\text{e--}4$ & ? & $1.58\text{e--}4$\\
\hline
\\
\multicolumn{8}{l}{Programs implementing polynomial functions with polynomial preconditions}\\
\\
\hline
\multirow{1}{*}{\texttt{floudas1}} & \texttt{o}
& (double) & $\mathbf{2.99\textbf{e--}13}$ & $\mathbf{2.99\textbf{e--}13}$ & $3.90\text{e--}13$ & $6.36\text{e--}13$ & $1.48\text{e--}13$ \\
\multirow{1}{*}{\texttt{floudas2}} & \texttt{p}
& (double) &  $\mathbf{9.03\textbf{e--}16}$ & $1.12\text{e--}15$ & $9.96\text{e--}16$ & $1.12\text{e--}15$ & $2.35\text{e--}16$ \\
\multirow{1}{*}{\texttt{floudas3}} & \texttt{q}
& (double) & $\mathbf{8.90\textbf{e--}15}$ & $1.00\text{e--}14$ & $1.16\text{e--}14$ & $1.87\text{e--}14$ & $7.31\text{e--}15$ \\
\hline
\\
\multicolumn{8}{l}{Programs implementing transcendental functions}\\
\\
\hline
\multirow{1}{*}{\texttt{logexp}} & \texttt{r}
& (double) & $\mathbf{1.65\textbf{e--}15}$ & $-$ & $1.71\text{e--}15$ & $8.29\text{e--}13$ & $1.19\text{e--}15$ \\
\multirow{1}{*}{\texttt{sphere}} & \texttt{s}
& (double) & $\mathbf{7.78\textbf{e--}15}$ & $-$ & $1.29\text{e--}14$ & $\mathbf{7.78\textbf{e--}15}$ & $5.05\text{e--}15$ \\
%\multirow{1}{*}{azimuth} %& (double) & ? & $-$ & $1.41\text{e--}14$ & ? & ? \\
\multirow{1}{*}{\texttt{hartman3}} & \texttt{t}
& (double) & $3.85\text{e--}13$ & $-$  & $\mathbf{1.34\textbf{e--}14}$ & $3.46\text{e+}05$ & $1.10\text{e--}14$ \\
\multirow{1}{*}{\texttt{hartman6}} & \texttt{u}
& (double) & $\mathbf{4.48\textbf{e--}13}$ & $-$ & $\text{OoM}$ & $2.82\text{e+}03$ & $6.50\text{e--}14$ \\
\hline
\\
\multicolumn{8}{l}{Programs involving conditional loops}    \\
\\
\hline
\multirow{1}{*}{\texttt{cav10}} & \texttt{v}
& (double) & $\mathbf{2.91\textbf{e+}00}$ & $\mathbf{2.91\textbf{e+}00}$ & $-$ & $1.02\text{e+}02$ & $2.90\text{e+}00$ \\
\multirow{1}{*}{\texttt{perin}} & \texttt{w}
& (double) & $\mathbf{2.01\textbf{e+}00}$ & $\mathbf{2.01\textbf{e+}00}$ & $-$ & $4.91\text{e+}01$ & $2.00\text{e+}00$ \\
%\multirow{1}{*}{?} & \texttt{x}
%& (double) & ? & ? & $-$ & ? & ? \\
\hline
\end{tabular}
\label{table:error}
\end{center}
\end{table*}
%
For the sake of further presentation, we associate an alphabet character (from \code{a} to \code{w}) to identify each of the $\nbenchs$ nonlinear programs:
%
\begin{itemize}
\item The first $14$ programs implement polynomial and semialgebraic functions: \code{a-e} come from physics, \code{f} and \code{h} from biology, \code{g} from control, \code{i-k} are derived from expressions involved in the proof of Kepler Conjecture (for more details, see~\cite{halesalgo} and the related formalisation project~\cite{Flyspeck06}) and \code{l-n} implement polynomial approximations of the sine and square root functions. With the exception of \code{i-k}, all these programs are used to compare $\fptaylor$ and $\rosa$ in~\cite{fptaylor15}. 
\item The three programs \code{o-q} come from the global optimization literature and correspond respectively to Problem 3.3, 4.6 and 4.7 in~\cite{Floudas90}. They implement polynomial functions and involve polynomial preconditions (i.e. $\X$ is not a simple box but rather a set defined with conjunction of nonlinear polynomial inequalities).
\item The four programs \code{r-u} involve transcendental functions. The two programs \code{r} and \code{s} are used in the $\fptaylor$ paper~\cite{fptaylor15} and correspond respectively to the program \code{logexp} (see Example~\ref{ex:logexp}) and the program \code{sphere} taken from NASA World Wind Java SDK~\cite{NASA}. The $2$ programs \code{t} and \code{q} respectively implement the functions coming from the optimization problems \textit{Hartman 3} and \textit{Hartman 6} in~\cite{Ali05}, involving both sums of exponential functions composed with quadratic polynomials.
\item The last two programs \code{v-w} involve conditional loops and come from the static analysis literature. They correspond to the two respective running examples of~\cite{Zonotope10} and~\cite{Marechal14}. The first program \code{v} is used in the $\rosa$ paper~\cite{Darulova14Popl} for the analysis of divergence path error.
\end{itemize}
%
The three tools $\realtofloat$, $\rosa$ and $\fptaylor$ can handle input variable uncertainties as well as multi-precision, but for the sake of conciseness, we only considered to compare their performance on programs implemented in single ($\epsilon = 2^{-24}$) or double ($\epsilon = 2^{-53}$) precision floating-point.
For the programs involving transcendental functions, we followed the same procedure as in $\fptaylor$ while adjusting the precision $\epsilon \,  (f_{\R}) = 1.5 \epsilon$ for each special function $f_{\R} \in \{\sin, \cos, \log, \exp \}$. Each univariate transcendental function is approximated from below (resp.~from above) using suprema (resp.~infima) of linear or quadratic polynomials (see Example~\ref{ex:logexp} for the case of program \code{logexp}). 
%
\subsection{Comparison Results}
%
Comparison results for error bound computation are presented in Table~\ref{table:error}. 

Our $\realtofloat$ tool computes the tightest upper bounds for $19$ out of $\nbenchs$ benchmarks. For the $3$ programs \code{a-c} involving rational functions and the program \code{t} encoding Problem~\textit{Hartman 3}, $\fptaylor$ computes the tightest bounds. The symbol $\divzero$ means that the IA procedure aborts with a division by zero exception, which typically occurs when computing too coarse interval enclosures of rational function denominators.
One current limitation of $\realtofloat$ is its ability to manipulate symbolic expressions, e.g. computing rational function derivatives or yielding reduction to the same denominator. Performance of $\fptaylor$ are better in this case, as it handles properly these operations through the interface with the \textsc{Maxima} computer algebra system~\cite{maxima}.
Accordingly with the $\fptaylor$ paper~\cite{fptaylor15}, the $\rosa$ real compiler often provides coarser bounds, except for the two conditional programs \code{v} and \code{w} and the polynomial program \code{o}.
%
\begin{figure*}[!ht]
\begin{center}
\includegraphics[scale=0.8]{timebound2.pdf}
\caption{Comparisons of execution times and upper bounds of rounding errors obtained with $\realtofloat$, $\rosa$ and $\fptaylor$}\label{fig:timebound}
\end{center}
\end{figure*}
%
\begin{table}[!ht]
%\small
\begin{center}
\caption{Comparison of execution times (in seconds) for absolute rounding error bounds (the best results are emphasized using \textbf{bold fonts})}
\begin{tabular}{p{2.3cm}cccc}
\hline
\multirow{1}{*}{Benchmark} & \texttt{id} & $\realtofloat$ & $\rosa$  & $\fptaylor$   \\
\hline
\multirow{1}{*}{\texttt{doppler1}} & \texttt{a} &
$7.73$ & $23.7$ & $\mathbf{7.45}$\\
%\if{
\multirow{1}{*}{\texttt{doppler2}} & \texttt{b} &
$\mathbf{4.97}$ & $24.0$ & $7.29$\\
\multirow{1}{*}{\texttt{doppler3}} & \texttt{c} &
$7.61$ & $35.4$ & $\mathbf{7.15}$\\
%\multirow{1}{*}{doppler1$\star$}
%? & $27.3$ & $8.84$\\
\multirow{1}{*}{\texttt{rigidBody1}} & \texttt{d} &
$6.18$ & $\mathbf{0.20}$ & $6.22$ \\
\multirow{1}{*}{\texttt{rigidBody2}} & \texttt{e} &
$8.51$ & $11.7$ & $\mathbf{8.37}$ \\
\multirow{1}{*}{\texttt{verhulst}} & \texttt{f} &
$\mathbf{2.50}$ & $9.93$ & $3.52$\\
\multirow{1}{*}{\texttt{carbonGas}} & \texttt{g} &
$\mathbf{5.05}$ & $39.4$ & $10.1$\\
%\multirow{1}{*}{jetEngine}
%? & $328$ & ?\\
\multirow{1}{*}{\texttt{predPrey}} & \texttt{h} &
$\mathbf{4.10}$ & $36.2$ & $5.05$\\
\multirow{1}{*}{\texttt{kepler0}} & \texttt{i} &
$\mathbf{0.93}$ & $4.28$ & $15.1$\\
\multirow{1}{*}{\texttt{kepler1}} & \texttt{j} &
$\mathbf{3.15}$ & $32.9$ & $330$\\
\multirow{1}{*}{\texttt{kepler2}} & \texttt{k} &
$\mathbf{21.6}$ & $56.1$ & $939$\\
\multirow{1}{*}{\texttt{sineTaylor}} & \texttt{l} &
$\mathbf{0.42}$ & $0.99$ & $9.66$ \\
\multirow{1}{*}{\texttt{sineOrder3}} & \texttt{m} &
$\mathbf{0.16}$ & $6.71$ & $5.19$\\
\multirow{1}{*}{\texttt{sqroot}} & \texttt{n} &
$\mathbf{0.38}$ & $1.89$ & $10.1$\\
\multirow{1}{*}{\texttt{floudas1}} & \texttt{o} &
$19.5$ & $17.8$ & $\mathbf{15.3}$\\
\multirow{1}{*}{\texttt{floudas2}} & \texttt{p} &
$\mathbf{0.23}$ & $2.49$ & $1.83$\\
\multirow{1}{*}{\texttt{floudas3}} & \texttt{q} &
$\mathbf{0.28}$ & $17.3$ & $4.40$\\
\multirow{1}{*}{\texttt{logexp}} & \texttt{r} &
$\mathbf{0.06}$ & $-$ & $2.04$\\
\multirow{1}{*}{\texttt{sphere}} & \texttt{s} &
$\mathbf{0.03}$ & $-$ & $4.48$\\
%\multirow{1}{*}{azimuth} &
%? & $-$ & $33.0$\\
\multirow{1}{*}{\texttt{hartman3}} & \texttt{t} &
$\mathbf{1.52}$ & $-$ & $62.5$ \\
\multirow{1}{*}{\texttt{hartman6}} & \texttt{u} &
$\mathbf{10.6}$ & $-$ & $\text{OoM}$ \\
\multirow{1}{*}{\texttt{cav10}} & \texttt{v} &
$\mathbf{0.37}$ & $12.3$ & $-$ \\
\multirow{1}{*}{\texttt{perin}} & \texttt{w} &
$\mathbf{1.38}$ & $41.4$ & $-$ \\
%\multirow{1}{*}{?} & \texttt{x} &
%? & ? & $-$ \\
\hline
%}\fi
\end{tabular}
\label{table:cpu}
\end{center}
\end{table}
%
Program \code{u} can only be tackled with $\realtofloat$ as $\fptaylor$ aborted after running out of memory (meaning of the symbol OoM). A possible failure explanation is the complexity of the corresponding Problem~\textit{Hartman 6}, involving $133$ arithmetic operations and $6$ input variables.
To the best of our knowledge, $\realtofloat$ is the only academic tool which is able to handle the general class of programs involving either transcendental functions or conditional loops. The $\fptaylor$ (resp.~$\rosa$) does not currently handle conditionals (resp.~transcendental functions), as meant by the symbol $-$ in the corresponding column entries. However, an interface between both software would embed them with their respective missing feature.

These error bound comparison results together with their corresponding execution timings (given in Table~\ref{table:cpu}) are used to plot the data points shown in Figure~\ref{fig:timebound}.

%Figure~\ref{fig:timebound} provides a way to compare both execution times and error bounds obtained with the different tools, and gathers results from Table~\ref{table:error} and Table~\ref{table:cpu}. 
For each benchmark identified by \code{id}, let $t_{\realtofloat}$ (in 3rd column of Table~\ref{table:cpu}) refers to the execution time of $\realtofloat$ to obtain the corresponding upper bound $\epsilon_{\realtofloat}$ (in 4th column of Table~\ref{table:error}). Similarly, let us define the execution times $t_{\rosa}$, $t_{\fptaylor}$ and the corresponding error bounds $\epsilon_{\rosa}$, $\epsilon_{\fptaylor}$. Then the x-axis coordinate of the point \circled{id} (resp.~\squared{id}) displayed in Figure~\ref{fig:timebound} corresponds to the relative difference between the execution time of $\rosa$ (resp.~$\fptaylor$) and $\realtofloat$, i.e.~the relative ratio $\frac{t_{\rosa} - t_{\realtofloat}}{t_{\realtofloat}}$ (resp.~$\frac{t_{\fptaylor} - t_{\realtofloat}}{t_{\realtofloat}}$). Similarly, the y-axis coordinate of the point \circled{id} (resp.~\squared{id}) is $\frac{\epsilon_{\rosa} - \epsilon_{\realtofloat}}{\epsilon_{\realtofloat}}$ (resp.~$\frac{\epsilon_{\fptaylor} - \epsilon_{\realtofloat}}{\epsilon_{\realtofloat}}$). For readability purpose, we used an appropriate logarithmic scale for the x-axis. 

The axes of the coordinate system of Figure~\ref{fig:timebound} divide the plane into four quadrants: 
the nonnegative quadrant $(+,+)$ contains data points referring to programs for which $\realtofloat$ computes the tighter bounds in less time, 
the second one $(+,-)$ contains points referring to programs for which $\realtofloat$ is slower but more accurate, 
the non-positive quadrant $(-, -)$ for which $\realtofloat$ is slower and computes coarser bounds 
and 
the last one $(-,+)$ for which $\realtofloat$ is faster but less accurate. On the quadrant $(+,-)$, one can see that $\realtofloat$ computes bound which are less accurate than $\fptaylor$ on programs \code{a-c} and \code{t}, but much faster for program \code{b}. The quadrant $(-, +)$ indicates that $\rosa$ (resp.~$\fptaylor$) is more efficient but less precise than $\realtofloat$ on program \code{d} (resp.~\code{o}). The presence of the majority of plots on the nonnegative quadrant $(+, +)$ confirms that $\realtofloat$ does not compromise efficiency at the expense of accuracy.
%

%


%However, $\fptaylor$ implements special cases to eliminate some error terms, for instance when $\op$ is the multiplication and one of the operands is a nonnegative power of two, then the error $e$ is set to zero. 


% We didn't succeed to find conditional programs with more than 2 variables and constraints of degree greater than 3. 

\section{Related Works}
%
SMT solvers allow to analyze programs with various semantics or specifications but are limited for the manipulation of problems involving nonlinear arithmetics. When SAT/SMT solvers output proof witnesses, they can be formally rechecked inside the $\coq$ proof assistant~\cite{smtcoq}. While ensuring soundness, the efficiency of the procedure is not compromised due to tactics enjoying the mechanism of computational reflection.

%linear programming relaxations~\cite{Boland10HGR}.
% The software $\fptaylor$, Gappa and Fluctuat implement another rounding model, corresponding to FPTaylor (b). This implies to subdivide each interval in \cup [2^n, 2^{n+1}] to be able to optimize the functions
% Chap 26 [Higham]
\section{Conclusion and Perspectives} %0.5p
%
Our verification framework allows us to over-approximate rounding errors occurring while executing nonlinear programs implemented with finite precision.
The framework relies on semidefinite optimization, ensuring tight and certified approximations. Our approach extends to medium-size nonlinear problems, due to  automatic detection of the correlation sparsity pattern of input variables and round-off error variables.

This work yields several directions for further investigation of research. 
First, we intend to increase the size of graspable instances by exploiting symmetry patterns of certain program sub-classes to use specific SDP hierarchies~\cite{Riener2013SymmetricSDP}. Next, we plan to improve the efficiency of the formal polynomial checker inside Coq by using the interval libraries available inside the proof assistant, rather than using exact polynomial arithmetic in the current framework. 
We also intend to provide rounding error bounds for more general programs, involving either finite or infinite loops. Such extensions has been made realistic that recent efforts to approximate inductive invariants with semidefinite relaxations.
Finally, we plan to combine this optimization framework with the procedure in~\cite{Gao15FPGA} to improve the automatic reordering of arithmetic expressions, thus allowing more efficient optimization of FPGA implementations.

%1) lower bounds with other hierarchies
%2) fixed precision
%3) while loops
%4) Formalization of maxplus (cf JFR)


\acks
This work was partly funded by the Engineering and Physical Sciences Research Council (EPSRC) Challenging Engineering Grant (EP/I020457/1).
% Thank John Wickerson, Alexey Solovyev, Eva Darulova
% We recommend abbrvnat bibliography style.
\newpage
~
\newpage
\bibliographystyle{abbrvnat}
\bibliography{roundsdp}
% The bibliography should be embedded for final submission.

\if{
\begin{thebibliography}{}
\softraggedright

\bibitem[Smith et~al.(2009)Smith, Jones]{smith02}
P. Q. Smith, and X. Y. Jones. ...reference text...

\end{thebibliography}
}\fi

\appendix
\section*{Appendix A: Nonlinear Programs}
\label{sec:appa}
%
{\scriptsize
\begin{lstlisting}

let box_doppler1 $u$ $v$ $T = [(-100, 100);(20, 20e3);(-30, 50)];;$ 
let obj_doppler1 $u$ $v$ $T = [($let $t_1 = 331.4 + 0.6 * T$ in 
$-t_1*v/((t_1 + u)*(t_1 + u)), \,2.75\text{e--}13)];;$

let box_doppler2 $u$ $v$ $T = [(-125, 125);(15, 25e3);(-40, 60)];;$ 
let obj_doppler2 $u$ $v$ $T = [($let $t_1 = 331.4 + 0.6 * T$ in 
$-t_1*v/((t_1 + u)*(t_1 + u)), \,8.04\text{e--}13)];;$

let box_doppler3 $u$ $v$ $T = [(-30, 120);(320, 20300);(-50, 30)];;$ 
let obj_doppler3 $u$ $v$ $T = [($let $t_1 = 331.4 + 0.6 * T$ in 
$-t_1*v/((t_1 + u)*(t_1 + u)), \,1.37\text{e--}13)];;$

let box_rigidbody1 $x_1 \, x_2 \, x_3 = [(-15, 15); (-15, 15); (-15, 15)];;$
let obj_rigidbody1 $x_1 \, x_2 \, x_3 = [($
$-x_1*x_2 - 2 * x_2 * x_3 - x_1 - x_3, \,3.80\text{e--}13)];;$

let box_rigidbody2 $x_1 \, x_2 \, x_3 = [(-15, 15); (-15, 15); (-15, 15)];;$
let obj_rigidbody2 $x_1 \, x_2 \, x_3 = [(2*x_1*x_2*x_3 + 3*x_3*x_3$ 
$- x_2*x_1*x_2*x_3 + 3*x_3*x_3 - x_2, 3.98\text{e--}11)];;$

let box_verhulst $x = [(0.1, 0.3)];;$
let obj_verhulst $x = [( 4 * x / (1 + (x/1.11)), \, 3.40\text{e--}16)];;$

let box_carbonGas $v = [(0.1, 0.5)];;$
let obj_carbonGas $v = [($let $p = 3.5e7$ in let $a = 0.401$ in 
let $b = 42.7\text{e--}6$ in let $t = 300$ in let $n = 1000$ in
$(p + a * (n/v)**2) * (v - n * b) - 1.3806503\text{e--}23 * n * t
,\, 1.10\text{e--}8)];;$

let box_predPrey$ \,x = [(0.1, 0.3)];;$
let obj_predPrey$\, x = [(4 * x * x/(1 + (x/1.11)**2), \,1.57\text{e--}16)];;$

let box_kepler0 $x_1 \, x_2 \, x_3 \, x_4 \, x_5 \, x_6 = $
$[(4, 6.36); (4, 6.36); (4, 6.36); (4, 6.36); (4, 6.36); (4, 6.36)];;$
let obj_kepler0 $x_1 \, x_2 \, x_3 \, x_4 \, x_5 \, x_6 = [(x_2 * x_5 + x_3 * x_6 - x_2 * x_3$
$- x_5 * x_6 + x_1 * ( - x_1 + x_2 + x_3 - x_4 + x_5 + x_6), \,8.76\text{e--}14)];;$

let box_kepler1 $x_1 \, x_2 \, x_3 \, x_4 = $
$[(4, 6.36); (4, 6.36); (4, 6.36); (4, 6.36)];;$
let obj_kepler1 $x_1 \, x_2 \, x_3 \, x_4 = [( x_1 * x_4 * (- x_1 + x_2 + x_3 - x_4)$
$+ x_2 * (x_1 - x_2 + x_3 + x_4) + x_3 * (x_1 + x_2 - x_3 + x_4) $
$- x_2 * x_3 * x_4 - x_1 * x_3 - x_1 * x_2 - x_4, \,2.96\text{e--}13)];; $

let box_kepler2 $x_1 \, x_2 \, x_3 \, x_4 \, x_5 \, x_6 = $
$[(4, 6.36); (4, 6.36); (4, 6.36); (4, 6.36); (4, 6.36); (4, 6.36)];;$
let obj_kepler2 $x_1 \, x_2 \, x_3 \, x_4 \, x_5 \, x_6 = [(x_1 * x_4 * (- x_1 + x_2 + x_3$
$- x_4 + x_5 + x_6) + x_2 * x_5 * (x_1 - x_2 +x_3 +x_4- x_5 +x_6) $
$+ x_3* x_6* (x_1 + x_2 - x_3 + x_4 + x_5 - x_6) - x_2* x_3* x_4 $
$- x_1* x_3* x_5 - x_1* x_2* x_6 - x_4* x_5* x_6, \, 1.90\text{e--}12)];;$

let box_sineTaylor$ \, x = [(-1.57079632679, 1.57079632679)];;$
let obj_sineTaylor$ \, x = [(x - (x*x*x)/6.0$
$+ (x*x*x*x*x)/120.0 $
$- (x*x*x*x*x*x*x)/5040.0,\, 5.53\text{e--}16)];;$

let box_sineOrder3 $z = [(-2, 2)];;$
let obj_sineOrder3 $z = [(0.954929658551372 * z $
$-0.12900613773279798*(z*z*z),\, 2.97\text{e--}16)];;$

let box_sqroot $y = [(0,1)];;$
let obj_sqroot $y = [(1.0 + 0.5*y - 0.125*y*y $
$+ 0.0625*y*y*y - 0.0390625*y*y*y*y, \,6.68\text{e--}16)];;$

let box_floudas1 $x_1 \, x_2 \, x_3 \, x_4 \, x_5 \, x_6 = $
$[(0, 6); (0, 6); (1, 5); (0, 6); (1, 5); (0, 10)];;$
let cstr_floudas1 $x_1 \, x_2 \, x_3 \, x_4 \, x_5 \, x_6 = $
$[(x_3 - 3)**2 + x_4 - 4; (x_5 - 3)**2 + x_6 - 4; $
$2 - x_1 + 3 * x_2; 2 + x_1 - x_2; 6 - x_1 - x_2; x_1 + x_2 - 2];;$
let obj_floudas1 $x_1 \, x_2 \, x_3 \, x_4 \, x_5 \, x_6 = [( -25 * (x_1 - 2)**2 $
$- (x_2 - 2)**2 - (x_3 - 1)**2 - (x_4 - 4)**2 $
$- (x_5 - 1)**2 - (x_6 - 4)**2, \, 2.99\text{e--}13)];;$


let box_floudas2$\,x_1 \, x_2 = [(0, 3); (0, 4)];;$
let cstr_floudas2$\,x_1 \, x_2 = [$
$2 * x_1**4 - 8 * x_1**3 + 8 * x_1*x_1 - x_2; $
$4 * x_1**4 - 32 * x_1**3 + 88 * x_1*x_1 - 96 * x_1 + 36 - x_2];;$
let obj_floudas2$\,x_1 \, x_2 = [(-x_1 - x_2, \, 9.03\text{e--}16)];;$

let box_floudas3$\,x_1 \, x_2 = [(0, 2); (0, 3)];;$
let cstr_floudas3$\,x_1 \, x_2 = [-2 * x_1**4 + 2 - x_2];;$
let obj_floudas3$\,x_1 \, x_2 = [($
$-12 * x_1 - 7 * x_2 + x_2*x_2, \, 8.90\text{e--}15)];;$

let box_logexp $x = [(-8,8)];;$
let obj_logexp $x = [(\log(1 + \exp(x)), \, 1.65\text{e--}15)];;$

let box_sphere $x \, r \, y \, z = [(-10, 10); (0, 10);$
$(-1.570796, 1.570796); (-3.14159265, 3.14159265)];;$
let obj_sphere $x \, r \, y \, z = [(x + r * \sin (y) * \cos(z),\,7.78\text{e--}15)];;$

let box_hartman3 $x_1 \, x_2 \, x_3 = [(0, 1); (0, 1);(0, 1)];;$
let obj_hartman3 $x_1 \, x_2 \, x_3 = [($
let $e1 = 3.0 * (x_1 - 0.3689) **2 + 10.0 * (x_2 - 0.117) **2$
$+ 30.0 * (x_3 - 0.2673) **2$ in
let $e2 = 0.1 * (x_1 - 0.4699) **2 + 10.0 * (x_2 - 0.4387) **2$
$+ 35.0 * (x_3 - 0.747) **2$ in
let $e3 = 3.0 * (x_1 - 0.1091) **2 + 10.0 * (x_2 - 0.8732) **2$
$+ 30.0 * (x_3 - 0.5547) **2$ in
let $e4 = 0.1 * (x_1 - 0.03815) **2 + 10.0 * (x_2 - 0.5743) **2$
$+ 35.0 * (x_3 - 0.8828) **2$ in
$- (1.0 * \exp(-e1) + 1.2 * \exp(-e2)$
$+ 3.0 * \exp(-e3) + 3.2 * \exp(-e4)), \, 3.85\text{e--}13)];;$

let box_hartman6 $x_1 \, x_2 \, x_3 \, x_4 \, x_5 \, x_6 = $
$[(0, 1); (0, 1);(0, 1);(0, 1);(0, 1);(0, 1)];;$
let obj_hartman6 $x_1 \, x_2 \, x_3 \, x_4 \, x_5 \, x_6 = [($
let $e1 = 10.0 * (x_1 - 0.1312)**2 + 3.0 * (x_2 - 0.1696)**2 $
$+ 17.0 * (x_3 - 0.5569)**2 + 3.5 * (x_4 - 0.0124)**2 $
$+ 1.7 * (x_5 - 0.8283)**2 + 8.0 * (x_6 - 0.5886)**2$ in
let $e2 = 0.05 * (x_1 - 0.2329)**2 + 10.0 * (x_2 - 0.4135)**2$ 
$+ 17.0 * (x_3 - 0.8307)**2 + 0.1 * (x_4 - 0.3736)**2$
$+ 8.0 * (x_5 - 0.1004)**2 + 14.0 * (x_6 - 0.9991)**2$ in
let $e3 = 3.0 * (x_1 - 0.2348)**2 + 3.5 * (x_2 - 0.1451)**2$ 
$+ 1.7 * (x_3 - 0.3522)**2 + 10.0 * (x_4 - 0.2883)**2$
$+ 17.0 * (x_5 - 0.3047)**2 + 8.0 * (x_6 - 0.665)**2$ in
let $e4 = 17.0 * (x_1 - 0.4047)**2 + 8.0 * (x_2 - 0.8828)**2$ 
$+ 0.05 * (x_3 - 0.8732)**2 + 10.0 * (x_4 - 0.5743)**2$
$+ 0.1 * (x_5 - 0.1091)**2 + 14.0 * (x_6 - 0.0381)**2$ in
$- (1.0 * \exp(-e1) + 1.2 * \exp(-e2)$
$+ 3.0 * \exp(-e3) + 3.2 * \exp(-e4)), \, 4.48\text{e--}13)];;$

let box_cav10 $x = [(0, 10)];;$
let obj_cav10 $x = [($ if $(x*x - x > 0)$ then $x*0.1$ 
else $x*x+2, \, 2.91)];;$

let box_perin $x \, y = [(1,7); (-2, 7)];;$
let cstr_perin $x \, y = [x-1; y+2; x-y; 5-y-x];;$
let obj_perin $x \, y = [($ if $(x*x + y*y \leq 4)$ then $y * x$
else $0, \, 2.01)];; $


\end{lstlisting}
}

\if{
{\scriptsize
\begin{lstlisting}
let box_doppler1 u v T = [$(-100, 100); (20, 2e4);(-30, 50)$];; 
let obj_doppler1 u v T = [(
let $t_1 = 331.4 + 0.6 * T$ in 
$-t_1*v/((t_1 + u)*(t_1 + u))$, $3\text{e--}13$)];;
\end{lstlisting}
}
{\scriptsize
\begin{lstlisting}
procedure doppler2(u : real, v : real, T : real) returns (r : real) {
 assume (-125.0 <= u && u <= 125.0 && 15.0 <= v && v <= 25000.0 && -40.0 <= T && T <= 60.0);
  var t1 := 331.4 + 0.6 * T;
  r := -t1*v/((t1 + u)*(t1 + u));
}
\end{lstlisting}
}
{\scriptsize
\begin{lstlisting}
procedure doppler3(u : real, v : real, T : real) returns (r : real) {
 assume (-30.0 <= u && u <= 120.0 && 320.0 <= v && v <= 20300.0 && -50.0 <= T && T <= 30.0);
  var t1 := 331.4 + 0.6 * T;
  r := -t1*v/((t1 + u)*(t1 + u));
}
\end{lstlisting}
}
{\scriptsize
\begin{lstlisting}
procedure rigidBody1(x_1 : real, x_2 : real, x_3 : real) returns (r : real) {
 assume (-15.0 <= x_1 && x_1 <= 15.0 && -15.0 <= x_2 && x_2 <= 15.0 && -15.0 <= x_3 && x_3 <= 15.0);
  r := -x_1*x_2 - 2.0 * x_2 * x_3 - x_1 - x_3;
}
\end{lstlisting}
}
{\scriptsize
\begin{lstlisting}
procedure rigidBody2(x_1 : real, x_2 : real, x_3 : real) returns (r : real) {
 assume (-15.0 <= x_1 && x_1 <= 15.0 && -15.0 <= x_2 && x_2 <= 15.0 && -15.0 <= x_3 && x_3 <= 15.0);
  r := 2.0*x_1*x_2*x_3 + 3.0*x_3*x_3 - x_2*x_1*x_2*x_3 + 3.0*x_3*x_3 - x_2;
}
\end{lstlisting}
}
{\scriptsize
\begin{lstlisting}
procedure sineTaylor(x : real) returns (r : real) {
 assume (-1.57079632679  <= x && x <= 1.57079632679);
    r := x - (x*x*x)/6.0 + (x*x*x*x*x)/120.0 - (x*x*x*x*x*x*x)/5040.0 ;
}
\end{lstlisting}
}
{\scriptsize
\begin{lstlisting}
procedure sineOrder3(x : real) returns (r : real) {
 assume (-2  <= x && x <= 2);
  r := 0.954929658551372 * x - 0.12900613773279798*x*x*x ;
}
\end{lstlisting}
}
{\scriptsize
\begin{lstlisting}
procedure sqroot(x : real) returns (r : real) {
 assume (0  <= x && x <= 1);
  r := 1.0 + 0.5*x - 0.125*x*x + 0.0625*x*x*x - 0.0390625*x*x*x*x;
}
\end{lstlisting}
}
}\fi


\end{document}

%                       Revision History
%                       -------- -------
%  Date         Person  Ver.    Change
%  ----         ------  ----    ------

%  2013.06.29   TU      0.1--4  comments on permission/copyright notices

