%-----------------------------------------------------------------------------
%
%               Template for sigplanconf LaTeX Class
%
% Name:         sigplanconf-template.tex
%
% Purpose:      A template for sigplanconf.cls, which is a LaTeX 2e class
%               file for SIGPLAN conference proceedings.
%
% Guide:        Refer to "Author's Guide to the ACM SIGPLAN Class,"
%               sigplanconf-guide.pdf
%
% Author:       Paul C. Anagnostopoulos
%               Windfall Software
%               978 371-2316
%               paul@windfall.com
%
% Created:      15 February 2005
%
%-----------------------------------------------------------------------------


\documentclass[preprint,fleqn,nocopyrightspace]{sigplanconf}

% The following \documentclass options may be useful:

% preprint      Remove this option only once the paper is in final form.
% 10pt          To set in 10-point type instead of 9-point.
% 11pt          To set in 11-point type instead of 9-point.
% authoryear    To obtain author/year citation style instead of numeric.
\usepackage{listings}
\def\lstlanguagefiles{defManOcaml.tex}
\lstset{language = Ocaml}
\newcommand{\code}[1]{\lstinline{#1}}
%\begin{lstlisting}\end{lstlisting}

\usepackage[utf8]{inputenc}
\usepackage[T1]{fontenc}
\usepackage{lmodern}
\usepackage{graphicx}  % for pdf, bitmapped graphics files
\usepackage{amsmath} % assumes amsmath package installed
\usepackage{amssymb}  % assumes amsmath package installed
\usepackage{color}
\usepackage{enumitem}
\usepackage{myalgo}
\let\proof\relax
\let\endproof\relax
\usepackage{amsthm}
%\usepackage{ntheorem}
\usepackage{hyperref}
\usepackage{subfigure}
\usepackage{enumerate}
%\usepackage{caption}
\usepackage{multirow}
\usepackage{tikz, pgfplots}
\pgfplotsset{compat=1.8}
%\usepackage{natbib} %for bibliography via bibtex
%\usepackage{enumitem}
%\lstMakeShortInline{$}
\newcommand{\add}[1]{#1}
\newcommand{\del}[1]{\textcolor{gray}{#1}}

%\newcommand{\P}{\mathbb{P}}
%\newcommand{\Q}{\mathbb{Q}}
%\newcommand{\E}{\mathbb{E}}
\def\sizefig{0.35}
\def\sizesmallfig{0.30}
\def\sizetinyfig{0.24}
\newcommand{\setA}{\mathcal{A}} % Semialgebraic functions
\newcommand{\setD}{\mathcal{D}} % Dictionnary of Univariate transcendental functions
\newcommand{\setU}{\mathcal{U}} % Univariate functions := \setT \cup {sqrt, abs, power functions}
\newcommand{\suppf}[1]{\text{supp}(#1)}
\newcommand{\mons}[2]{\N_{#1}^{#2}}
\newcommand{\R}{\mathbb{R}}
\newcommand{\F}{\mathbb{F}}
\newcommand{\Pcal}{\mathcal{P}}
\newcommand{\Pcalp}{\Pcal^{(p)}}
\newcommand{\Scal}{\mathcal{S}}
\newcommand{\N}{\mathbb{N}}
\newcommand{\x}{\mathbf{x}}
\newcommand{\e}{\mathbf{e}}
\newcommand{\y}{\mathbf{y}}
\newcommand{\z}{\mathbf{z}}
\newcommand{\nbenchs}{23}
\newcommand{\alphab}{\boldsymbol{\alpha}}
\newcommand{\epsilonb}{\boldsymbol{\epsilon}}
\newcommand{\deltab}{\boldsymbol{\delta}}
\renewcommand{\b}{\mathbf{b}}
\newcommand{\f}{\mathbf{f}} 
\newcommand{\Plam}{\P_{\lambda}}
\newcommand{\Plamp}{\P_{\lambda}^{(p)}}
\def\P{\mathbf{P}}
\def\Q{\mathbf{Q}}
\def\L{\mathbf{L}}
\def\D{\mathbf{D}}
\def\q{\mathbf{q}}
\newcommand{\M}{\mathbf{M}}
\def\m{\mathbf{m}}
\def\H{\mathbf{H}}
\def\h{\mathbf{h}}
\def\f{f}
\def\a{\mathbf{a}}
\def\m{\mathbf{m}}
\def\p{\mathbf{p}}
\def\S{\mathbf{S}}
\def\B{\mathbf{B}}
\def\E{\mathbf{E}}
\def\K{\mathbf{K}}
\def\S{\mathbf{S}}
\def\Q{\mathbf{Q}}
\def\X{\mathbf{X}}
\def\Y{\mathbf{Y}}
\newcommand{\A}{\mathbf{A}}
\newcommand{\Shat}{\hat{\S}}
\newcommand{\Bb}{\mathbf{B}}
\newcommand{\flam}{{f}_{{\lambda}}}
\newcommand{\flamfun}[1]{f_{\lambda}(#1)}
\newcommand{\flamfunp}[2]{f^{(#2)}_{\lambda}(#1)}
\newcommand{\flamx}{\flamfun{\x}}
\newcommand{\flampx}{\flamfunp{\x}{p}}
\newcommand{\flamstar}{f^*({\lambda})}
\newcommand{\flamstarj}[1]{f_{#1}^*({\lambda})}
\newcommand{\Mcal}{\mathcal{M}}
\newcommand{\nsdp}{n_{\text{sdp}}}
\newcommand{\divzero}{\text{Div0}}
\newcommand{\nlift}{n_{\text{lift}}}
\newcommand{\Slift}{\S_{\text{lift}}}
\renewcommand{\prec}{\text{prec}}
\newcommand{\mlift}{m_{\text{lift}}}
\newcommand{\dlift}{r_{\text{lift}}}
\newcommand{\msdp}{m_{\text{sdp}}}
\newcommand{\xlamstar}{{\x}^*({\lambda})}
\newcommand{\Kpol}{\K_{\text{poly}}}
\newcommand{\transpose}{\top}%\newcommand{\transpose}{\mathbf{\intercal}}
\DeclareMathOperator{\vol}{vol}
%\DeclareMathOperator{\op}{op}
\DeclareMathOperator{\conv}{conv}
\newcommand{\red}[1]{\textbf{{\color{red}#1}}}
\newcommand{\iaboundfun}[2]{\mathtt{ia\_bound}(#1, #2)}
\newcommand{\iabound}{\mathtt{ia\_bound}}
\newcommand{\sdpboundfun}[3]{\mathtt{sdp\_bound}(#1, #2, #3)}
\newcommand{\sdpbound}{\mathtt{sdp\_bound}}
\newcommand{\boundfun}[7]{\mathtt{bound}(#1, #2, #3, #4, #5, #6, #7)}
\newcommand{\bound}{\mathtt{bound}}
\newcommand{\boundnlprogfun}[7]{\mathtt{bound\_nlprog}(#1, #2, #3, #4, #5, #6, #7)}
\newcommand{\boundnlprog}{\mathtt{bound\_nlprog}}
\newcommand{\sdppolyfun}[3]{\mathtt{sdp\_poly}(#1, #2, #3)}
\newcommand{\sdppoly}{\mathtt{sdp\_poly}}
\newcommand{\liftfun}[3]{\mathtt{lift}(#1, #2, #3)}
\newcommand{\lift}{\mathtt{lift}}
\newcommand{\poly}{_\text{poly}}
\newcommand{\sa}{_\text{sa}}
\newcommand{\sdpsafun}[3]{\mathtt{sdp\_sa}(#1, #2, #3)}
\newcommand{\sdpsa}{\mathtt{sdp\_sa}}
\newcommand{\sdptranscfun}[3]{\mathtt{sdp\_trancs}(#1, #2, #3)}
\newcommand{\sdptransc}{\mathtt{sdp\_transc}}
%\newcommand{\brev}{\color{red}}
%\newcommand{\erev}{\color{black}}
\newcommand{\sthreefp}{\mathtt{s3fp}}
\newcommand{\realtofloat}{\mathtt{Real2Float}}
\newcommand{\smtcoq}{\mathtt{smtcoq}}
\newcommand{\dreal}{\mathtt{dReal}}
\newcommand{\hol}{\text{\sc Hol-light}}
\newcommand{\op}{\mathtt{op}}
\newcommand{\bop}{\mathtt{bop}}
\newcommand{\coq}{\text{\sc Coq}}
\newcommand{\ocaml}{\text{\sc OCaml}}
\newcommand{\rosa}{\mathtt{Rosa}}
\newcommand{\sdpa}{\text{\sc Sdpa}}
\newcommand{\fptaylor}{\mathtt{FPTaylor}}
\newcommand{\nlcertify}{\mathtt{NLCertify}}
\makeatletter
\newcommand*{\circled}{\@ifstar\circledstar\circlednostar}
\newcommand*{\squared}{\@ifstar\squaredstar\squarednostar}
\makeatother

\newcommand*\circledstar[1]{%
  \tikz[baseline=(C.base)]
    \node[%
      fill,
      circle,
      minimum size=1.em,
      text=white,
%      font=\sffamily,
      inner sep=0.5pt
    ](C) {\texttt{#1}};%
}
\newcommand*\circlednostar[1]{%
  \tikz[baseline=(C.base)]
    \node[%
      draw,
      circle,
      minimum size=1.em,
%      font=\sffamily,
      inner sep=0.5pt
    ](C) {\texttt{#1}};%
}
\newcommand*\squaredstar[1]{%
  \tikz[baseline=(C.base)]
    \node[%
      fill,
      rectangle,
      minimum size=1.em,
      text=white,
%      font=\sffamily,
      inner sep=0.5pt
    ](C) {\texttt{#1}};%
}
\newcommand*\squarednostar[1]{%
  \tikz[baseline=(C.base)]
    \node[%
      draw,
      rectangle,
      minimum size=1.em,
%      font=\sffamily,
      inner sep=0.5pt
    ](C) {\texttt{#1}};%
}
%\newcommand{\II}{\mathbb{I}}
%\theoremstyle{plain}
%\newtheorem{thm}{Theorem}[section]
\newtheorem{theorem}{Theorem}[section]
\newtheorem{lemma}[theorem]{Lemma}
\newtheorem{proposition}[theorem]{Proposition}
\newtheorem{corollary}[theorem]{Corollary}

\theoremstyle{plain}
\newtheorem{definition}[theorem]{Definition}
\newtheorem{hypothesis}[theorem]{Assumption}
\newtheorem{conjecture}[theorem]{Conjecture}
\newtheorem{assumption}[theorem]{Assumption}
\newtheorem{example}{Example}

%\theoremstyle{remark}
\newtheorem{remark}{Remark}
%\newtheorem*{note}{Note}
\newtheorem{case}{Case}




\begin{document}

\conferenceinfo{CONF 'yy}{Month d--d, 20yy, City, ST, Country} 
\copyrightyear{20yy} 
\copyrightdata{978-1-nnnn-nnnn-n/yy/mm} 
\doi{nnnnnnn.nnnnnnn}

% Uncomment one of the following two, if you are not going for the 
% traditional copyright transfer agreement.

%\exclusivelicense                % ACM gets exclusive license to publish, 
                                  % you retain copyright

%\permissiontopublish             % ACM gets nonexclusive license to publish
                                  % (paid open-access papers, 
                                  % short abstracts)

%\titlebanner{banner above paper title}        % These are ignored unless
%\preprintfooter{short description of paper}   % 'preprint' option specified.
\title{}
%\title{Automated Precision Tuning using Semidefinite Programming}
%\title{Certified Roundoff Error Bounds Using Semidefinite Programming}

%\subtitle{Subtitle Text, if any}

\authorinfo{}{}{}
%\authorinfo{Name2\and Name3}
%           {Affiliation2/3}
%           {Email2/3}

\maketitle
\appendix
\section*{Appendix A: Nonlinear Programs}
\label{sec:appa}
%
{\scriptsize
\begin{lstlisting}

let box_doppler1 $u$ $v$ $T = [(-100, 100);(20, 20e3);(-30, 50)];;$ 
let obj_doppler1 $u$ $v$ $T = [($let $t_1 = 331.4 + 0.6 * T$ in 
$-t_1*v/((t_1 + u)*(t_1 + u)), \,2.75\text{e--}13)];;$

let box_doppler2 $u$ $v$ $T = [(-125, 125);(15, 25e3);(-40, 60)];;$ 
let obj_doppler2 $u$ $v$ $T = [($let $t_1 = 331.4 + 0.6 * T$ in 
$-t_1*v/((t_1 + u)*(t_1 + u)), \,8.04\text{e--}13)];;$

let box_doppler3 $u$ $v$ $T = [(-30, 120);(320, 20300);(-50, 30)];;$ 
let obj_doppler3 $u$ $v$ $T = [($let $t_1 = 331.4 + 0.6 * T$ in 
$-t_1*v/((t_1 + u)*(t_1 + u)), \,1.37\text{e--}13)];;$

let box_rigidbody1 $x_1 \, x_2 \, x_3 = [(-15, 15); (-15, 15); (-15, 15)];;$
let obj_rigidbody1 $x_1 \, x_2 \, x_3 = [($
$-x_1*x_2 - 2 * x_2 * x_3 - x_1 - x_3, \,3.80\text{e--}13)];;$

let box_rigidbody2 $x_1 \, x_2 \, x_3 = [(-15, 15); (-15, 15); (-15, 15)];;$
let obj_rigidbody2 $x_1 \, x_2 \, x_3 = [(2*x_1*x_2*x_3 + 3*x_3*x_3$ 
$- x_2*x_1*x_2*x_3 + 3*x_3*x_3 - x_2, 3.98\text{e--}11)];;$

let box_verhulst $x = [(0.1, 0.3)];;$
let obj_verhulst $x = [( 4 * x / (1 + (x/1.11)), \, 3.40\text{e--}16)];;$

let box_carbonGas $v = [(0.1, 0.5)];;$
let obj_carbonGas $v = [($let $p = 3.5e7$ in let $a = 0.401$ in 
let $b = 42.7\text{e--}6$ in let $t = 300$ in let $n = 1000$ in
$(p + a * (n/v)**2) * (v - n * b) - 1.3806503\text{e--}23 * n * t
,\, 1.10\text{e--}8)];;$

let box_predPrey$ \,x = [(0.1, 0.3)];;$
let obj_predPrey$\, x = [(4 * x * x/(1 + (x/1.11)**2), \,1.57\text{e--}16)];;$

let box_kepler0 $x_1 \, x_2 \, x_3 \, x_4 \, x_5 \, x_6 = $
$[(4, 6.36); (4, 6.36); (4, 6.36); (4, 6.36); (4, 6.36); (4, 6.36)];;$
let obj_kepler0 $x_1 \, x_2 \, x_3 \, x_4 \, x_5 \, x_6 = [(x_2 * x_5 + x_3 * x_6 - x_2 * x_3$
$- x_5 * x_6 + x_1 * ( - x_1 + x_2 + x_3 - x_4 + x_5 + x_6), \,8.76\text{e--}14)];;$

let box_kepler1 $x_1 \, x_2 \, x_3 \, x_4 = $
$[(4, 6.36); (4, 6.36); (4, 6.36); (4, 6.36)];;$
let obj_kepler1 $x_1 \, x_2 \, x_3 \, x_4 = [( x_1 * x_4 * (- x_1 + x_2 + x_3 - x_4)$
$+ x_2 * (x_1 - x_2 + x_3 + x_4) + x_3 * (x_1 + x_2 - x_3 + x_4) $
$- x_2 * x_3 * x_4 - x_1 * x_3 - x_1 * x_2 - x_4, \,2.96\text{e--}13)];; $

let box_kepler2 $x_1 \, x_2 \, x_3 \, x_4 \, x_5 \, x_6 = $
$[(4, 6.36); (4, 6.36); (4, 6.36); (4, 6.36); (4, 6.36); (4, 6.36)];;$
let obj_kepler2 $x_1 \, x_2 \, x_3 \, x_4 \, x_5 \, x_6 = [(x_1 * x_4 * (- x_1 + x_2 + x_3$
$- x_4 + x_5 + x_6) + x_2 * x_5 * (x_1 - x_2 +x_3 +x_4- x_5 +x_6) $
$+ x_3* x_6* (x_1 + x_2 - x_3 + x_4 + x_5 - x_6) - x_2* x_3* x_4 $
$- x_1* x_3* x_5 - x_1* x_2* x_6 - x_4* x_5* x_6, \, 1.90\text{e--}12)];;$

let box_sineTaylor$ \, x = [(-1.57079632679, 1.57079632679)];;$
let obj_sineTaylor$ \, x = [(x - (x*x*x)/6.0$
$+ (x*x*x*x*x)/120.0 $
$- (x*x*x*x*x*x*x)/5040.0,\, 5.53\text{e--}16)];;$

let box_sineOrder3 $z = [(-2, 2)];;$
let obj_sineOrder3 $z = [(0.954929658551372 * z $
$-0.12900613773279798*(z*z*z),\, 2.97\text{e--}16)];;$

let box_sqroot $y = [(0,1)];;$
let obj_sqroot $y = [(1.0 + 0.5*y - 0.125*y*y $
$+ 0.0625*y*y*y - 0.0390625*y*y*y*y, \,6.68\text{e--}16)];;$

let box_floudas1 $x_1 \, x_2 \, x_3 \, x_4 \, x_5 \, x_6 = $
$[(0, 6); (0, 6); (1, 5); (0, 6); (1, 5); (0, 10)];;$
let cstr_floudas1 $x_1 \, x_2 \, x_3 \, x_4 \, x_5 \, x_6 = $
$[(x_3 - 3)**2 + x_4 - 4; (x_5 - 3)**2 + x_6 - 4; $
$2 - x_1 + 3 * x_2; 2 + x_1 - x_2; 6 - x_1 - x_2; x_1 + x_2 - 2];;$
let obj_floudas1 $x_1 \, x_2 \, x_3 \, x_4 \, x_5 \, x_6 = [( -25 * (x_1 - 2)**2 $
$- (x_2 - 2)**2 - (x_3 - 1)**2 - (x_4 - 4)**2 $
$- (x_5 - 1)**2 - (x_6 - 4)**2, \, 2.99\text{e--}13)];;$


let box_floudas2$\,x_1 \, x_2 = [(0, 3); (0, 4)];;$
let cstr_floudas2$\,x_1 \, x_2 = [$
$2 * x_1**4 - 8 * x_1**3 + 8 * x_1*x_1 - x_2; $
$4 * x_1**4 - 32 * x_1**3 + 88 * x_1*x_1 - 96 * x_1 + 36 - x_2];;$
let obj_floudas2$\,x_1 \, x_2 = [(-x_1 - x_2, \, 9.03\text{e--}16)];;$

let box_floudas3$\,x_1 \, x_2 = [(0, 2); (0, 3)];;$
let cstr_floudas3$\,x_1 \, x_2 = [-2 * x_1**4 + 2 - x_2];;$
let obj_floudas3$\,x_1 \, x_2 = [($
$-12 * x_1 - 7 * x_2 + x_2*x_2, \, 8.90\text{e--}15)];;$

let box_logexp $x = [(-8,8)];;$
let obj_logexp $x = [(\log(1 + \exp(x)), \, 1.65\text{e--}15)];;$

let box_sphere $x \, r \, y \, z = [(-10, 10); (0, 10);$
$(-1.570796, 1.570796); (-3.14159265, 3.14159265)];;$
let obj_sphere $x \, r \, y \, z = [(x + r * \sin (y) * \cos(z),\,7.78\text{e--}15)];;$

let box_hartman3 $x_1 \, x_2 \, x_3 = [(0, 1); (0, 1);(0, 1)];;$
let obj_hartman3 $x_1 \, x_2 \, x_3 = [($
let $e1 = 3.0 * (x_1 - 0.3689) **2 + 10.0 * (x_2 - 0.117) **2$
$+ 30.0 * (x_3 - 0.2673) **2$ in
let $e2 = 0.1 * (x_1 - 0.4699) **2 + 10.0 * (x_2 - 0.4387) **2$
$+ 35.0 * (x_3 - 0.747) **2$ in
let $e3 = 3.0 * (x_1 - 0.1091) **2 + 10.0 * (x_2 - 0.8732) **2$
$+ 30.0 * (x_3 - 0.5547) **2$ in
let $e4 = 0.1 * (x_1 - 0.03815) **2 + 10.0 * (x_2 - 0.5743) **2$
$+ 35.0 * (x_3 - 0.8828) **2$ in
$- (1.0 * \exp(-e1) + 1.2 * \exp(-e2)$
$+ 3.0 * \exp(-e3) + 3.2 * \exp(-e4)), \, 3.85\text{e--}13)];;$

let box_hartman6 $x_1 \, x_2 \, x_3 \, x_4 \, x_5 \, x_6 = $
$[(0, 1); (0, 1);(0, 1);(0, 1);(0, 1);(0, 1)];;$
let obj_hartman6 $x_1 \, x_2 \, x_3 \, x_4 \, x_5 \, x_6 = [($
let $e1 = 10.0 * (x_1 - 0.1312)**2 + 3.0 * (x_2 - 0.1696)**2 $
$+ 17.0 * (x_3 - 0.5569)**2 + 3.5 * (x_4 - 0.0124)**2 $
$+ 1.7 * (x_5 - 0.8283)**2 + 8.0 * (x_6 - 0.5886)**2$ in
let $e2 = 0.05 * (x_1 - 0.2329)**2 + 10.0 * (x_2 - 0.4135)**2$ 
$+ 17.0 * (x_3 - 0.8307)**2 + 0.1 * (x_4 - 0.3736)**2$
$+ 8.0 * (x_5 - 0.1004)**2 + 14.0 * (x_6 - 0.9991)**2$ in
let $e3 = 3.0 * (x_1 - 0.2348)**2 + 3.5 * (x_2 - 0.1451)**2$ 
$+ 1.7 * (x_3 - 0.3522)**2 + 10.0 * (x_4 - 0.2883)**2$
$+ 17.0 * (x_5 - 0.3047)**2 + 8.0 * (x_6 - 0.665)**2$ in
let $e4 = 17.0 * (x_1 - 0.4047)**2 + 8.0 * (x_2 - 0.8828)**2$ 
$+ 0.05 * (x_3 - 0.8732)**2 + 10.0 * (x_4 - 0.5743)**2$
$+ 0.1 * (x_5 - 0.1091)**2 + 14.0 * (x_6 - 0.0381)**2$ in
$- (1.0 * \exp(-e1) + 1.2 * \exp(-e2)$
$+ 3.0 * \exp(-e3) + 3.2 * \exp(-e4)), \, 4.48\text{e--}13)];;$

let box_cav10 $x = [(0, 10)];;$
let obj_cav10 $x = [($ if $(x*x - x > 0)$ then $x*0.1$ 
else $x*x+2, \, 2.91)];;$

let box_perin $x \, y = [(1,7); (-2, 7)];;$
let cstr_perin $x \, y = [x-1; y+2; x-y; 5-y-x];;$
let obj_perin $x \, y = [($ if $(x*x + y*y \leq 4)$ then $y * x$
else $0, \, 2.01)];; $


\end{lstlisting}
}

\if{
{\scriptsize
\begin{lstlisting}
let box_doppler1 u v T = [$(-100, 100); (20, 2e4);(-30, 50)$];; 
let obj_doppler1 u v T = [(
let $t_1 = 331.4 + 0.6 * T$ in 
$-t_1*v/((t_1 + u)*(t_1 + u))$, $3\text{e--}13$)];;
\end{lstlisting}
}
{\scriptsize
\begin{lstlisting}
procedure doppler2(u : real, v : real, T : real) returns (r : real) {
 assume (-125.0 <= u && u <= 125.0 && 15.0 <= v && v <= 25000.0 && -40.0 <= T && T <= 60.0);
  var t1 := 331.4 + 0.6 * T;
  r := -t1*v/((t1 + u)*(t1 + u));
}
\end{lstlisting}
}
{\scriptsize
\begin{lstlisting}
procedure doppler3(u : real, v : real, T : real) returns (r : real) {
 assume (-30.0 <= u && u <= 120.0 && 320.0 <= v && v <= 20300.0 && -50.0 <= T && T <= 30.0);
  var t1 := 331.4 + 0.6 * T;
  r := -t1*v/((t1 + u)*(t1 + u));
}
\end{lstlisting}
}
{\scriptsize
\begin{lstlisting}
procedure rigidBody1(x_1 : real, x_2 : real, x_3 : real) returns (r : real) {
 assume (-15.0 <= x_1 && x_1 <= 15.0 && -15.0 <= x_2 && x_2 <= 15.0 && -15.0 <= x_3 && x_3 <= 15.0);
  r := -x_1*x_2 - 2.0 * x_2 * x_3 - x_1 - x_3;
}
\end{lstlisting}
}
{\scriptsize
\begin{lstlisting}
procedure rigidBody2(x_1 : real, x_2 : real, x_3 : real) returns (r : real) {
 assume (-15.0 <= x_1 && x_1 <= 15.0 && -15.0 <= x_2 && x_2 <= 15.0 && -15.0 <= x_3 && x_3 <= 15.0);
  r := 2.0*x_1*x_2*x_3 + 3.0*x_3*x_3 - x_2*x_1*x_2*x_3 + 3.0*x_3*x_3 - x_2;
}
\end{lstlisting}
}
{\scriptsize
\begin{lstlisting}
procedure sineTaylor(x : real) returns (r : real) {
 assume (-1.57079632679  <= x && x <= 1.57079632679);
    r := x - (x*x*x)/6.0 + (x*x*x*x*x)/120.0 - (x*x*x*x*x*x*x)/5040.0 ;
}
\end{lstlisting}
}
{\scriptsize
\begin{lstlisting}
procedure sineOrder3(x : real) returns (r : real) {
 assume (-2  <= x && x <= 2);
  r := 0.954929658551372 * x - 0.12900613773279798*x*x*x ;
}
\end{lstlisting}
}
{\scriptsize
\begin{lstlisting}
procedure sqroot(x : real) returns (r : real) {
 assume (0  <= x && x <= 1);
  r := 1.0 + 0.5*x - 0.125*x*x + 0.0625*x*x*x - 0.0390625*x*x*x*x;
}
\end{lstlisting}
}
}\fi


\end{document}

%                       Revision History
%                       -------- -------
%  Date         Person  Ver.    Change
%  ----         ------  ----    ------

%  2013.06.29   TU      0.1--4  comments on permission/copyright notices

