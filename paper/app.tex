%-----------------------------------------------------------------------------
%
%               Template for sigplanconf LaTeX Class
%
% Name:         sigplanconf-template.tex
%
% Purpose:      A template for sigplanconf.cls, which is a LaTeX 2e class
%               file for SIGPLAN conference proceedings.
%
% Guide:        Refer to "Author's Guide to the ACM SIGPLAN Class,"
%               sigplanconf-guide.pdf
%
% Author:       Paul C. Anagnostopoulos
%               Windfall Software
%               978 371-2316
%               paul@windfall.com
%
% Created:      15 February 2005
%
%-----------------------------------------------------------------------------


\documentclass[preprint,fleqn,nocopyrightspace]{sigplanconf}

% The following \documentclass options may be useful:

% preprint      Remove this option only once the paper is in final form.
% 10pt          To set in 10-point type instead of 9-point.
% 11pt          To set in 11-point type instead of 9-point.
% authoryear    To obtain author/year citation style instead of numeric.
\usepackage{listings}
\def\lstlanguagefiles{defManOcaml.tex}
\lstset{language = Ocaml}
\newcommand{\code}[1]{\lstinline{#1}}
%\begin{lstlisting}\end{lstlisting}

\usepackage[utf8]{inputenc}
\usepackage[T1]{fontenc}
\usepackage{lmodern}
\usepackage{graphicx}  % for pdf, bitmapped graphics files
\usepackage{amsmath} % assumes amsmath package installed
\usepackage{amssymb}  % assumes amsmath package installed
\usepackage{color}
\usepackage{enumitem}
\usepackage{myalgo}
\let\proof\relax
\let\endproof\relax
\usepackage{amsthm}
%\usepackage{ntheorem}
\usepackage{hyperref}
\usepackage{subfigure}
\usepackage{enumerate}
%\usepackage{caption}
\usepackage{multirow}
\usepackage{tikz, pgfplots}
\pgfplotsset{compat=1.8}
%\usepackage{natbib} %for bibliography via bibtex
%\usepackage{enumitem}
%\lstMakeShortInline{$}
\newcommand{\add}[1]{#1}
\newcommand{\del}[1]{\textcolor{gray}{#1}}

%\newcommand{\P}{\mathbb{P}}
%\newcommand{\Q}{\mathbb{Q}}
%\newcommand{\E}{\mathbb{E}}
\def\sizefig{0.35}
\def\sizesmallfig{0.30}
\def\sizetinyfig{0.24}
\newcommand{\setA}{\mathcal{A}} % Semialgebraic functions
\newcommand{\setD}{\mathcal{D}} % Dictionnary of Univariate transcendental functions
\newcommand{\setU}{\mathcal{U}} % Univariate functions := \setT \cup {sqrt, abs, power functions}
\newcommand{\suppf}[1]{\text{supp}(#1)}
\newcommand{\mons}[2]{\N_{#1}^{#2}}
\newcommand{\R}{\mathbb{R}}
\newcommand{\F}{\mathbb{F}}
\newcommand{\Pcal}{\mathcal{P}}
\newcommand{\Pcalp}{\Pcal^{(p)}}
\newcommand{\Scal}{\mathcal{S}}
\newcommand{\N}{\mathbb{N}}
\newcommand{\x}{\mathbf{x}}
\newcommand{\e}{\mathbf{e}}
\newcommand{\y}{\mathbf{y}}
\newcommand{\z}{\mathbf{z}}
\newcommand{\nbenchs}{23}
\newcommand{\alphab}{\boldsymbol{\alpha}}
\newcommand{\epsilonb}{\boldsymbol{\epsilon}}
\newcommand{\deltab}{\boldsymbol{\delta}}
\renewcommand{\b}{\mathbf{b}}
\newcommand{\f}{\mathbf{f}} 
\newcommand{\Plam}{\P_{\lambda}}
\newcommand{\Plamp}{\P_{\lambda}^{(p)}}
\def\P{\mathbf{P}}
\def\Q{\mathbf{Q}}
\def\L{\mathbf{L}}
\def\D{\mathbf{D}}
\def\q{\mathbf{q}}
\newcommand{\M}{\mathbf{M}}
\def\m{\mathbf{m}}
\def\H{\mathbf{H}}
\def\h{\mathbf{h}}
\def\f{f}
\def\a{\mathbf{a}}
\def\m{\mathbf{m}}
\def\p{\mathbf{p}}
\def\S{\mathbf{S}}
\def\B{\mathbf{B}}
\def\E{\mathbf{E}}
\def\K{\mathbf{K}}
\def\S{\mathbf{S}}
\def\Q{\mathbf{Q}}
\def\X{\mathbf{X}}
\def\Y{\mathbf{Y}}
\newcommand{\A}{\mathbf{A}}
\newcommand{\Shat}{\hat{\S}}
\newcommand{\Bb}{\mathbf{B}}
\newcommand{\flam}{{f}_{{\lambda}}}
\newcommand{\flamfun}[1]{f_{\lambda}(#1)}
\newcommand{\flamfunp}[2]{f^{(#2)}_{\lambda}(#1)}
\newcommand{\flamx}{\flamfun{\x}}
\newcommand{\flampx}{\flamfunp{\x}{p}}
\newcommand{\flamstar}{f^*({\lambda})}
\newcommand{\flamstarj}[1]{f_{#1}^*({\lambda})}
\newcommand{\Mcal}{\mathcal{M}}
\newcommand{\nsdp}{n_{\text{sdp}}}
\newcommand{\divzero}{\text{Div0}}
\newcommand{\nlift}{n_{\text{lift}}}
\newcommand{\Slift}{\S_{\text{lift}}}
\renewcommand{\prec}{\text{prec}}
\newcommand{\mlift}{m_{\text{lift}}}
\newcommand{\dlift}{r_{\text{lift}}}
\newcommand{\msdp}{m_{\text{sdp}}}
\newcommand{\xlamstar}{{\x}^*({\lambda})}
\newcommand{\Kpol}{\K_{\text{poly}}}
\newcommand{\transpose}{\top}%\newcommand{\transpose}{\mathbf{\intercal}}
\DeclareMathOperator{\vol}{vol}
%\DeclareMathOperator{\op}{op}
\DeclareMathOperator{\conv}{conv}
\newcommand{\red}[1]{\textbf{{\color{red}#1}}}
\newcommand{\iaboundfun}[2]{\mathtt{ia\_bound}(#1, #2)}
\newcommand{\iabound}{\mathtt{ia\_bound}}
\newcommand{\sdpboundfun}[3]{\mathtt{sdp\_bound}(#1, #2, #3)}
\newcommand{\sdpbound}{\mathtt{sdp\_bound}}
\newcommand{\boundfun}[7]{\mathtt{bound}(#1, #2, #3, #4, #5, #6, #7)}
\newcommand{\bound}{\mathtt{bound}}
\newcommand{\boundnlprogfun}[7]{\mathtt{bound\_nlprog}(#1, #2, #3, #4, #5, #6, #7)}
\newcommand{\boundnlprog}{\mathtt{bound\_nlprog}}
\newcommand{\sdppolyfun}[3]{\mathtt{sdp\_poly}(#1, #2, #3)}
\newcommand{\sdppoly}{\mathtt{sdp\_poly}}
\newcommand{\liftfun}[3]{\mathtt{lift}(#1, #2, #3)}
\newcommand{\lift}{\mathtt{lift}}
\newcommand{\poly}{_\text{poly}}
\newcommand{\sa}{_\text{sa}}
\newcommand{\sdpsafun}[3]{\mathtt{sdp\_sa}(#1, #2, #3)}
\newcommand{\sdpsa}{\mathtt{sdp\_sa}}
\newcommand{\sdptranscfun}[3]{\mathtt{sdp\_trancs}(#1, #2, #3)}
\newcommand{\sdptransc}{\mathtt{sdp\_transc}}
%\newcommand{\brev}{\color{red}}
%\newcommand{\erev}{\color{black}}
\newcommand{\sthreefp}{\mathtt{s3fp}}
\newcommand{\realtofloat}{\mathtt{Real2Float}}
\newcommand{\smtcoq}{\mathtt{smtcoq}}
\newcommand{\dreal}{\mathtt{dReal}}
\newcommand{\hol}{\text{\sc Hol-light}}
\newcommand{\op}{\mathtt{op}}
\newcommand{\bop}{\mathtt{bop}}
\newcommand{\coq}{\text{\sc Coq}}
\newcommand{\ocaml}{\text{\sc OCaml}}
\newcommand{\rosa}{\mathtt{Rosa}}
\newcommand{\sdpa}{\text{\sc Sdpa}}
\newcommand{\fptaylor}{\mathtt{FPTaylor}}
\newcommand{\nlcertify}{\mathtt{NLCertify}}
\makeatletter
\newcommand*{\circled}{\@ifstar\circledstar\circlednostar}
\newcommand*{\squared}{\@ifstar\squaredstar\squarednostar}
\makeatother

\newcommand*\circledstar[1]{%
  \tikz[baseline=(C.base)]
    \node[%
      fill,
      circle,
      minimum size=1.em,
      text=white,
%      font=\sffamily,
      inner sep=0.5pt
    ](C) {\texttt{#1}};%
}
\newcommand*\circlednostar[1]{%
  \tikz[baseline=(C.base)]
    \node[%
      draw,
      circle,
      minimum size=1.em,
%      font=\sffamily,
      inner sep=0.5pt
    ](C) {\texttt{#1}};%
}
\newcommand*\squaredstar[1]{%
  \tikz[baseline=(C.base)]
    \node[%
      fill,
      rectangle,
      minimum size=1.em,
      text=white,
%      font=\sffamily,
      inner sep=0.5pt
    ](C) {\texttt{#1}};%
}
\newcommand*\squarednostar[1]{%
  \tikz[baseline=(C.base)]
    \node[%
      draw,
      rectangle,
      minimum size=1.em,
%      font=\sffamily,
      inner sep=0.5pt
    ](C) {\texttt{#1}};%
}
%\newcommand{\II}{\mathbb{I}}
%\theoremstyle{plain}
%\newtheorem{thm}{Theorem}[section]
\newtheorem{theorem}{Theorem}[section]
\newtheorem{lemma}[theorem]{Lemma}
\newtheorem{proposition}[theorem]{Proposition}
\newtheorem{corollary}[theorem]{Corollary}

\theoremstyle{plain}
\newtheorem{definition}[theorem]{Definition}
\newtheorem{hypothesis}[theorem]{Assumption}
\newtheorem{conjecture}[theorem]{Conjecture}
\newtheorem{assumption}[theorem]{Assumption}
\newtheorem{example}{Example}

%\theoremstyle{remark}
\newtheorem{remark}{Remark}
%\newtheorem*{note}{Note}
\newtheorem{case}{Case}




\begin{document}

\conferenceinfo{CONF 'yy}{Month d--d, 20yy, City, ST, Country} 
\copyrightyear{20yy} 
\copyrightdata{978-1-nnnn-nnnn-n/yy/mm} 
\doi{nnnnnnn.nnnnnnn}

% Uncomment one of the following two, if you are not going for the 
% traditional copyright transfer agreement.

%\exclusivelicense                % ACM gets exclusive license to publish, 
                                  % you retain copyright

%\permissiontopublish             % ACM gets nonexclusive license to publish
                                  % (paid open-access papers, 
                                  % short abstracts)

%\titlebanner{banner above paper title}        % These are ignored unless
%\preprintfooter{short description of paper}   % 'preprint' option specified.
\title{}
%\title{Automated Precision Tuning using Semidefinite Programming}
%\title{Certified Roundoff Error Bounds Using Semidefinite Programming}

%\subtitle{Subtitle Text, if any}

\authorinfo{}{}{}
%\authorinfo{Name2\and Name3}
%           {Affiliation2/3}
%           {Email2/3}

\maketitle
\appendix
\section*{Appendix A: Nonlinear Programs}
\label{sec:appa}
%
{\scriptsize
\begin{lstlisting}
let box_doppler1 u v T = [$(-100, 100); (20, 2e4);(-30, 50)$];; 
let obj_doppler1 u v T = [(
let $t_1 = 331.4 + 0.6 * T$ in 
$-t_1*v/((t_1 + u)*(t_1 + u))$, $3e-13$)];;
\end{lstlisting}
}
{\scriptsize
\begin{lstlisting}
procedure doppler2(u : real, v : real, T : real) returns (r : real) {
 assume (-125.0 <= u && u <= 125.0 && 15.0 <= v && v <= 25000.0 && -40.0 <= T && T <= 60.0);
  var t1 := 331.4 + 0.6 * T;
  r := -t1*v/((t1 + u)*(t1 + u));
}
\end{lstlisting}
}
{\scriptsize
\begin{lstlisting}
procedure doppler3(u : real, v : real, T : real) returns (r : real) {
 assume (-30.0 <= u && u <= 120.0 && 320.0 <= v && v <= 20300.0 && -50.0 <= T && T <= 30.0);
  var t1 := 331.4 + 0.6 * T;
  r := -t1*v/((t1 + u)*(t1 + u));
}
\end{lstlisting}
}
{\scriptsize
\begin{lstlisting}
procedure rigidBody1(x1 : real, x2 : real, x3 : real) returns (r : real) {
 assume (-15.0 <= x1 && x1 <= 15.0 && -15.0 <= x2 && x2 <= 15.0 && -15.0 <= x3 && x3 <= 15.0);
  r := -x1*x2 - 2.0 * x2 * x3 - x1 - x3;
}
\end{lstlisting}
}
{\scriptsize
\begin{lstlisting}
procedure rigidBody2(x1 : real, x2 : real, x3 : real) returns (r : real) {
 assume (-15.0 <= x1 && x1 <= 15.0 && -15.0 <= x2 && x2 <= 15.0 && -15.0 <= x3 && x3 <= 15.0);
  r := 2.0*x1*x2*x3 + 3.0*x3*x3 - x2*x1*x2*x3 + 3.0*x3*x3 - x2;
}
\end{lstlisting}
}
{\scriptsize
\begin{lstlisting}
procedure sineTaylor(x : real) returns (r : real) {
 assume (-1.57079632679  <= x && x <= 1.57079632679);
    r := x - (x*x*x)/6.0 + (x*x*x*x*x)/120.0 - (x*x*x*x*x*x*x)/5040.0 ;
}
\end{lstlisting}
}
{\scriptsize
\begin{lstlisting}
procedure sineOrder3(x : real) returns (r : real) {
 assume (-2  <= x && x <= 2);
  r := 0.954929658551372 * x - 0.12900613773279798*x*x*x ;
}
\end{lstlisting}
}
{\scriptsize
\begin{lstlisting}
procedure sqroot(x : real) returns (r : real) {
 assume (0  <= x && x <= 1);
  r := 1.0 + 0.5*x - 0.125*x*x + 0.0625*x*x*x - 0.0390625*x*x*x*x;
}
\end{lstlisting}
}

\end{document}

%                       Revision History
%                       -------- -------
%  Date         Person  Ver.    Change
%  ----         ------  ----    ------

%  2013.06.29   TU      0.1--4  comments on permission/copyright notices

