%-----------------------------------------------------------------------------
%
%               Template for sigplanconf LaTeX Class
%
% Name:         sigplanconf-template.tex
%
% Purpose:      A template for sigplanconf.cls, which is a LaTeX 2e class
%               file for SIGPLAN conference proceedings.
%
% Guide:        Refer to "Author's Guide to the ACM SIGPLAN Class,"
%               sigplanconf-guide.pdf
%
% Author:       Paul C. Anagnostopoulos
%               Windfall Software
%               978 371-2316
%               paul@windfall.com
%
% Created:      15 February 2005
%
%-----------------------------------------------------------------------------


\documentclass[preprint]{sigplanconf}

% The following \documentclass options may be useful:

% preprint      Remove this option only once the paper is in final form.
% 10pt          To set in 10-point type instead of 9-point.
% 11pt          To set in 11-point type instead of 9-point.
% authoryear    To obtain author/year citation style instead of numeric.

\usepackage{amsmath}

\usepackage{listings}
\def\lstlanguagefiles{defManOcaml.tex}
\lstset{language = Ocaml}
\newcommand{\code}[1]{\lstinline{#1}}
%\begin{lstlisting}\end{lstlisting}

\usepackage[utf8]{inputenc}
\usepackage[T1]{fontenc}
\usepackage{lmodern}
\usepackage{graphicx}  % for pdf, bitmapped graphics files
\usepackage{amsmath} % assumes amsmath package installed
\usepackage{amssymb}  % assumes amsmath package installed
\usepackage{color}
\usepackage{myalgo}
\let\proof\relax
\let\endproof\relax
%\usepackage{amsthm}
\usepackage{ntheorem}
\usepackage{hyperref}
\usepackage{subfigure}
\usepackage{enumerate}
%\usepackage{caption}
\usepackage{multirow}
%\usepackage{natbib} %for bibliography via bibtex
%\usepackage{enumitem}
%\lstMakeShortInline{$}
\newcommand{\add}[1]{#1}
\newcommand{\del}[1]{\textcolor{gray}{#1}}

%\newcommand{\P}{\mathbb{P}}
%\newcommand{\Q}{\mathbb{Q}}
%\newcommand{\E}{\mathbb{E}}
\def\sizefig{0.35}
\def\sizesmallfig{0.30}
\def\sizetinyfig{0.24}
\newcommand{\setA}{\mathcal{A}} % Semialgebraic functions
\newcommand{\setD}{\mathcal{D}} % Dictionnary of Univariate transcendental functions
\newcommand{\setU}{\mathcal{U}} % Univariate functions := \setT \cup {sqrt, abs, power functions}
\newcommand{\suppf}[1]{\text{supp}(#1)}
\newcommand{\mons}[2]{\N_{#1}^{#2}}
\newcommand{\R}{\mathbb{R}}
\newcommand{\F}{\mathbb{F}}
\newcommand{\Pcal}{\mathcal{P}}
\newcommand{\Pcalp}{\Pcal^{(p)}}
\newcommand{\Scal}{\mathcal{S}}
\newcommand{\N}{\mathbb{N}}
\newcommand{\x}{\mathbf{x}}
\newcommand{\e}{\mathbf{e}}
\newcommand{\y}{\mathbf{y}}
\newcommand{\z}{\mathbf{z}}
\newcommand{\alphab}{\boldsymbol{\alpha}}
\newcommand{\epsilonb}{\boldsymbol{\epsilon}}
\newcommand{\deltab}{\boldsymbol{\delta}}
\renewcommand{\b}{\mathbf{b}}
\newcommand{\f}{\mathbf{f}} 
\newcommand{\Plam}{\P_{\lambda}}
\newcommand{\Plamp}{\P_{\lambda}^{(p)}}
\def\P{\mathbf{P}}
\def\Q{\mathbf{Q}}
\def\L{\mathbf{L}}
\def\D{\mathbf{D}}
\def\q{\mathbf{q}}
\newcommand{\M}{\mathbf{M}}
\def\m{\mathbf{m}}
\def\H{\mathbf{H}}
\def\h{\mathbf{h}}
\def\f{f}
\def\a{\mathbf{a}}
\def\m{\mathbf{m}}
\def\p{\mathbf{p}}
\def\S{\mathbf{S}}
\def\B{\mathbf{B}}
\def\E{\mathbf{E}}
\def\K{\mathbf{K}}
\def\S{\mathbf{S}}
\def\Q{\mathbf{Q}}
\def\X{\mathbf{X}}
\def\Y{\mathbf{Y}}
\newcommand{\A}{\mathbf{A}}
\newcommand{\Shat}{\hat{\S}}
\newcommand{\Bb}{\mathbf{B}}
\newcommand{\flam}{{f}_{{\lambda}}}
\newcommand{\flamfun}[1]{f_{\lambda}(#1)}
\newcommand{\flamfunp}[2]{f^{(#2)}_{\lambda}(#1)}
\newcommand{\flamx}{\flamfun{\x}}
\newcommand{\flampx}{\flamfunp{\x}{p}}
\newcommand{\flamstar}{f^*({\lambda})}
\newcommand{\flamstarj}[1]{f_{#1}^*({\lambda})}
\newcommand{\Mcal}{\mathcal{M}}
\newcommand{\nsdp}{n_{\text{sdp}}}
\newcommand{\nan}{\text{NaN}}
\newcommand{\nlift}{n_{\text{lift}}}
\newcommand{\Slift}{\S_{\text{lift}}}
\renewcommand{\prec}{\text{prec}}
\newcommand{\mlift}{m_{\text{lift}}}
\newcommand{\dlift}{r_{\text{lift}}}
\newcommand{\msdp}{m_{\text{sdp}}}
\newcommand{\xlamstar}{{\x}^*({\lambda})}
\newcommand{\transpose}{\top}%\newcommand{\transpose}{\mathbf{\intercal}}
\DeclareMathOperator{\vol}{vol}
%\DeclareMathOperator{\op}{op}
\DeclareMathOperator{\conv}{conv}
\newcommand{\red}[1]{\textbf{{\color{red}#1}}}
\newcommand{\iaboundfun}[2]{\mathtt{ia\_bound}(#1, #2)}
\newcommand{\sdpboundfun}[3]{\mathtt{sdp\_bound}(#1, #2, #3)}
\newcommand{\iabound}{\mathtt{ia\_bound}}
\newcommand{\sdpbound}{\mathtt{sdp\_bound}}
%\newcommand{\brev}{\color{red}}
%\newcommand{\erev}{\color{black}}
\newcommand{\sthreefp}{\mathtt{s3fp}}
\newcommand{\realtofloat}{\mathtt{Real2Float}}
\newcommand{\hol}{\text{\sc Hol-light}}
\newcommand{\op}{\mathtt{op}}
\newcommand{\bop}{\mathtt{bop}}
\newcommand{\coq}{\text{\sc Coq}}
\newcommand{\ocaml}{\text{\sc OCaml}}
\newcommand{\rosa}{\mathtt{rosa}}
\newcommand{\sdpa}{\text{\sc sdpa}}
\newcommand{\fptaylor}{\mathtt{FPTaylor}}
\newcommand{\nlcertify}{\mathtt{NLCertify}}
%\newcommand{\II}{\mathbb{I}}
%\theoremstyle{plain}
\newtheorem{thm}{Theorem}[section]
\newtheorem{theorem}{Theorem}[section]
\newtheorem{lemma}[theorem]{Lemma}
\newtheorem{proposition}[theorem]{Proposition}
\newtheorem{corollary}[theorem]{Corollary}

\theoremstyle{plain}
\newtheorem{definition}[theorem]{Definition}
\newtheorem{hypothesis}[theorem]{Assumption}
\newtheorem{conjecture}[theorem]{Conjecture}
\newtheorem{assumption}[theorem]{Assumption}
\newtheorem{example}{Example}

%\theoremstyle{remark}
\newtheorem{remark}{Remark}
%\newtheorem*{note}{Note}
\newtheorem{case}{Case}




\begin{document}

\special{papersize=8.5in,11in}
\setlength{\pdfpageheight}{\paperheight}
\setlength{\pdfpagewidth}{\paperwidth}

\conferenceinfo{CONF 'yy}{Month d--d, 20yy, City, ST, Country} 
\copyrightyear{20yy} 
\copyrightdata{978-1-nnnn-nnnn-n/yy/mm} 
\doi{nnnnnnn.nnnnnnn}

% Uncomment one of the following two, if you are not going for the 
% traditional copyright transfer agreement.

%\exclusivelicense                % ACM gets exclusive license to publish, 
                                  % you retain copyright

%\permissiontopublish             % ACM gets nonexclusive license to publish
                                  % (paid open-access papers, 
                                  % short abstracts)

%\titlebanner{banner above paper title}        % These are ignored unless
%\preprintfooter{short description of paper}   % 'preprint' option specified.

\title{Automated Precision Tuning using Semidefinite Programming}
%\subtitle{Subtitle Text, if any}

\authorinfo{Name1}
           {Affiliation1}
           {Email1}
\authorinfo{Name2\and Name3}
           {Affiliation2/3}
           {Email2/3}

\maketitle

\begin{abstract}
This is the text of the abstract.
\end{abstract}

\category{CR-number}{subcategory}{third-level}

% general terms are not compulsory anymore, 
% you may leave them out
%\terms
%term1, term2

\keywords
hardware precision tuning; round-off error; numerical accuracy; floating-point arithmetic; fixed-precision arithmetic; semidefinite programming; sums of squares; correlation sparsity pattern.
\section{Introduction} %3p
\label{sec:intro}
%
Constructing numerical programs which output accurate computation turns out to be difficult, due to finite numerical precision of implementations such as floating-point or fixed-point representations. Finite-precision numbers induce rounding errors, whose range knowledge is required to fulfil safety criteria of critical programs, as typically arising in modern embedded systems such as aircraft controllers.
To obtain lower bounds over rounding errors, one can rely on testing approaches, such as meta-heuristic search~\cite{Borges12Test} or under-approximations tools (e.g.~$\sthreefp$~\cite{Chiang14s3fp}). Here, we are interested in handling efficiently the complementary over-approximation problem, namely to obtain tight upper bounds over the error. This problem boils down to finding tight abstractions of non-linearities while being able to bound the resulting approximations in a efficient way.  
%


The aim of this work is to provide such a framework to perform automated precision analysis of computer programs consisting of nonlinear operations. 
This framework can be applied in general for developing accurate numerical software, but appears to be particularly relevant while considering algorithms migration onto reconfigurable hardware (e.g. FPGAs). The advantage of architectures based on FPGAs is that they allow more flexible choices, rather than choosing either for IEEE standard single or double precision. Indeed, in this case, one benefits from a more flexible number representation while ensuring guaranteed bounds on the program output. 

For computer programs consisting of linear operations, automatic error analysis can be obtained with well-studied optimization techniques based on SAT/SMT solvers~\cite{Darulova14Popl,hgbk2012fmcad}, Taylor-interval methods~\cite{fptaylor15}, affine arithmetic~\cite{fluctuat} and linear programming relaxations~\cite{Boland10HGR}.
These techniques can also be applied in the nonlinear setting but do not fully take into account the correlations between program variables. Thus they may output coarse error bounds or perform analysis within a large amount of time.  Another drawback of these tools is that they usually provide no formal guarantees. At best, they can produce proof certificates but the formal verification of these certificates is computationally expensive. It follows from the fact that most of computation performed in the informal optimization procedure are redone inside the proof assistant.

For computer programs with nonlinear operations, guarantees can be provided with certified programming techniques.
Semidefinite programming (SDP) is relevant to a wide range of mathematical fields, including combinatorial optimization, control theory, matrix completion. In 2001, Lasserre introduced a hierarchy of SDP relaxations~\cite{Lasserre01moments} for approximating polynomial infima. Our method to bound the error is a decision procedure based on an specialized variant of Lasserre hierarchy. The procedure relies on SDP to provide sparse sum-of-squares decompositions of positive polynomials. This framework handles polynomial program analysis (involving the operations $+,\times,-$) but can be extended to the more general class of semialgebraic or transcendental programs (involving $\sqrtsign, /, \min, \max, \arctan, \exp$), following the approximation scheme described in~\cite{Magron15sdp}.

\subsection{Overview of our Method}
%
Consider the program implementing the following polynomial expression $f$:
\begin{align*}
f(\x) := x_2 \times x_5 + x_3 \times x_6 - x_2 \times x_3  - x_5 \times x_6 \\
+ x_1 \times ( - x_1 +  x_2 +  x_3  - x_4 +  x_5 +  x_6) \,,
\end{align*}
%
where the six variable vector $\x :=  (x_1, x_2, x_3, x_4, x_5, x_6)$ is the input of the program. Here, the set $\X$ of possible input values is taken as a product of closed intervals: $\X = [4.00, 6.36]^6$ but could be defined in general with a set of inequality constraints among the variables $x_1, \dots, x_6$. 

The polynomial expression $f$ is obtained by performing 15 basic operations (1 negation, 3 subtractions, 6 additions and 5 multiplications). 
When executing this program with a set of floating-point numbers $\hat{\x} :=  (\hat{x}_1, \hat{x}_2, \hat{x}_3, \hat{x}_4, \hat{x}_5, \hat{x}_6) \in \X$, one actually computes a floating-point result $\hat{f}$, where all operations $+, -, \times$ are replaced by the respectively associated floating-point operations $\oplus, \ominus, \otimes$. 
The results of these operations comply with IEEE 754 standard arithmetic~\cite{IEEE} (see relevant background in Section~\ref{sec:fpbackground}). For instance, one can write $\hat{x}_2 \otimes \hat{x}_5 =  (x_2 \times x_5) (1 + e_1)$, by introducing an error variable $e_1$ such that $-\epsilon \leq e_1 \leq \epsilon$, where the bound $\epsilon$ is the machine precision (e.g.~$2^{-24}$ for single precision). One would like to bound the absolute rounding error $|r(\x, \e)| := | \hat{f}(\x, \e) - f (\x) |$ over  all possible input variables $\x \in \X$ and error variables $e_1, \dots, e_{15} \in [-\epsilon, \epsilon]$. Let define $\E := [-\epsilon, \epsilon]^{15}$ and $\K := \X \times \E$, then note that our bound problem can be cast as a nonlinear optimization problem:
%
\begin{align}
\begin{split}
\label{eq:roptim}
r^\star := & \max_{(\x, \e) \in \K} | r(\x, \e) | \\
 = & \ \ \max \{-\min_{(\x, \e) \in \K} r(\x, \e), \max_{(\x, \e) \in \K} r(\x,\e)\} \enspace. 
\end{split}
\end{align}
%
One can directly try to solve these two polynomial optimization problems using classical SDP relaxations~\cite{Lasserre01moments}.
As in~\cite{fptaylor15}, one can also decompose the error term $r$ as the sum of a term $l(\x,\e)$, which is linear w.r.t.~$\e$, and a nonlinear term $h(\x,\e) := r(\x,\e) - l(\x,\e)$. Then the triangular inequality yields:
%
\begin{equation}
\label{eq:lhoptim} 
r^\star \leq \max_{(\x, \e) \in \K} |l(\x, \e)| + \max_{(\x, \e) \in \K} |h(\x, \e)| \enspace. 
\end{equation}
%
It follows that $l(\x,\e) = x_2 x_5 e_1 + x_3 x_6 e_2 +  (x_2 x_5 + x_3 x_6) e_3 + \dots + f(\x) e_{15} = \sum_{i=1}^{15} s_i(\x) e_i$. The {\em Symbolic Taylor expansions} method~\cite{fptaylor15} consists in using Taylor-interval optimization to compute a rigorous interval enclosure of each polynomial $s_i$, $i = 1,\dots,15$, over $\X$ and finally obtain an upper bound of $|l| + |h|$ over $\K$.
% over $\X \times \E$
\begin{itemize}
\item A direct attempt to solve the two polynomial problems occurring in Equation~\eqref{eq:roptim} fails as the SDP solver (in our case $\sdpa$~\cite{sdpa7}) runs out of memory. 
\item Using sparse semidefinite optimization (derived from \cite{Las06SparseSOS}) to bound $l$ and basic interval arithmetic to bound $h$, one obtains an upper bound of $789 \epsilon$ for $|l| + |h|$ over $\K$. This computation is performed in less than one second on a modern computer.
\item Using basic interval arithmetic, one obtains more quickly (about 17 times less CPU) a coarser bound of $2023 \epsilon$. 
\item Symbolic Taylor expansions provide an intermediate bound of $936 \epsilon$ but it takes about 16.3 times more CPU than our method. 
\item Finally, our bound is also obtained with the $\rosa$ tool~\cite{Darulova14Popl} but it takes about 4.6 times more CPU.
\end{itemize}

\if{
For instance, given a four variable vector $\x := (x_1, x_2, x_3, x_4)$, consider the  determinant computation of the $2 \times 2$ matrix  $\begin{pmatrix}
x_1 & x_2 \\
x_3 & x_4 \\
\end{pmatrix}$.
%
Then, the polynomial $f$ which represents the absolute floating-point error is given by $f(\x, \epsilonb) := [x_1 x_4 (1 + \epsilon_1) - x_3 x_2 (1 + \epsilon_2)] (1 + \epsilon_3) - x_1 x_4 + x_3 x_2$.
This polynomial $f$ is already of degree 3 and involves 7 variables.
}\fi




\subsection{Contributions}
Our key contributions can be summarized as follows:
\begin{itemize}
\item We present an optimization algorithm providing certified over-approximations for round-off errors of nonlinear programs. This algorithm is based on sparse sums of squares programming. By comparison with other methods, our algorithm allows to obtain tighter upper bounds, while overcoming scalability and numerical issues inherent to SDP solvers. We also propose some extensions of our algorithm to programs involving non-polynomial components, including either semialgebraic or transcendental operations (e.g. $/, \sqrtsign, \arctan, \exp$), as well as conditionals.
%\item 
\item Our framework is fully implemented in the $\realtofloat$ tool.  Among several features, the tool can optionally perform formal verification of round-off error bounds for polynomial programs, inside the $\coq$ proof assistant~\cite{CoqProofAssistant}. The last software release of $\realtofloat$ provides $\ocaml$~\cite{OCaml} and $\coq$ libraries and is freely available from 
\begin{center}
\url{https://github.com/afd/real2float}.
\end{center}
%
Our implementation tool is built in top of the $\nlcertify$ verification system~\cite{icms14}. Precision and efficiency of the tool are evaluated on several benchmarks coming from the existing literature. Numerical experiments demonstrate that our method competes well with recent approaches relying on Taylor-interval approximations~\cite{Darulova14Popl} or combining SMT solvers with affine arithmetic~\cite{Darulova14Popl}.
\end{itemize}
%


The paper is organized as follows.
%
In Section~\ref{sec:background}, we recall mandatory background on rounding errors due to finite precision arithmetic before describing our nonlinear program semantics (Section~\ref{sec:fpbackground}). Then we explain how to perform certified polynomial optimization based on semidefinite programming (Section~\ref{sec:sdpbackground}) and finally how to obtain formal bounds while checking the certificates inside the $\coq$ proof assistant (Section~\ref{sec:coqbackground}).
%
Section~\ref{sec:fpsdp} contains the main contribution of the paper, namely how to compute tight over-approximations for rounding errors of nonlinear programs with sparse semidefinite relaxations.
%
Finally, Section~\ref{sec:benchs} is devoted to the evaluation of our nonlinear verification tool $\realtofloat$ on benchmarks arising from control systems, optimization, physics and biology.

%Verifying simple linear algebra algorithms can be troublesome from the  computational point of view.
\section{Preliminaries}
\label{sec:background}

\subsection{Program Semantics and Floating-point Arithmetic}
\label{sec:fpbackground}
We adopt the standard practice~\cite{higham2002accuracy} to approximate a real float $x$ with its closest floating-point representation $\hat{x} = x (1 + e)$, with $|e|$ being less than the machine precision $\epsilon$. This model is only valid when neglecting both overflow and denormal range values.
The operator $\hat{\cdot}$ is called the rounding operator and can be selected among rounding to nearest, rounding toward zero (resp.~$\infty$).
The scientific notation of a binary (resp.~decimal) floating-point number $\hat{x}$ is a triple $(s, sig, exp)$ consisting of a sign bit $s$, a {\em significand} $sig$ and an {\em exponent} $exp$, so that its numerical evaluation yields $(-1)^{s} \times sig \times 2^{exp}$ (resp.~$(-1)^{s} \times sig \times 10^{exp}$). 

The value of $\epsilon$ actually gives the upper bound of the relative floating-point error and is equal to $2^{-\prec}$, where $\prec$ is called the {\em precision}, referring to the number of significand bits used. For single floating-point precision, one has $\prec = 24$. For double (resp.~quadruple) precision, one has $\prec = 53$ (resp.~$\prec=113$). Let define $\R$ the set of real numbers and $\F$ the set of binary floating-point numbers.
For each real-valued operation $\bop_\R \in \{+, -, \times, \slash \}$ complying with IEEE 754 standard arithmetic~\cite{IEEE}, the result of the corresponding floating-point operation $\bop_\F \in \{\oplus, \ominus, \otimes, \oslash \}$ satisfies the following:
\begin{equation}
\label{eq:roundbop}
\bop_\F \, (\hat{x}, \hat{y}) = \bop_\R \, (x, y) \, (1 + e) \enspace, \quad \mid e \mid \leq \epsilon = 2^{-\prec} \enspace.
\end{equation}
%
Other operations include special functions taken from a {\em dictionary} $\setD$, containing the unary functions
$\tan$, $\arctan$, $\cos$, $\arccos$, $\sin$, $\arcsin$, $\exp$, $\log$, $(\cdot)^{r}$ with $r\in \R\setminus\{0\}$. For each $f_\R \in \setD$, the corresponding floating-point evaluation satisfies 
\begin{equation}
\label{eq:roundtransc}
f_\F (\hat{x}) = f_\R (x) (1 + e) \enspace, \quad \mid e \mid \leq \epsilon (f_\R) \enspace.
\end{equation}
%$f_\F (\hat{x}) = f_\R (x) (1 + e)$, with $|e| \leq \epsilon (f_\R)$. 
The value of the relative error bound $\epsilon (f_\R)$ differs from the machine precision $\epsilon$ in Equation~\eqref{eq:roundbop} and has to be properly adjusted. We refer the interested reader to~\cite{VerifCADTransc} for relative error bound verification of transcendental functions (see also~\cite{VerifHOLTransc} for formalization in $\hol$).
%
%In a similar fashion, 
\paragraph{Program semantics.}
%
We consider generic programs encoded in an ML-like language:
\begin{lstlisting}
let box_prog    $x_1 \dots x_n = [(a_1, b_1); \dots ; (a_n, b_n)]$;;
let obj_prog    $x_1 \dots x_n = [(f(\x), \epsilon_{\text{total}})]$;;
let cstr_prog   $x_1 \dots x_n = [g_1 (\x), \dots, g_k(\x)]$;;
let uncert_prog $x_1 \dots x_n = [u_1, \dots, u_n]$;;
\end{lstlisting}
Here, the first line encodes interval constraints for input variables, namely $\x := (x_1, \dots, x_n) \in [a_1, b_1]\times \dots \times [a_n, b_n]$.
The second line provides the function $f(\x)$ as well as the total rounding error bound $\epsilon_{\text{total}}$.
Then, one encodes polynomial nonnegativity constraints over the input variables, namely $g_1(\x) \geq 0, \dots, g_k(\x) \geq 0$. Finally, the last line allows the user to specify a numerical constant $u_i$ to associate a given uncertainty to the variable $x_i$, for each $i= 1, \dots, n$.

The type of numerical constants is denoted by \code{C}. In our current implementation, the user can choose either 64 bits floating-point or arbitrary-size rational numbers. This type \code{C} is used for the terms $\epsilon_{\text{total}}$, $u_1, \dots, u_n$, $a_1, \dots, a_n$, $b_1, \dots, b_n$.

The inductive type of polynomial expressions with coefficients in \code{C} is \code{polC} defined as follows:
\begin{lstlisting}
type polC = | Pc of C | Px of positive 
| Psub of polC$\,$*$\,$polC | Pneg of polC 
| Padd of polC$\,$*$\,$polC | Pmul of polC$\,$*$\,$polC
\end{lstlisting}
%
The constructor \code{Px} takes a positive integer as argument to represent either an input or local variable.
%The polynomial expressions $g_1(\x), \dots, g_k(\x)$ have this type \code{polC}.
The inductive type \code{nlexpr} of nonlinear expressions (such as $f(\x)$) is defined as follows:
\begin{lstlisting}
type nlexpr = 
| Pol of polC | Neg of nlexpr
| Add of nlexpr$\,$*$\,$nlexpr 
| Mul of nlexpr$\,$*$\,$nlexpr 
| Sub of nlexpr$\,$*$\,$nlexpr 
| Div of nlexpr$\,$*$\,$nlexpr | Sqrt of nlexpr 
| Transc of transc$\,$*$\,$nlexpr
| IfThenElse of polC$\,$*$\,$nlexpr$\,$*$\,$nlexpr
| Let of positive$\,$*$\,$nlexpr$\,$*$\,$nlexpr
\end{lstlisting}
%
The type \code{transc} corresponds to the dictionary $\setD$ of special functions. For instance, the term~\lstinline|Transc ($\exp$, $f(\x)$)| represents the program implementing $\exp(f(\x)$.
Given a polynomial expression $p$ and two nonlinear expressions $f$ and $g$, the term ~\lstinline|IfThenElse($p(\x)$, $f(\x)$, $g(\x)$)| represents the conditional program implementing~\lstinline|if ($p(\x) \leq 0$) $f (\x)$ else $g (\x)$|. The constructor \code{Let} allows to define local variables in an ML fashion, e.g.~\lstinline|let $t_1 = 331.4 + 0.6 T$ in $-t_1 v /(t_1 + u)^2$| (part of the Doppler program considered in Section~\ref{sec:benchs}).
%
%\end{lstlisting}
%

Finally, one obtains rounded nonlinear expressions using an inductive procedure~\lstinline|rounding : nlexpr $\to$ nlexpr|, defined accordingly to Equation~\eqref{eq:roundbop} and Equation~\eqref{eq:roundtransc}. When an uncertainty $u_i$ is specified for an input variable $x_i$, the corresponding rounded expression is given by $x_i \, (1 + e)$, with $\mid e \mid \, \leq u_i$.
%to when solving optimization problems involving maximal absolute rounding errors. allowing to consider a single error variable bounded using $(k + 1) \epsilon$, thus saving $(k - 1)$ error variables.

%One can exploit sparsity in a way similar to the one described in~\cite{Waki06SparseSOS,Las06SparseSOS} to handle high dimensional problems.
\subsection{Sparse semidefinite relaxations for polynomial optimization}
\label{sec:sdpbackground}
Here, we recall mandatory background about the method that we use to handle the optimization problem of Equation~\eqref{eq:roptim}, when the nonlinear function $r$ is a polynomial expression.

\paragraph{Sums of squares certificates and semidefinite programming.}
We remind basic facts about generation of sums of squares certificates for polynomial optimization, using semidefinite programming.
Denote by $\R[\x]$ the vector space of polynomials and by $\R_{2 d}[\x]$ the restriction of $\R[\x]$ to polynomials of degree at most $2 d$. Let define the cone of sums of squares:
\begin{equation}
\label{eq:cone_sos}
\Sigma[\x] := \Bigl\{\sum_i q_i^2, \, \text{ with } q_i \in \R[\x] \Bigr\}\enspace,
\end{equation}
%
as well as its restriction $\Sigma_{2 d}[\x] := \Sigma[\x] \bigcap \R_{2 d}[\x]$ to polynomials of degree at most $2 d$. For instance, the following bivariate polynomial  $\sigma (\x) := 1 + (x_1^2 - x_2^2)^2$ lies in $\Sigma_4[\x] \subseteq \R_4[\x]$.

At some point, optimization methods based on sums of squares use the implication $p \in \Sigma[\x] \implies \forall \x \in \R^n, \, p(\x) \geq 0$, i.e. the inclusion of $\Sigma[\x]$ in the cone of nonnegative polynomials.

Given $r \in \R[\x]$, one considers the following polynomial minimization problem:
\begin{equation}
\label{eq:minpop}
r^*  :=  \inf_{\x \in \R^n} \, \{ \, r (\x) \, : \, \x \in \K \, \} \enspace,
\end{equation}
%
where the set of constraints $\K \subseteq \R^n$ is defined by
%
\[\K := \{ \x \in \R^{n} : g_1 (\x) \geq 0, \dots, g_k (\x) \geq 0\}\enspace,\]
for polynomial functions $g_1, \dots, g_k$. The set $\K$ is called a {\em basic semialgebraic} set. Membership to semialgebraic sets is ensured by satisfying conjunctions of polynomial nonnegativity constraints. 
%
\begin{remark}
\label{rk:arch}
 When the input variables satisfy interval constraints $\x \in [a_1, b_1] \times \dots \times [a_n, b_n]$ then one can easily show that there exists some integer $M > 0$ such that $M - \sum_{i=1}^n x_i^2 \geq 0$. 
In the sequel, one assumes that this nonnegativity constraint appears explicitly in the definition of $\K$. Such an assumption is mandatory to prove the convergence of semidefinite relaxations stated in Theorem~\ref{th:densesdp}.
\end{remark}
%
In general, the objective function $r$ and the set of constraints $\K$ can be nonconvex, which makes the resolution of Problem~\eqref{eq:minpop} difficult to solve in practice. 
Note that one can rewrite Problem~\eqref{eq:maxpop} as the equivalent maximization problem:
\begin{equation}
\label{eq:maxpop}
r^*  :=  \sup_{\x \in \R^n, \mu \in \R} \{ \, \mu \, : \, r (\x) - \mu \geq 0 \,, \ \x \in \K \, \} \,.
\end{equation}
%
The existence of a sums of squares decomposition $p = \sum_i q_i^2$ valid
over $\R^n$ is ensured by the existence of a symmetric real matrix $Q$, solution of the following linear matrix feasibility problem:
\begin{align}
\label{eq:sdp}
p(\x) = \m_d(\x)^\intercal \, \Q \, \m_d(\x) \,, \quad \forall \x \in \R^n, \,
\end{align}
%
where $\m_d(\x) := (1, x_1, \dots, x_n, x_1^2,x_1 x_2,\dots, x_n^d)$ and the matrix $\Q$ has only nonnegative eigenvalues. Such a matrix $\Q$ is called {\em semidefinite positive}. The vector $\m_d$ and matrix $\Q$ have both a size equal to $s_n^d := \binom{n + d}{d}$. Problem~\eqref{eq:sdp} can be handled with semidefinite programming (SDP) solvers, such as {\sc Mosek}~\cite{mosek} or {\sc SDPA}~\cite{sdpa7} (see~\cite{Vandenberghe94SDP} for specific background about SDP). Then, one computes the ``LDL'' decomposition $\Q = \L^\intercal \D \L$ (variant of the classical Cholesky decomposition), where $\L$ is a lower triangular matrix and $\D$ is a diagonal matrix. Finally, one obtains $p(\x) =  (\L \,
\m_d(\x))^\intercal \, \D \, (\L \, \m_d(\x)) = \sum_{i=0}^{s_n^d} q_i(\x)^2$. Such a decomposition is called a sums of squares (SOS) {\em certificate}.
%
\begin{example}
\label{ex:sdp}
Let define $p(\x) := \frac{1}{4} + x_1^4 - 2 x_1^2 x_2^2 + x_2^4$. With $\m_2 (\x) = (1, x_1, x_2, x_1^2, x_1 x_2, x_2^2)$, one solves the linear matrix feasibility problem $p(\x) = \m_2 (\x)^\intercal \, \Q \, \m_2(\x)$. One can show that the solution writes $\Q = \L^\intercal \D \L$ for a $6 \times 6$ matrix $\L$ and a diagonal matrix $\D$ with entries $(\frac{1}{2},0,0,1,0,0)$, yielding the SOS decomposition: $p(\x) = (\frac{1}{2})^2 + (x_1^2 - x_2^2)^2 =: \sigma(\x)$.
\end{example}
%
\paragraph{Dense SDP relaxations for polynomial optimization.}
We first explain how to obtain tractable appoximations of Problem~\eqref{eq:maxpop}. Define $g_0 := 1$. The hierarchy of SDP relaxations developed by Lasserre \cite{Lasserre01moments} provides lower bounds of $r^*$, through solving the following programs $(\P_d)$:
\[
(\P_d):\left\{			
\begin{array}{rlr}
p_d^\star := \sup\limits_{\sigma_j, \mu} & \mu \enspace, \\			 
\text{s.t.} & r (\x) - \mu = \sum_{j = 0}^{k} \sigma_j(\x) g_j(\x) \,, \ \forall \x \,, \\
\\
& \mu\in \R \,, \sigma_j \in \Sigma[\x] \,, \quad \ \, \quad  \ j = 0,\dots,k \,, \\
\\
& \deg (\sigma_j g_j) \leq  2 d,             \quad \ \, \, \qquad  j = 0,\dots,k \,.\\
\end{array} \right.
\]
%
The next theorem is a consequence of the assumption mentioned in Remark~\ref{rk:arch}.
\begin{theorem}[Lasserre~\cite{Lasserre01moments}]
\label{th:densesdp}
Let $p_d^{\star}$ be the optimal value of the sparse SDP relaxation~$(\P_d)$.
Then, the sequence of optimal values $(p_d^\star)_{d \in \N}$ is nondecreasing and converges to $r^\star$.
\end{theorem}
%
The size of the truncated SDP variables
grows polynomially with the SDP-relaxation order $d$.
Indeed, at fixed $n$, the relaxation~$(\P_d)$ involves $O((2 d)^{n})$ SDP
variables and $(k + 1)$ linear matrix inequalities (LMIs) of size
$O(d^n)$. When $d$ increases, then more accurate lower bounds of $r^\star$ can be obtained, at an increasing computational cost.
At fixed $d$,  the relaxation $(\P_d)$ involves $O(n^{2d})$ SDP variables and $(d + 1)$ linear matrix inequalities (LMIs) of size
$O(n^{d})$.

There are several ways to decrease the size of the SDP problems. 
First, symmetries in SDP relaxations for polynomial optimization problems can be exploited to replace one SDP problem~$(\P_d)$ by
several smaller SDPs~\cite{Riener2013SymmetricSDP}. Notice that it is possible only if the multivariate polynomials of the initial problem are invariant under the action of a finite subgroup $G$ of the group $GL_{n}(\R)$. 
%

\paragraph{Exploiting sparsity.} Here we describe how to exploit the structured sparsity of the
problem to replace one SDP problem~$(\P_d)$ by an SDP problem~$(\S_d)$ of
size $O (\kappa^ {2 d})$ where $\kappa$ is the average size
of the maximal cliques of the correlation pattern of the polynomial
variables (see~\cite{Waki06SparseSOS}). We now present these notions as well as the formulation of sparse SDP relaxations~$(\S_d)$.

We note $\N^n$ the set of $n$-tuple of nonnegative integers. The support of a polynomial $r(\x) := \sum_{\alphab \in \N^n} r_{\alphab} \x^{\alphab}$ is defined as $\suppf{r} := \{ \, \alphab \in \N^n \, : \, r_{\alphab} \neq 0 \, \}$. For instance the support of $p(\x) := \frac{1}{4} + x_1^4 - 2 x_1^2 x_2^2 + x_2^4$ is $\suppf{p} = \{ \, (0,0), (4, 0), (2,2), (0,4) \, \}$.

Let $F_j$ be the index set of variables which are involved in the polynomial $g_j$, for each $j=1, \dots, k$.
The correlative sparsity is represented by the 
$n \times n$ correlative sparsity matrix (csp matrix) $\mathbf{R}$ defined by:
\begin{equation*}
\label{eq:csp}
\mathbf{R}(i, j) := \left \{
\begin{array}{ll}
  1 & \text{ if }  i = j \enspace, \\
  1 & \text{ if }  \exists \alphab \in \suppf{f} \text{ such that } \alpha_i, \alpha_j \geq 1 \,, \\
  1 & \text{ if }  \exists k \in \{1, \dots, m\} \text{ such that } i, j \in F_k  \,,\\
  0 & \text{otherwise .} 
\end{array} \right.
\end{equation*}

We define the undirected csp graph $G(N, E)$ with
 $N = \{ 1, \dots, n \}$ and $E = \{\{i, j\} : i, j \in N , \ i < j , \mathbf{R}(i, j) = 1 \}$. 
Then, let $C_1,\dots, C_m \subseteq N$ denote the maximal cliques of $G(N, E)$ and 
%and define the sets of supports: 
%\[\mons{d}{C_q} := \{ \alphab \in  \mons{d}{n} : \alpha_i = 0 \text{ if } i %\notin C_q \}, \ (q=1 ,\dots,l)\enspace. \]
 define $n_j := \#C_j$, for each $j=1 ,\dots,m$.

\begin{remark}
\label{rk:sparsearch}
From the assumption of Remark~\ref{rk:arch}, one can add the $m$ redundant additional constraints:
\begin{equation}
\label{eq:assum_sos_sparse}
g_{k + j} := n_j M^2 - \sum_{i \in C_j} {x_i^2} \geq 0\,, \  j=1 ,\dots, m \,,
\end{equation}
set $k' = k + m$, define the compact semialgebraic set:
\[\K' := \{\, \x \in \R^n \, : \, g_1 (\x) \geq 0, \dots, g_{k'} (\x) \geq 0 \,\} \,,\]
and modify Problem~\eqref{eq:minpop} into the following optimization problem:
\begin{equation}
\label{eq:sparseminpop}
r^*  :=  \inf_{\x \in \R^n} \, \{ \, r (\x) \, : \, \x \in \K' \, \} \,.
\end{equation}
\end{remark}
%Notice that the sums of squares of polynomials that belong to $\Sigma [\x, \mons{d}{C_q}]$ only involve variables $x_i (i \in C_q)$.
For each $j=1 ,\dots,m$, we note $\R_{2 d}[\x, C_j]$ the set of polynomials of $\R_{2 d}[\x]$ which involve the variables $(x_i)_{i \in C_j}$. We note $\Sigma [\x, C_j] := \Sigma [\x] \bigcap \R_{2 d}[\x, C_j]$. Similarly, we define $\Sigma [\x, F_j]$, for each $j=1, \dots, k$.

The following program is the sparse variant of the SDP program $(\P_d)$:
\[
(\S_d):\left\{			
\begin{array}{rl}
r_d^\star := \sup\limits_{\mu, \sigma_j} & \mu\enspace, \\	 
\text{s.t.} & r (\x) - \mu = \sum_{j = 0}^{k'} \sigma_j(\x) g_j(\x) \,, \ \forall \x \,, \\
\\
& \mu\in \R \,,\  \sigma_0 \in \sum_{j = 1}^m \Sigma [\x, C_j] \,, \\
\\
& \sigma_j \in \Sigma[\x, F_j]  \,,\ j = 1,\dots,k' \,, \\
\\
& \deg (\sigma_j g_j) \leq 2 d  \,,\ j = 0,\dots,k' \,,
\end{array} \right.
\]
%
where $\sigma \in \sum_{j = 1}^m \Sigma [\x, C_j]$ if and only if there exist $\sigma^1 \in \Sigma[\x, C_1], \dots, \sigma^m \in \Sigma[\x,C_m]$ such that $\sigma (\x) = \sum_{j = 1}^m \sigma^j (\x)$, for all $\x \in \R^n$.
%

The number of SDP variables of the relaxation~$(\S_d)$ is $\sum_{j=1}^m \binom{n_j + 2 d}{2 d}$. At fixed $d$, it yields an SDP problem with $O(\kappa^{2d})$ variables, where $\kappa := \frac{1}{m} \sum_{j=1}^m n_j$ is the average size of the cliques $C_1, \dots, C_m$.
%
Moreover, the cliques $C_1, \dots, C_m$ satisfy the running intersection property: 
%
\begin{definition}[RIP]
\label{def:rip}
Let $l \in \N_0$  and $I_1, \dots, I_m$ be subsets of $\{1, \dots, n\}$. We say that $I_1, \dots, I_m$ satisfy the running intersection property (RIP) when for all $i=1, \dots, m$, there exists $k < i$ such that $I_i \bigcap (\bigcup_{j < i} I_j) \subseteq I_m$.
%\[ \forall i = 2, \dots, r, (\exists k < i \,, \ (I_i \bigcap \bigcup_{j < i} I_j) \subset I_k)\enspace. \]
\end{definition}
This RIP property together with the assumption mentioned in Remark~\ref{rk:sparsearch} allow to state the sparse variant of Theorem~\ref{th:densesdp}:
%The optimal values of $\r_d$ converge to the global minimum $f^*$, as a corollary of~\cite[Theorem 3.6]{Las06SparseSOS}.
%
\begin{thm}[\protect{Lasserre~\cite[Theorem 3.6]{Las06SparseSOS}}]
\label{th:sparsesdp}
Let $r_d^{\star}$ be the optimal value of the sparse SDP relaxation~$(\S_d)$. Then the sequence $(r_d^{\star})_{d \in \N}$ is nondecreasing and converges to $r^\star$.
\end{thm}
The interested reader can find more details in~\cite{Waki06SparseSOS} about additional ways to exploit sparsity in order to derive analogous sparse SDP relaxations.
We illustrate the benefits of the SDP relaxations~$(\S_d)$ with the following example:
\begin{example}
\label{ex:sparse}
Consider the polynomial $f$ mentioned in Section~\ref{sec:intro}:
$f(\x) := x_2 x_5 + x_3 x_6 - x_2 x_3  - x_5 x_6 
+ x_1 ( - x_1 +  x_2 +  x_3  - x_4 +  x_5 +  x_6)$.
%
Here, $n = 6, d = 2, N = \{1,\dots, 6 \}$. The $6 \times 6$ correlative sparsity matrix $\mathbf{R}$ is:
\[
\mathbf{R} = 
\begin{pmatrix}
  1 & 1 & 1 & 1 & 1 & 1 \\
  1 & 1 & 1 & 0 & 1 & 0 \\
  1 & 1 & 1 & 0 & 1 & 1 \\
  1 & 0 & 0 & 1 & 0 & 0 \\
  1 & 1 & 1 & 0 & 1 & 1 \\
  1 & 0 & 1 & 0 & 1 & 1 
 \end{pmatrix}
\]
The csp graph $G$ associated to $\mathbf{R}$ is depicted in Figure~\ref{fig:csp_deltax}. 
%
\begin{figure}[!ht]	
\begin{center}
\includegraphics[scale=0.7]{csp_deltax.pdf}
\caption{Correlative sparsity pattern graph for the variables of $f$}
\label{fig:csp_deltax}
\end{center}
\end{figure}
%
The maximal cliques of $G$ are $C_1 := \{1, 4\}$, $C_2 := \{1, 2, 3, 5\}$ and $C_3 := \{1, 3, 5, 6\}$. For $d=2$, the dense SDP relaxation~$(\P_2)$ involves $\binom{6 + 4}{4} = 210$ variables against $\binom{2 + 4}{4} + 2 \binom{4 + 4}{4} = 115$ for the sparse variant~$(\S_2)$. This difference becomes even more significant while comparing the dense SDP relaxation~$(\P_3)$ involving 924 variables against 448 for the sparse variant~$(\S_3)$.
\end{example}
%
\subsection{Computer assisted proofs for polynomial optimization}
\label{sec:coqbackground}
Here, we briefly recall some existing features of the $\coq$ proof assistant to handle formal polynomial optimization.
%about the mechanisms of formal polynomial optimization within the $\coq$ proof assistant.
For more details on the $\coq$ system, we recommend the
documentation available in~\cite{bertot2004interactive}.
Giving a polynomial $r$ and a set of constraints $\K$, one can obtain a lower bound of $r$ by solving any instance of Problem~$(\P_d)$. Then, one can verify formally the correctness of the lower bound $r_d^\star$, using the SOS certificate output $\sigma_0, \dots, \sigma_k$. Indeed it is enough to prove the polynomial equality $r(\x) - r_d^\star = \sum_{j=0}^k \sigma_j(\x) g_j(\x)$ inside $\coq$. Such equalities can be efficiently proved using $\coq$'s ring tactic~\cite{ring05} via the mechanism of computational reflection~\cite{Boutin97usingreflection}.
For the sake of clarity, let consider the unconstrained case, i.e.~$\K = \R^n$. One encodes an SOS certificate $\sigma_0(\x) = \sum_{i=1}^m q_i^2$  with the sequence of polynomials $[q_1; \dots; q_m]$, with each $q_i$ being of type \code{polC} (see Section~\ref{sec:fpbackground}). To prove the equality $r = \sigma_0$, our version of ring's tactic normalizes both $r$ and the sequence $[q_1; \dots; q_m]$ and compares the two normalization results. This mechanism is illustrated in Figure~\ref{fig:reflexion} with the polynomial $p := \frac{1}{4} + x_1^4 - 2 x_1^2 x_2^2 + x_2^4$ (see Example~\ref{ex:sdp}) being encoded by \code{p : polC}  and the polynomials $1/2$ and $x_1^2 - x_2^2$ being encoded respectively by $\mathtt{q_1}$ and $\mathtt{q_2}$. 

\if{
In the general case, one applies a correctness lemma:
\begin{lstlisting}
Lemma correct_pop env $r$ cert_pop $ $:  
g_nonneg env g $\to$ checker g $\fpop$ $\mu_k^-$ cert_pop  $\eqcoq$ true $\to$ 
$\mu_k^-$ $\leq$ $\evalexpr{\fpop}$.
\end{lstlisting}
}\fi
%
\begin{figure}[!ht]
\centering
\includegraphics[scale=0.75]{reflexion.pdf}
\caption{An illustration of computational reflection}	
\label{fig:reflexion}
\end{figure}
%
In the general case, this computational step is done through a \code{checker_sos} procedure which returns a boolean value. If this value is true, one applies a correctness lemma, whose conclusion yields the nonnegativity of $r - r_d^\star$ over $\K$.
In practice, the SDP solvers are implemented in floating-point precision, thus the above equality between $r - r_d^\star$ and the SOS certificate does not hold. However, following Remark~\ref{rk:arch}, each variable lies in a closed interval, thus one can bound the remainder polynomial $\epsilon(\x) := r(\x) - r_d^\star - \sum_{j=0}^k \sigma_j(\x) g_j(\x)$ using basic interval arithmetic, so that the lower bound $\epsilon^\star$ of $\epsilon$ yields the valid inequality: $\forall \x \in \K, r(\x) \geq r_d^\star + \epsilon^\star$.
For more explanations, we refer the interested reader to the formal framework developed in~\cite{jfr14}. Note that this formal verification remains valid when considering the sparse variant~$(\S_d)$.
%
\section{Guaranteed Roundoff Error Bounds using SDP Relaxations}
\label{sec:fpsdp}
In this section, we present our new algorithm to bound rounding errors of nonlinear programs, relying on sparse SDP relaxations. We first explain how to handle polynomial programs and then present extensions to the non-polynomial case.
\subsection{Polynomial Programs}
Here we consider a given program which implements a polynomial expression $f$ with input variables $\x$ satisfying a set of constraints $\X$. We assume that  $\X$ is included in a box (i.e.~a product of closed intervals) $[\a, \b] := [a_1, b_1] \times \dots \times [a_n, b_n]$ and that $\X$ is encoded as follows: 
\[ 
\X := \{\, \x \in \R^n \, : \, g_1 (\x) \geq 0, \dots, g_{k} (\x) \geq 0 \,\} \,,
\]
for polynomial functions $g_1, \dots, g_k$. 
%
Then, we denote by $\hat{f}(\x,\e)$ the rounded expression of $f$ after applying the ~\lstinline|rounding| procedure (see Section~\ref{sec:fpbackground}), introducing the introduction of error variables $\e$. 

The algorithm \code{bound_poly}, depicted in Figure~\ref{alg:bound_poly}, takes as input $\x$, $\X$, $f$, $\hat{f}$, $\e$ as well as the set $\E$ of bound constraints over $\e$. For a given machine $\epsilon$, one has $\E := [-\epsilon, \epsilon]^m$, with $m$ being the number of error variables. This algorithm actually relies on the sparse SDP optimization procedure $(\S_d)$ (see Section~\ref{sec:sdpbackground} for more details), thus~\code{bound_poly} also takes as input a relaxation order $d \in \N$. The algorithm provides as output an upper bound of the maximal absolute error bound $r^\star := \max_{(\x,\e)\in \K} \mid \hat{f}(\x,\e) - f(\x) \mid $.

After computing the rounding error polynomial $r := \hat{f} - f$ (Line~\lineref{line:r}), one decomposes $r$ as the sum of a polynomial $l$ which is linear w.r.t.~the error variable $\e$ and a remainder polynomial $h$ (Line~\lineref{line:h}). One way to obtain $l$ is to compute the vector of partial derivatives of $r$ w.r.t.~$\e$ evaluated at $(\x, 0)$ and finally to take the inner product of this vector and $\e$ (Line~\lineref{line:l}). Then, the idea is to compute a precise bound of $l$ and a coarse bound of $h$. The underlying reason is that $h$ involves error term products of degree greater than 2 (e.g.~$e_1 e_2$), yielding an interval enclosure $I^h$ of a-priori much smaller width, compared to the interval enclosure $I^l$ of $l$. One obtains $I^h$ using the procedure $\iabound$ implementing basic interval arithmetic (Line~\lineref{line:iabound}). The bound of $l$ is provided through solving two sparse SDP instances of Problem~$(\S_d)$, at relaxation order $d$. We now give more explanations about this optimization procedure. 
%

\begin{figure}[!ht]
\begin{algorithmic}[1]                    
\Require input variables $\x$, input constraints $\X$, real polynomial $f(\x)$, rounded polynomial $\hat{f}(\x, \e)$, error variables $\e$, error constraints $\E$, relaxation order $d$
\Ensure upper bound $f_d$ of the absolute error $\mid \hat{f} - f  \mid$ over $\K := \X \times \E$
\State Define the error polynomial $r(\x, \e) := \hat{f}(\x,\e) - f(\x)$ \label{line:r}
\State Compute $l(\x,\e) := \sum_{j=1}^m \frac{\partial r(\x,\e)} {\partial e_j} (\x,0) \, e_j$ \label{line:l}
\State Compute $h := r - l$ \label{line:h}
\State Compute interval bounds for $h$: $I^h := \iaboundfun{h}{\K}$ \label{line:iabound}
\State Compute interval bounds for $l$: $I_d^l := \sdpboundfun{l}{\K}{d}$  \label{line:sdpbound}
\State Compute $I_d := I_d^l + I^h = [\underline{f_d}, \overline{f_d}]$ 
\State \Return $f_d := \max \{- \underline{f_d}, \overline{f_d} \}$
\end{algorithmic}
\caption{\code{bound_poly}}
\label{alg:bound_poly}
\end{figure}
%
\paragraph{Optimization procedure $\sdpbound$.}
First, note that one can map each input variables $x_i$ to the integer $i$, for all $i=1,\dots,n$, as well as each error variable $e_j$ to $n+j$, for all $j=1,\dots,m$. Then, define the sets $C_1 := {1,\dots,n,n+1}, \dots, C_m := {1,\dots,n,n+m}$. Here, we take advantage of the sparsity correlation pattern of $l$ by using $m$ distinct sets of cardinal $n+1$ rather than a single one of cardinal $n+m$, i.e.~the total number of variables. 
We also scale the optimization problems by writing 
\begin{align}
\label{eq:lscale}
l(\x,\e) = \sum_{j=1}^m s_j (\x) e_j = \epsilon \sum_{j=1}^m s_j (\x) \frac{e_j}{\epsilon} \,,
\end{align}
%
with $s_j(\x) := \frac{\partial r(\x,\e)} {\partial e_j} (\x,0)$, for all $j=1,\dots,m$. It leads to compute an interval enclosure of $l/\epsilon$ over $\K' := \X \times [-1, 1]^m$.
Recall that from Remark~\ref{rk:arch}, there exists an integer $M > 0$ such that $M - \sum_{i=1}^n x_i^2 \geq 0$, as the input variable satisfy box constraints.
Moreover, to fulfil the assumption of Remark~\ref{rk:sparsearch},  one encodes $\K'$ as follows: 
\begin{align*}
\K' := \{\, (\x,\e) \in \R^{n+m} \, : \, g_1 (\x) \geq 0, \dots, g_k(\x) \geq 0 \,, \\
g_{k+1}(\x,e_1) \geq 0, \dots, g_{k+m} (\x, e_m) \geq 0 \,\} \,,
\end{align*}
%
with $g_{k+j}(\x, e_j) := M + 1 -  \sum_{i=1}^n x_i^2 + e_j^2$, for all $j=1,\dots, m$. 
The index set of variables involved in $g_j$ is $F_j := N = \{1, \dots, n\}$ for all $j=1, \dots, k$. 
The index set of variables involved in $g_{k+j}$ is $F_j := C_j$ for all $j=1, \dots, m$. 

Then, one can compute a lower bound of the minimum of $l'(\x,\e) := l(\x, \e) / \epsilon$ over $\K'$ by solving the following optimization problem:
%
\begin{align}
\begin{split}
\label{eq:lscalesdp1}
\left\{			
\begin{array}{rl}
\underline{l_d'} := \sup\limits_{\mu, \sigma_j} & \mu\enspace, \\	 
\text{s.t.} & l' - \mu = \sigma_0 + \sum_{j = 0}^{k+m} \sigma_j g_j \,, \\
\\
& \mu\in \R \,,\ \sigma_0 \in \sum_{j = 1}^m \Sigma [(\x, \e), C_j] \,, \\
\\
& \sigma_j \in \Sigma[(\x,\e), F_j] \,, \ j = 1,\dots,k+m \,, \\
\\
& \deg (\sigma_j g_j) \leq 2 d  \,, \ j = 0,\dots,k+m \,.
\end{array} \right.
\end{split}
\end{align}
%
A feasible solution of Problem~\eqref{eq:lscalesdp1} ensures the existence of $\sigma^1 \in \Sigma[(\x,e_1)], \dots, \sigma^m \in \Sigma[(\x,e_m)]$ such that $\sigma_0 = \sum_{j=0}^m \sigma^j$, allowing the following reformulation:
%
\begin{align}
\begin{split}
\label{eq:lscalesdp2}
\left\{			
\begin{array}{rl}
\underline{l_d'} := \sup\limits_{\mu, \sigma_j} & \mu\enspace, \\	
\text{s.t.} & l' - \mu = \sum_{j=1}^m \sigma^j + \sum_{j = 1}^{k+m} \sigma_j g_j \,, \\
\\
& \mu\in \R \,, \ \sigma_j \in \Sigma[\x] \,, \ j = 1,\dots,m \,, \\
\\
& \sigma^j  \in \Sigma [(\x, e_j)] \,,  \deg (\sigma^j) \leq 2 d  \,, \ j = 1,\dots,m \,, \\
\\
&  \quad \deg (\sigma_j g_j) \leq 2 d  \,, \ j = 1,\dots,k+m \,.
\end{array} \right.
\end{split}
\end{align}
%
An upper bound $\overline{l_d'}$ can be obtained by replacing $\sup$ with $\inf$ and $l' - \mu$ by $\mu - l'$ in Problem~\eqref{eq:lscalesdp2}.
Our optimization procedure $\sdpbound$ computes the lower bound $\underline{l_d'}$ as well as an upper bound $\overline{l_d'}$ of $l'$ over $\K'$ then returns the interval $I_d^l := [\epsilon \, \underline{l_d'}, \epsilon \, \overline{l_d'}] $, which is a sound enclosure of the values of $l$ over $\K$.
%
\begin{remark}
We emphasize the importance of scaling the error variables. First, it allows to consider a polynomial optimization problem in $\K' := \X \times [-1, 1]$, ensuring that the range of input variables does not significantly differs from the range of error variables, which is mandatory when considering SDP relaxations. Indeed, most SDP solvers (e.g.~{\sc Mosek}, {\sc SDPA}) are implemented using double floating-point precision. Hence, when the maximal value $\epsilon$ of error variables is less than $2^{-53}$, these solvers will treat each error variable term as 0, and consequently $l$ as the zero polynomial.
In addition, 
\end{remark}
\paragraph{Convergence of the SDP hierarchy of error bounds.}

\subsection{Non-polynomial Programs}

Let $f : \S_f \to \R$ defined by $f(\x) := x_1 + \frac{x_2}{1 + x_3}$, for all $\x \in \S_f$. \\
To perform semialgebraic optimization via sums of squares, one usually introduces a lifting variable  $z$ to represent the division. Let $\S_g := \{(\x,z) : \x \in \S_f\,, \ z (1 + x_3) = 1  \}$.
%One can compute an interval enclosure $I_z$ of $z := 1 + x_3$ over $\S_f$.
%Let $\S_g := \S_f \times I_z$.
Then, one has $f(\x) = g(\x,z) := x_1 + x_2 z$, for all $(\x,z) \in \S_g$. \\
Here things are slightly different as we need to take into account the floating point representation of the denominator.\\
The floating point representation of $f$ is $\hat{f}(\x,\epsilonb) := [x_1 + \frac{x_2}{(1 + x_3)(1 + \epsilon_3)}(1 + \epsilon_2)](1 + \epsilon_1)$, for all $\x \in \S_f, \epsilonb \in \B$.
Then, one introduces a second lifting variable $\hat{z}$ and 
\[\K_f := \{(\x, \epsilonb, \hat{z}) : \x \in \S_f \,, \ \epsilonb \in \B \,, \ \hat{z} (1 + x_3) (1 + \epsilon_3) = 1  \} \,. \]
One has $\hat{f}(\x,\epsilonb) = \hat{g}(\x,\epsilonb,\hat{z}) := [x_1 + x_2 \hat{z}(1+\epsilon_2) ](1+\epsilon_1)$, for all $(\x, \epsilonb, \hat{z}) \in \K_f$. 
To bound the error $\hat{f} -f$, one optimizes the function $\hat{g} - g$ over the set 
\[\K_g := \{(\x, z, \epsilonb, \hat{z}) : (\x,z) \in \S_g \,, \ \epsilonb \in \B \,, \ \hat{z} (1 + x_3) (1 + \epsilon_3) = 1  \} \,. \]
\\
Here, one can actually reduce the number of variables of the problem, noticing that $z = \hat{z} (1 + \epsilon_3)$. However, in general one can not write $z$ as a polynomial of $\hat{z}$, $\x$ and $\epsilonb$ and one needs to consider two lifting variables for each non-polynomial operation involved in $f$.
%
%\subsection{Transcendental programs}

\paragraph{Simplification of error term.}
%
In practice, the algorithm integrates features to save error variables. First, it memorizes all sub-expressions of the nonlinear expression tree to associate  the same error variable to the same operation when the operands are identical. For instance, several occurrences of $x_1 + x_2$ are replaced each time by $(x_1 + x_2) (1 + e_1)$.
We can also simplify error term products, thanks to the following lemma.
\begin{lemma}[\protect{Higham~\cite[Lemma 3.3]{higham2002accuracy}}]
\label{th:redproduct}
Let $\epsilon < \frac{1}{k}$ and $\gamma_k := \frac{k \epsilon}{1 - k \epsilon}$. Then, for all $e_1, \dots, e_k \in [-\epsilon, \epsilon]$, there exists $\theta_k$ such that ${\displaystyle \prod_{i=1}^k (1 + e_i) = 1 + \theta_k}$ and $\mid \theta_k \mid \leq \gamma_k$.
\end{lemma}
Lemma~\ref{th:redproduct} implies that for any $k$ such that $\epsilon < \frac{1}{k}$, one has $\gamma_k \leq (k + 1) \epsilon$. Hence, one can derive safe over-approximations of the absolute rounding error while introducing only one variable $e_1$ (bounded by $(k + 1) \epsilon$) instead of $k$ error variables $e_1, \dots, e_k$ (bounded by $\epsilon$). The cost of solving the corresponding optimization problem can be significantly reduced but it yields coarser error bounds.
%
\section{Implementation Benchmarks} %4p
\label{sec:benchs}
\begin{table*}[!ht]
%\small
\begin{center}
\caption{Comparison of error bounds $r_k^*$ obtained with different methods}
\begin{tabular}{p{2.3cm}lccccc}
\hline
\multirow{2}{*}{Benchmark} & \multirow{2}{*}{precision} & \multirow{2}{*}{$\realtofloat$} & $\rosa$  & $\fptaylor$  &\multirow{2}{*}{IA} & \multirow{2}{*}{Simulated error}
\\
& & & \cite{Darulova14Popl} & \cite{fptaylor15} & & \\
\hline            
%\multirow{2}{*}{Method 1} & $n^{(1)}$/$m^{(1)}$ &  $40/30$ & $212/111$ & $1039/350 $ & $4211/915$ & $130768/1991$ & $40251/3822$ \\
\multirow{1}{*}{doppler1$\star$}
& (double) & $5.05\text{e--}05$ & $2.36\text{e--}06$ & $6.40\text{e--}07$ & $3.95\text{e+}02$ & $5.97\text{e--}07$\\
\multirow{1}{*}{doppler1}
& (double) & $2.85\text{e--}13$ & $4.97\text{e--}13$ & $1.57\text{e--}13$ & $3.95\text{e+}02$ & $7.11\text{e--}14$\\
\multirow{1}{*}{doppler2}
& (double) & $1.02\text{e--}12$ & $1.29\text{e--}12$ & $2.87\text{e--}13$ & $\nan$ & $1.14\text{e--}13$\\
\multirow{1}{*}{doppler3}
& (double) & $1.45\text{e--}13$ & $2.03\text{e--}13$ & $8.16\text{e--}14$ & $1.09\text{e+}02$ & $4.27\text{e--}14$\\
\multirow{1}{*}{rigidBody1}
& (double) & $3.55\text{e--}13$ & $5.08\text{e--}13$ & $3.87\text{e--}13$ & $3.55\text{e--}13$ & $2.28\text{e--}13$\\
\multirow{1}{*}{rigidBody2}
& (double) & $3.98\text{e--}11$ & $6.48\text{e--}11$ & $5.24\text{e--}11$ & $3.98\text{e--}11$ & $2.19\text{e--}11$\\
\multirow{1}{*}{verhulst}
& (double) & $3.40\text{e--}16$ & $6.82\text{e--}16$ & $3.50\text{e--}16$ & $7.86\text{e--}01$ & $2.23\text{e--}16$\\
\multirow{1}{*}{kepler1}
& (double) & $2.96\text{e--}13$ & ? & $4.49\text{e--}13$ & $1.67\text{e--}12$ & $5.\text{e--}14$\\
\hline
\multirow{2}{*}{sineTaylor}
& (double) & $5.08\text{e--}16$ & $9.57\text{e--}16$ & $6.71\text{e--}16$ & $9.39\text{e--}16$ & $4.45\text{e--}16$\\
& (float) & $2.73\text{e--}07$ & $1.03\text{e--}06$ & $3.51\text{e--}07$ & $5.07\text{e--}07$ & $1.79\text{e--}07$\\
%& (16bit) & $2.24\text{e--}3$ & $2.87\text{e--}4$ & ? & $1.55\text{e--}4$\\
\hline
\multirow{2}{*}{sineOrder3}
& (double) & $6.53\text{e--}16$ & $1.11\text{e--}15$ & $9.96\text{e--}16$ & $8.82\text{e--}16$ & $3.34\text{e--}16$\\
& (float) & $3.51\text{e--}07$ & $1.19\text{e--}06$ & $5.35\text{e--}07$ & $4.74\text{e--}07$ & $2.12\text{e--}07$\\
\hline
\multirow{2}{*}{sqroot}
& (double) & $7.56\text{e--}16$ & $8.41\text{e--}16$ & $7.87\text{e--}16$ & $8.48\text{e--}16$ & $4.45\text{e--}16$\\
& (float) & $4.06\text{e--}07$ & $9.03\text{e--}07$ & $4.23\text{e--}07$ & $4.56\text{e--}07$ & $2.45\text{e--}07$\\
%& (16bit) & $3.33\text{e--}3$ & $5.97\text{e--}4$ & ? & $1.58\text{e--}4$\\
\hline
\end{tabular}
\label{table:error}
\end{center}
\end{table*}


However, $\fptaylor$ implements special cases to eliminate some error terms, for instance when $\op$ is the multiplication and one of the operands is a nonnegative power of two, then the error $e$ is set to zero. 

{\scriptsize
\begin{lstlisting}
let box_doppler1 u v T = [$(-100, 100); (20, 2e4);(-30, 50)$];; 
let obj_doppler1 u v T = [(
let $t_1 = 331.4 + 0.6 * T$ in 
$-t_1*v/((t_1 + u)*(t_1 + u))$, $3e-13$)];;
\end{lstlisting}
}
{\scriptsize
\begin{lstlisting}
procedure doppler2(u : real, v : real, T : real) returns (r : real) {
 assume (-125.0 <= u && u <= 125.0 && 15.0 <= v && v <= 25000.0 && -40.0 <= T && T <= 60.0);
  var t1 := 331.4 + 0.6 * T;
  r := -t1*v/((t1 + u)*(t1 + u));
}
\end{lstlisting}
}
{\scriptsize
\begin{lstlisting}
procedure doppler3(u : real, v : real, T : real) returns (r : real) {
 assume (-30.0 <= u && u <= 120.0 && 320.0 <= v && v <= 20300.0 && -50.0 <= T && T <= 30.0);
  var t1 := 331.4 + 0.6 * T;
  r := -t1*v/((t1 + u)*(t1 + u));
}
\end{lstlisting}
}
{\scriptsize
\begin{lstlisting}
procedure rigidBody1(x1 : real, x2 : real, x3 : real) returns (r : real) {
 assume (-15.0 <= x1 && x1 <= 15.0 && -15.0 <= x2 && x2 <= 15.0 && -15.0 <= x3 && x3 <= 15.0);
  r := -x1*x2 - 2.0 * x2 * x3 - x1 - x3;
}
\end{lstlisting}
}
{\scriptsize
\begin{lstlisting}
procedure rigidBody2(x1 : real, x2 : real, x3 : real) returns (r : real) {
 assume (-15.0 <= x1 && x1 <= 15.0 && -15.0 <= x2 && x2 <= 15.0 && -15.0 <= x3 && x3 <= 15.0);
  r := 2.0*x1*x2*x3 + 3.0*x3*x3 - x2*x1*x2*x3 + 3.0*x3*x3 - x2;
}
\end{lstlisting}
}
{\scriptsize
\begin{lstlisting}
procedure sineTaylor(x : real) returns (r : real) {
 assume (-1.57079632679  <= x && x <= 1.57079632679);
    r := x - (x*x*x)/6.0 + (x*x*x*x*x)/120.0 - (x*x*x*x*x*x*x)/5040.0 ;
}
\end{lstlisting}
}
{\scriptsize
\begin{lstlisting}
procedure sineOrder3(x : real) returns (r : real) {
 assume (-2  <= x && x <= 2);
  r := 0.954929658551372 * x - 0.12900613773279798*x*x*x ;
}
\end{lstlisting}
}
{\scriptsize
\begin{lstlisting}
procedure sqroot(x : real) returns (r : real) {
 assume (0  <= x && x <= 1);
  r := 1.0 + 0.5*x - 0.125*x*x + 0.0625*x*x*x - 0.0390625*x*x*x*x;
}
\end{lstlisting}
}
\section{Related Works} % 1.5p
%
SMT solvers allow to analyze programs with various semantics or specifications but are limited for the manipulation of problems involving nonlinear arithmetics. When SAT/SMT solvers output proof witnesses, they can be formally rechecked inside the $\coq$ proof assistant~\cite{smtcoq}. While ensuring soundness, the efficiency of the procedure is not compromised due to tactics enjoying the mechanism of computational reflection.

\section{Conclusion and Prospectives} %0.5p
%
We have presented a verification framework to over-approximate round-off errors occurring while executing nonlinear programs implemented with finite precision.
The framework relies on semidefinite optimization, ensuring tight and certified approximations. Our approach extends to medium-size nonlinear problems, due to  automatic detection of the correlation sparsity pattern of input variables and round-off error variables.

\appendix
\if{
\section{Appendix Title}

This is the text of the appendix, if you need one.
}\fi
\acks
This work was partly funded by the Engineering and Physical Sciences Research Council (EPSRC) Challenging Engineering Grant (EP/I020457/1).

% We recommend abbrvnat bibliography style.

\bibliographystyle{abbrvnat}
\bibliography{roundsdp}
% The bibliography should be embedded for final submission.

\if{
\begin{thebibliography}{}
\softraggedright

\bibitem[Smith et~al.(2009)Smith, Jones]{smith02}
P. Q. Smith, and X. Y. Jones. ...reference text...

\end{thebibliography}
}\fi

\end{document}

%                       Revision History
%                       -------- -------
%  Date         Person  Ver.    Change
%  ----         ------  ----    ------

%  2013.06.29   TU      0.1--4  comments on permission/copyright notices

